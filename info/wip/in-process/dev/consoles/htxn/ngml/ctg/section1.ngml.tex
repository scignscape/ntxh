
\section{Cognitive Transform Grammar and Transform-Pairs}
\label{s1}

\p{The idea that inter-word pairs are a foundational linguistic 
unit {\mdash} from which larger aggregates can be built up recursively 
{\mdash} is an central tenet of Dependency Grammar.  Here I will 
generalize this perspective outside (but not excluding) 
grammar, to overall semantic, pragmatic, and even 
extralinguistic relations indicated via interword relations.}

\p{In some cases word-relations can still be theorized mostly via 
syntax.  Consider hypothetical, example sentences like

\begin{sentenceList}

\sentenceItem{} \swl{itm:having}{His having lied in the past damages his credibilty in the present.}{syn}
\sentenceItem{} \swl{itm:whether}{Voters question whether he is truthful this time around.}{syn}
\end{sentenceList}

In (\ref{itm:having}), \i{having} is necessary to syntactically transform its ground
\i{lied} from a verb-form to a noun (something which can be inserted into a 
possessive clause).  Analogously, in (\ref{itm:whether}) \i{whether} modifies
\i{is} (since this is the head of a subordinate clause), wrapping a propositional 
clause into a noun so that it furnishes a direct object to the verb
\i{question}.  The essential transformation in these cases is 
motivated by grammatic considerations, particularly part-of-speech: 
a verb and a subordinate, propositionally complete clause (in (\ref{itm:having}) and 
(\ref{itm:whether}), respectively) need (for syntactic propriety) to be 
modified so as to play a role in a site where a noun is expected 
(in effect, they need to be bundled into a noun-phrase).}

\p{The relevant transforms here {\mdash} signified by \i{having} and \i{whether} {\mdash} have a 
semantic dimension also, and we can speculate that the syntactic 
rules (requiring a verb or propositional-clause to be transformed into a 
noun) are actually driven by semantic considerations.  Conceptually, 
for example, \i{his having lied} packages a verb into a possessive 
context because the sentence is not foregrounding a specific lying-event 
but rather the fact of the existence of such occasions.  We cannot perhaps
\q{possess} an event, but we possess (as part of our nature or history) 
the fact of past occurrences, viz., events in the form of things we
have done.  In this sense \i{his having lied} marks a conceptual transformation, 
from events qua occurrents to events (as factical givens) qua states or
possessions, and the grammatical norm {\mdash} how we cannot just say
\q{his lied} {\mdash} is epiphenomenal to the conceptual logic here; 
the erroneous \q{his lied} sounds flawed because it does not 
match a coherent conceptual pattern in how events and states fit together.  
But, still, the syntactic requirement {\mdash} the expectation 
that a noun or noun-phrase serve as the ground of a possessive 
adjective, or the direct object of a verb {\mdash} manifests these 
underlying conceptual patterns in the order of everyday language.  
Syntactic patterns become entrenched \i{because} they are comfortable 
translations of conceptual schema, but \i{as} entrenched we hear 
these patterns as grammatically correct, not just as 
conceptually well-formed.  Likewise, we hear errata like
\q{his lied} as \i{ungrammatical}, not as conceptually incongruous.}

\p{I contend, therefore, that many conceptually-motivated word-pairing 
patterns become syntactically entrenched and thus engender a class 
of transform-pairs where the crucial, surface-level transformation
is syntactic, often in the form of translations between parts of speech, 
or between morphological classes (singular/plural, object/location, etc.).  
Consider locative constructions like

\begin{sentenceList}

\sentenceItem{} \swl{itm:grandma}{Let's go to Grandma.}{syn}
\sentenceItem{} \swl{itm:lawyers}{Let's go to the lawyers.}{syn}
\sentenceItem{} \swl{itm:press}{Let's go to the press.}{syn}
\end{sentenceList}

Here nouns like \i{Grandma}, \i{the lawyers}, and \i{the press} are 
used at sites in the surrounding sentence-forms that call for a designation 
of place {\mdash} this compels us to read the nouns as describing a place, 
even while they are not intrinsically spatial or geographical 
(e.g., Grandma is associated with the place where she lives).}

\p{In (\ref{itm:press}) and perhaps (\ref{itm:lawyers}), this locative 
figuring may be metaphorical: going \i{to the press} does not necessarily 
mean going to the newspaper's offices.  Indeed, each of these 
usages are to some degree conventionalized: going \i{to Grandma}
is subtler than going \i{to Grandmas house}, because the former 
construction implies that you are going to a \i{place}
(\q{Grandma} is proxy for her house, say), but also that Grandma is actually 
there, and that seeing her is the purpose of your visit.  In other words, 
the specific \i{go to Grandma} formation carries a supply of 
situational expectations.  There are analogous implications in 
(\ref{itm:lawyers}) and (\ref{itm:press}) {\mdash} going \i{to the press}
means trying to get some news story or information published.  
But the underlying manipulation of concepts, which structures the 
canonical situations implicated in (\ref{itm:grandma})-(\ref{itm:press}), 
is organized around the locative grammatical form as binding noun-concepts to a 
locative interpretation.  However metaphorical or imbued with additional 
situational implications, a person-to-location or institution-to-location 
mapping is the kernel conceptual operation around which the further 
expectations are organized.  Accordingly, the locative case qua 
grammatic phenomenon signals the operation of these situational 
conventions, and the syntactic norms in turn are manifest via 
word-pairings, such as \i{to Grandma}.}

\p{In short, a transform-pair like \i{to Grandma} can be analyzed in 
several registers; we can see it as the straightforward 
syntactic rendering of a locative construction (via inter-word morphology, 
insofar as English has no locative case-markers) or explore further 
situational implications.  In these examples, though, there is an 
obvious grammatic account of pairs' transformations, notwithstanding 
that there are also more semantic and conceptual accounts.  
Part-of-speech transforms (like \i{having lied}) and 
case transforms (like \i{to Grandma}) are mandated by 
syntactic norms and therefore can be absorbed into conventional 
grammatic models, such as Dependency Grammar: the head/dependent 
pairings in \i{having lied} and \i{to Grandma} are each 
covered by specific relations within the theory's inventory 
of possible inter-word connections.  So a subset of 
transform-pairs overlaps with (or can be associated with) 
corresponding Dependency Grammar pairings.}

\p{Another potential embedding of transform-pairs into 
formal models can be motivated by Type Theory.  
Such analysis may proceed on several levels, but 
in general terms we can assume that parts of speech 
form a functional type system (as elucidated, say, in 
Combinatory Categorial Grammar; e.g.,
\cite{BiskriDescles}).
For instance, we can recognize 
nouns and propositions (sentences or sentence-parts forming 
logically complete clauses) as primitive types, and 
treat other parts of speech as akin to \q{functions} between 
other types.  A verb, say, combines with a noun to 
form a proposition, or complete idea: \i{go} acts on
\i{We} to yield the proposition \i{We go}.  Schematically, 
then, verbs are akin to functions that map nouns to propositions.  
Similarly, adjectives map nouns to nouns, and 
adverbs map verbs to other verbs (here I use \q{noun} or \q{verb} to 
mean a linguistic unit which functions 
(conceptually and/or \visavis{} syntactic propriety) as a noun, or verb; 
in this sense a noun-phrase is a kind of noun {\mdash} i.e., 
a linguistic unit whose \i{type} is nominal).  
This provides a type-theoretic architecture through 
which transform-pairs can be analyzed.  
An adverb modifies a verb; so an adverb in a transform 
pair must have a verb as a ground.  Moreover, the \q{product}
of that transform is also a verb, in the sense that the
adverb-verb pair, parsed as a phrase, can only be
situated in grammatic contexts where a verb is expected.}

\p{In effect, we can apply type-theoretic models to both 
parts of a transform-pair and to the pair as a whole, 
producing structural requirements on how words link up 
into transform-pairs.  We can then see an entire sentence 
as built up from a chain of such pairs, with the rules 
of this construction expressed type-theoretically.  
Given, say, \i{his having lied flagrantly} we can identify a 
chain of pairs \i{flagrantly}-\i{lied},
\i{having}-\i{flagrantly}, and \i{his}-\i{having}, where 
the \q{outcome} of one transform becomes subject to a 
subsequent transform.  So \i{flagrantly} modifies \i{lied}
by expressing measure and emphasis, adding conceptual detail; 
grammatically the outcome is still a verb.  Then \i{having}, 
as I argued earlier, applies a transform that maps this
verb-outcome to a noun, which is then transformed by
the possessive \i{his}.  Each step in the chain is 
governed by type-related requirements: the output of one 
transform must be type-compatible with the modifier for 
the next transform.  This induces a notion of \i{type-checking}
transform-chains, which is analogous to how type-checking 
works in formal settings like computer programming languages, 
Typed Lambda Calculus, and Dimensional Analysis.}

\p{This gloss actually understates the explanatory power of 
type-theoretic models for linguistics, since I have 
mentioned only very coarse-grained type classifications 
(noun, proposition, verb, adjective, adverb); more 
complex type-theoretic constructions come into play 
when this framework is refined to consider plural/singular, 
classes of nouns, and so forth, establishing a basis for 
more sophisticated structures adapted to language from 
formal type theory, like type-coercions and
dependent types (I will revisit these theories 
in a later section).  Here, though, I will just point out 
that Dependency Grammar and Type-Theoretic Semantics can
often overlap in their analysis of word-pairs (inter-word 
relations is not centralized in type-oriented methodology as much 
as in Dependency Grammar, but type concepts can certainly
be marshaled toward word-pair analysis).}

\p{Even though Dependency and Type-Theoretic analyses will often reinforce 
one another, they can offer distinct perspectives on 
how pairs aggregate to form complete phrases and sentences.  
In the transform-pair \i{having lied}, \i{lied} is clearly the 
more significant word semantically.  This is reflected in
\i{having} being annotated (at least according to the Universal Dependency 
framework) as auxiliary, and the dependent element of the pair, while
\i{lied} is the \q{head}.  Then \i{lied} is also connected to
\i{his}, establishing a verb-subject relation.  So \i{lied} becomes 
the nexus around which other, supporting sentence elements are 
connected.  This is a typical pattern in Dependency Grammar parses, where
the most semantically significant sentence elements also tend to be 
the most densely connected (if we treat the parse-diagram as a 
graph, these nodes tend to have the highest \q{degree}, a measure of 
nodes' importance at least as this is reflected in how many 
other nodes connect to it).  Indeed, by counting word connections we 
can get a rough estimation of semantic importance, distinguishing
\q{central} and \q{peripheral} elements.  These are not 
standard terms, but they suggest a norm in Dependency Grammar that the
structure of parse-graphs generally reflects semantic priority: 
the central \q{spine} of a graph, so to speak, captures the 
primary signifying intentions of the original sentence, while the 
more peripheral areas capture finer details or syntactic auxiliaries 
whose role is for grammatical propriety more than meaningful content.}

\p{Conversely, a type-theoretic analysis might incline us to question this
sense of semantic core versus periphery: in the case of
\i{his having lied}, the transform \i{having} supplies the outcome 
which is content for the possessive \i{has}.  If we see the sentence 
as a cognitive unfolding, a series of mental adjustments toward an 
ever-more-precise reading of speaker intent, then each step in the 
transformation contributes consequential details to the final
understanding.  Moreover, \i{lied} is only present in the transformation 
signified by \i{his} insofar as it has in turn been transformed by
\i{having}: each modifier in a transform-pair has a degree of 
temporal priority because \i{its} effects are directly present
in the context of the following transformation.  This motivates a 
flavor of Dependency Grammar where the head/dependent ordering is 
inverted: a seemingly auxiliary component (like the function-word 
to a content-word) can be notated as the head because its 
output serves as \q{input} to a subsequent transform.  In the analogy 
to Lambda Calculus, \i{his having lied} would be graphed with
\i{having} being the head for \i{lied}, and \i{his} the head for
\i{having}, reflecting the relation of functions to their arguments.  
In lisp-like code, this could be written functionally as 
(his (having lied)), showing \i{having} as one function, and
\i{his} as a second one, the former's output being the latter's input.  
(Later I will include diagrams contrasting these different 
styles of parse-representation.)}

\p{Implicitly, then, Type-Theoretic Semantics and Dependency Grammar 
can connote different perspectives on semantic importance and
the unfolding of linguistic understanding.  I will explore 
this distinction further below, with explicit juxtaposition of 
parse graphs using the two methods.  I contend, however, that the 
distinction reflects a manifest duality in linguistic meaning: 
we can treat a linguistic artifact as an unfolding process or 
as a static signification with more central and more peripheral 
parts.  Both of these aspects coexist: on the one hand, we 
understanding sentences via an unfolding cognitive process;
on the other hand, this cognition includes forming a mental 
review of the essential points of the sentence, a collation of 
key ideas such as (for \ref{itm:having})
\i{his}, \i{lied}, \i{damages}, and \i{credibility}.
Given this two-toned cognitive status {\mdash} part dynamic process, 
part static outline {\mdash} it is perhaps understandable that 
different methodologies for deconstructing a sentence into 
word-pair aggregates would converge on different structural 
norms for how the pairs are interrelated, internally and to one another.}

\p{This analysis, which I will extend later, has considered transform-pairs 
from a syntactic angle {\mdash} in the sense that I have highlighted pairs 
which obviously come to the fore via grammatic principles.  As I indicated, 
I believe the notion of transform-pairs cuts across both syntax 
and semantics, so I will pivot to some analyses which attend more 
to the semantic dimension.}

\spsubsectiontwoline{Semantic Analyses of Transform-Pairs}
\p{In the simplest cases, a transform-pair represents a modifier 
adding conceptual detail to a ground, like \i{black dogs}
from \i{dogs}.  But the nature of this added detail 
{\mdash} and its evident relation to surface language {\mdash} can be highly 
varied, even among similar sentences at the surface level.  
Compare between examples like:

\begin{sentenceList}

\sentenceItem{} \swl{itm:black}{I saw my neighbor's two black dogs.}{sem}
\sentenceItem{} \swl{itm:rescued}{I saw my neighbor's two rescued dogs.}{sem}
\sentenceItem{} \swl{itm:latest}{I saw my neighbor's two latest dogs.}{sem}
\end{sentenceList}

Whereas (\ref{itm:black}) presents a fairly straightforward conceptual 
transformation, the detailing in (\ref{itm:rescued}) is a lot subtler;
mentioning \i{rescued} dogs makes no reference to perceptual qualities, 
but rather implies intricate situational background.  The term
\i{rescued dogs} strongly suggests that the dogs were adopted by their 
current owner, probably after an animal-welfare organization 
found them abandoned, or removed them from a prior abusive owner.  
This kind of backstory is packaged up, as a kind of 
situational prototype, in the conventionalized phrase
\i{rescued dogs}, implying a level of specificity more 
precise than the ajective \i{rescued} alone implies.  
Correspondingly, the verb \i{to rescue} when applied to 
dogs suggest more information than in more 
generic contexts.}

\p{The phrase \i{latest dogs} carries implications in its own 
right; we assume the neighbor had owned other dogs before.  
Of course \q{latest} implies some temporal order, but the 
understood time-scale depends on context.  
If we hear talk about a \i{vet}'s two latest dogs, we would presumably 
interpret this in terms of patients the vet has seen over the course of 
a day:

\begin{sentenceList}

\sentenceItem{} \swl{itm:vet}{We have to wait until after the vet's two latest dogs.}{sem}
\sentenceItem{} \swl{itm:organization}{I'm concerned for the rescue organization's
two latest dogs.}{sem}
\end{sentenceList}

Understanding the relevant time-frame depends on understanding the relation
between the dogs and the possessive antecedent.  In (\ref{itm:latest})
the neighbor (in a typical case) actually owns the dogs, so the 
situational context grounding the modifier \i{latest} would be understood 
against the normal time-scale for dog ownership (at least several years).  
In (\ref{itm:vet}), the vet only \q{possesses} the dogs in the sense 
of endeavoring to examine them, a process of minutes or hours.  
In (\ref{itm:organization}), the implication of the \i{organization's}
possessive \visavis{} rescued dogs is that the group endeavors 
to rehabilitate and find permanent homes for the rescuees.  So in each
case \i{latest} implies a succession of dogs, leading over time to 
two most recent ones, but the implied time-frame for our conceptualizing 
this sequence can be minutes-to-hours, or days-to-months, or years.}

\p{We should also observe that the implied time-frames and backstories in 
(\ref{itm:rescued})-(\ref{itm:organization}) are not directly signified via 
morphosyntax or lexical resources alone.  The word \i{rescued} only 
carries the \i{rescued dog} backstory when used in a context 
involving the dogs' eventual owners; in some context the more 
generic meaning of \i{rescue} could supersede:

\begin{sentenceList}

\sentenceItem{} \swl{itm:boatmen}{Boatmen rescued dogs from the flooded streets.}{sem}
\sentenceItem{} \swl{itm:firemen}{Firemen rescued dogs from the burning building.}{sem}
\end{sentenceList}

Neither (\ref{itm:boatmen}) nor (\ref{itm:firemen}) imply that the dogs were abandoned, or 
will have new owners, or be sent to a shelter, or that their rescuers are 
members of an animal-welfare organization {\mdash} in short, no element of 
the conventionalized backstory usually invoked by \i{rescued dogs} is present.  
Analogously, there is no lexical subdivision for \i{latest} which regulates 
the variance in time-frames among (\ref{itm:latest})-(\ref{itm:organization}).  
It is only by inferring a likely situational background that 
conversants will make time-scale assumptions based on one situation 
involving dog ownership, another involving veterinary exams, and a 
third involving animal-welfare rehabilitation.}

\p{That is to say, the time-scale inference I have analyzed is essentially
\i{extralinguistic}: there is no specific \i{linguistic} knowledge 
(lexical or grammatical, or even pragmatic inferences in the sense of 
deictic or anaphora resolution) which warrants the situational classification 
of (\ref{itm:latest})-(\ref{itm:organization}) into different time scales.  
Instead, the inference is driven by (to some degree socially or culturally 
specific) background-knowledge about phenomena like veterinary clinics
or animal rescue groups.  Whether or not the nuances in \i{rescued dogs}
are similarly extra-linguistic is an interesting question {\mdash} we can 
argue that the phrase is now entrenched as a \i{de facto} lexical 
entrant in its own right, so the role of \i{rescued} is not only to 
lend adjectival detail but to construct a recurring phrase with a 
distinct meaning, like \i{red card} (in football) or \i{stolen base}
(in baseball).  Lexical entrenchment is, I would argue, an 
intra-linguistic phenomenon, in the sense that understanding entrenched 
phrases is akin to familiarity with specific word-senses, which is a 
properly linguistic kind of knowledge.  But even in that case, entrenchment 
is only possible because the phrase has a signifying precision more 
rigorous than its purely linguistic composition would imply.  
There are, in short, extralinguistic considerations governing \i{when}
phrases are candidates for entrenchment, and a language-user's 
ability to learn the conventionalized meaning (which I believe 
is an intra-linguistic cognitive development) depends on their 
having the relevant (extra-linguistic) background knowledge.}

\p{If we consider then the contrast between transform-pairs like
\i{black dogs}, \i{rescued dogs}, and \i{latest dogs}, 
the similar grammatic constructions {\mdash} indeed similar semantic 
constructions, in that each pair has an adjective modifying a 
straightforward plural noun (\i{dogs} designates a similar 
concept in each case; this is not a case of surface grammar 
hiding semantic diversity, like \i{strong wine} vs.
\i{strong opinion} vs. \i{strong leash} or
\i{long afternoon} vs. \i{long history} vs. \i{long leash}) 
{\mdash} package transforms whose cognitive resolution spans a 
range of linguistic and extralinguistic considerations.  
Straightforward adjectival modification in \i{black dogs} gives 
way to lexical entrenchment in \i{rescued dogs} which, as 
I argued, carries significant extra-linguistic background knowledge 
even though possession of this knowledge is packaged into basic 
linguistic familiarity with \i{rescued dogs} as a signifying unit; 
and in the case of \i{latest dogs} the morphosyntactic evocation of 
temporal precedence and two different multiplicities (the latest
dogs and earlier ones) is fleshed out by 
extra-linguistic estimations of time scale.  
The same surface-level linguistic structures, in short, can 
(or so such examples argue for) lead conversants on a 
cognitive trajectory in which linguistic and extra-linguistic factors 
interoperate in many different ways.}

\p{This diversity should call into question the ability of conventional 
syntactic and semantic analysis to elucidate sentence-meanings with 
any precision or granularity.  Lexical and morphosyntactic 
observations may certainly reflect details which \i{contribute} to 
sentence-meanings, but the overall understanding of each sentence in
context depends on holistic, interpretive acts by competent 
language users in light of extra-linguistic, socially mediated 
background knowledge and situational understanding.  Contextuality 
applies here not only in the pragmatic sense that pronoun resolution, 
say, depends on discursive context (who is \i{her} in \i{her dogs}); 
more broadly, transcending even pragmatics, context describes 
presumptive familiarity with conceptual structures like veterinary 
clinics, animal shelters, and any other real-world domain which 
provides an overall system wherein particular lexical significations 
can be standardized.  Without the requisite conceptual background it 
is hard to analyze how speakers can make sense even of 
well-established variations in word-sense, like \i{treat} as in 
a veterinarian treating a dog, a doctor treating a wound, a carpenter 
treating a piece of wood, or how an actor treats a part.  
These senses have lexical specificity only in the domain-specific 
contexts of medicine, carpentry, theater, and so forth.}

\p{The problem of holistic cognitive interpretation (as requisite for 
sentence-meanings) can be seen even more baldly in examples where 
semantic readings bifurcate in ways wholly dependent on 
extra-linguitic conceptualization. Consider for instance:

\begin{sentenceList}

\sentenceItem{} \swl{itm:boroughs}{All New Yorkers live in one of five boroughs.}{sem}
\sentenceItem{} \swl{itm:commute}{All New Yorkers complain about how long it 
takes to commute to New York City.}{sem}
\sentenceItem{} \swl{itm:Cambridge}{The south side of Cambridge voted Conservative.}{ref}
\sentenceItem{} \swl{itm:lower}{Lower Manhattan voted Republican.}{ref}
\sentenceItem{} \swl{itm:si}{Staten Island voted Republican.}{ref}
\end{sentenceList}

Sentence (\ref{itm:Cambridge}) is taken from the \i{Handbook of Pragmantics}
(example 40, page 379, chapter 15) which borrows in turn from Ann Copestake and 
Ted Briscoe.  In the \i{Handbook} analysis (chapter by Geoffrey Nunberg), 
(\ref{itm:Cambridge}) is seen as ambiguous between 
reading \q{The south side of Cambridge} as an oblique description of 
the \i{voters} in that territory or as topicalizing the territory 
itself as a civic entity:

\begin{quote}
On the face of things, we might analyze (40) in either of two ways: either the
description within the subject NP has a transferred meaning that describes a
group of people, or the VP has a transferred meaning in which it conveys the
property that jurisdictions acquire in virtue of the voting behavior of their
residents.
\end{quote}

The duality is significant because the designation in \q{south side of Cambridge}
would be more informal in the prior reading, both geographically 
and in the how the implied collective of people is figured.  The prior 
reading accommodates a hearing wherein the speaker construes
\q{south side} not as a precise electoral district (or districts) but as a 
vaguely defined part of the city.  That imprecision also allows the claim
\q{voted Conservative} to be only loosely committal, implying that 
some majority of voters appeared to vote Conservative but not that 
this tendency is directly manifest in election results.  In short, 
we can interpret (\ref{itm:Cambridge}) as making epistemically more 
rigorous or more noncommittal claims depending on how we read 
the geographical reference \q{south side of Cambridge} (as crisp 
or fuzzy), the group of people selected via that reference 
(mapping the region to its inhabitants, a kind of \q{type coercion}
since places do not vote), and the assertive force of the 
speech-act: how precisely the speaker intends her claim to be 
understood.  Each of these \q{axes} contribute to the sentence's 
meaning insofar as they constrain what would be dialogically 
appropriate responses.}

\p{Meanwhile, in (\ref{itm:boroughs}), \i{New Yorkers} refers specifically to everyone who lives 
in the City of New York, since the five boroughs collectively span the 
whole of city.  In (\ref{itm:commute}), by contrast, we should understand
\i{New Yorkers} as referring to residents of the metropolitan area \i{outside}
the city itself (who commute \i{to} the city); and moreover \i{All} should 
be read less then literally: we do not hear the speaker in (\ref{itm:commute}) 
committing to the proposition that \i{every single} New Yorker complains.  
So both \i{All} and \i{New Yorkers} have noticeably different meanings in the 
two sentences.}

\p{And yet, I cannot find any purely linguistic mechanism 
(lexical, semantic, syntactic, morphological) which would account for 
these difference as linguistic signifieds \i{per se}: the actual differences 
depend on conversants knowing some details about New York 
(or, respectively, Cambridge) geography, and 
also general cultural background.  It does not make too much sense to 
commute to a place where you already live, so our conventional picture of 
the word \i{commute} constrains our interpretation of (\ref{itm:commute}) {\mdash} 
but this depends on \i{commute} having a specific meaning, of traveling 
in to a city, usually from a suburban home, on a regular basis; a meaning 
in turn indebted to the norms of the modern urban lifestyle (it would be 
hard derive an analogous word-sense in the language spoken by a nomadic 
tribe, or a pre-industrial agrarian community).  Likewise, 
reading \i{All} in (\ref{itm:boroughs}) as \i{literally} \q{all}
depends on knowing that the five boroughs are in fact the whole of 
the city's territory.  I am from New York, not Cambridge; 
perhaps residents of the latter city would clearly read
\q{south side} as referencing a fixed civic/electoral area 
(like \i{Staten Island}) or as only vaguely defined (like
\i{Lower Manhattan}).  For New Yorkers, 
(\ref{itm:lower}) would be read as fuzzy and 
(\ref{itm:si}) as fixed: the latter sentence has a clearly 
prescribed fact-check (since Staten Island is a distinct 
electoral district) which the former lacks.}

\p{Given that in everyday speech quantifiers like \i{all} or \i{every}
are often only approximate {\mdash} and that designations like
\i{New Yorker} are often used imprecisely, with not-identical 
alternative meanings intended on a case-by-case basis {\mdash} 
these kind of examples point to signifying ambiguities that 
can easily arise as a consequence.  Often extra-linguistic 
considerations resolve the ambiguity by rejecting one or another 
(otherwise linguistically plausible) reading as non-sensical.  
Consider:

\begin{sentenceList}

\sentenceItem{} \swl{itm:beat}{The Leafs failed to beat the Habs for the first time this year.}{amb}
\sentenceItem{} \swl{itm:consecutive}{The Leafs failed to win two consecutive games for the 
first time this year.}{amb}
\sentenceItem{} \swl{itm:goal}{The Leafs failed to score a goal for the 
first time this year.}{amb}
\end{sentenceList}

Sentence (\ref{itm:beat}) has two competing readings: either the Toronto Maple Leafs 
won \i{all} or \i{none} of their previous games, in the relevant year, against the 
Montreal Canadiens.  The difference is whether \i{for the first time this year}
attaches to \i{beat} or to \i{fail}.  In (\ref{itm:consecutive}), 
on the other hand, the only sensible interpretation is that the Leafs had 
not yet won two games: while it is logically accurate to describe a 
team on a long winning streak as repeatedly winning two consecutive games, it would 
be very unexpected for (\ref{itm:consecutive}) to be used in a case where the Leafs 
lost for the first time, after a three-plus-game winning streak.  
And in (\ref{itm:goal}) any hockey fan would hear that the Leafs had scored 
at least one goal in all prior games; even though there is no linguistic rule 
foreclosing the reading such that the Leafs have not scored a goal in \i{any} game.}

\p{These variations {\mdash} the degree to which superficial ambiguity is actually 
perceived by competent language-users as presenting competing plausible 
meanings {\mdash} depend on background factors; the contingencies of 
hockey fix how potential ambiguities resolve out because one or another 
alternative is extralinguistically incoherent.  But these cases point to 
how linguistic criteria alone, no matter how broadly understood, 
cannot necessarily predict in what sense linguistic structurations 
have empirically plausible meanings {\mdash} or whether they have 
sensible meanings at all.}

\p{Notice however that all these examples have alternate versions which are less 
subtle or ambiguous, which shows that the complications are not 
localized in the communicated ideas themselves, but in their 
typical linguistic encoding:

\begin{sentenceList}

\sentenceItem{} \swl{itm:allresidents}{All residents of the city of New York live in one of five boroughs.}{log}
\sentenceItem{} \swl{itm:manyresidents}{Many residents of the New York metropolitan area 
complain about how long it takes to commute to New York City.}{log}
\sentenceItem{} \swl{itm:SouthCambridge}{All districts on the south side of Cambridge voted Conservative.}{log}
\sentenceItem{} \swl{itm:leafshabs}{For the first time this year, the Leafs failed to beat 
the Habs.}{log}
\end{sentenceList}

These versions are more logically transparent, in that their propositional 
content is more directly modeled by the structure of the sentences. 
Indeed, hearers unfamiliar with New York (respectively Cambridge) 
or with hockey might find these 
versions easier to understand; more context-neutral and journalistic.  
But perhaps for this reason the \q{journalistic} versions actually 
sound stilted or non-idiomatic for everyday discourse.}

\p{Geoffrey Nunberg's chapter (15) in the
\i{Handbook} also has interesting examples that point to extralinguistic 
factors determining proxies and substitutions in referential contexts:

\begin{sentenceList}

\sentenceItem{} \swl{itm:parked}{I am parked out back.}[(example 3, page 371)]{ref}
\sentenceItem{} \swl{itm:waiting}{I am parked out back and have been waiting for 15 minutes.}[(example 3)]{ref}
\sentenceItem{} \swl{itm:start}{I am parked out back and may not start.}[(example 6, page 371)]{ref}
\sentenceItem{} \swl{itm:Whitney}{I'm in the Whitney Museum.}[(page 375)]{ref}
\sentenceItem{} \swl{itm:beers}{I drank two beers.}[(page 375)]{ref}
\sentenceItem{} \swl{itm:Michelobs}{I drank two Michelobs.}[(page 375)]{ref}
\sentenceItem{} \swl{itm:Sauternes}{I drank two Sauternes last night.}[(page 375)]{ref}
\end{sentenceList}

Someone says \q{I am parked} as proxy for their car, or \q{in the Whitney} as 
proxy for their art work.  Nunberg questions the acceptability of 
(\ref{itm:start}) on the premise that once we establish talking of 
people as an oblique reference to their cars, it sounds awkward to switch 
to content that can \i{only} apply to the car, like \i{may not start}.  
He also argues that we interpret (\ref{itm:beers}) and (\ref{itm:Michelobs}) 
as references, quite possibly, two glasses or bottles of \i{the same}
beverage, while (\ref{itm:Sauternes}) implies two samples of
\i{different} wines.  The analysis turns on conventions for 
creating referring expressions; in particular, he argues, 
extralinguistic factors (albeit not using this term) 
regulate when referential substitutions as exemplified in 
(\ref{itm:parked})-(\ref{itm:Sauternes}) are comfortable.
We accept an artist proxying her work in a museum, but 
(\ref{itm:Whitney}) arguably does not generalize to a case like 
(\ref{itm:library}), and (\ref{itm:waiting}) does not generalize to 
(\ref{itm:autoshop})

\begin{sentenceList}

\sentenceItem{} \swl{itm:library}{I'm in the library.}[(said by an author whose books are there)]{ref}
\sentenceItem{} \swl{itm:autoshop}{I've been in the auto shop for the 
last 15 days.}{ref}
\end{sentenceList}

Nunberg attributes these discrepancies to cultural conventions, like the
difference in prestige between having a painting in a museum compared to 
having one copy of one's book in a library; or chugging a cheap been 
versus savoring a celebrated wine.  Here too there are 
logically transparent alternatives:

\begin{sentenceList}

\sentenceItem{} \swl{}{My car is parked out back and have been waiting for 15 minutes.}{log}
\sentenceItem{} \swl{}{I drank two pints of beer last night.}{log}
\sentenceItem{} \swl{}{I tried two different Sauternes last night.}{log}
\end{sentenceList}

Again, though, the seemingly simpler versions {\mdash} whose forms 
generalize more readily to variant cases {\mdash} seem \i{less} natural 
or fluent, as if their very generalizability makes them sound 
awkward compared to logically more opaque, but idiomatically 
more sociable, renderings.  Impersonal, journalistic 
language can sound rather cold or unfriendly in contexts 
where speakers expect a dialect register characteristic 
of conversations among peers.}

\p{In short, even if sentences have a basically transparent 
logical content, \i{how} sentences holistically signify this 
content does not always emerge straightforwardly from semantic 
or syntactic structures on their own.  I think this weakens the 
case for semantic paradigms that concentrate on logically-structured 
content which appears to be signified through sentences 
{\mdash} even if we grant that this propositional ground of meaning 
is real, it does not follow that propositional contents are 
designated by purely linguistic means, rather than by 
a cohort of cognitive processes many of which are extra-linguistic.  
This is the basis of my proposing \q{logicomorphic} qualities 
as one axis for evaluating sentences, which I will now discuss further.}

\spsubsectiontwoline{Gaps in Logical Phrase-Models}
\p{Assume we have a baseline lambda-calculus-like 
functional summary of sentences and derived types.  That is, 
any sentence can be rewritten as if a sequence of \q{function calls}, 
assuming an underlying representational vocabulary of a typed 
Lambda Calculus, with sentences having overall \q{proposition} types 
(I will present this model in detail next section).}

\p{My overall goal is to embrace a hybrid methodology {\mdash} 
accepting formal analyses when they shed light on linguistic 
processes, but not going so far as to treat logical, mathematical, 
or computational models as full explanations for linguistic 
rationality qua scientific phenomenon.  
Cognitive Grammar, in particular, challenges our assumption that 
grammar and semantics are methodologically separate.  Received 
wisdom suggests that grammar concerns the \q{form} of sentences 
whereas semantics considers the meaning of words {\mdash} implicitly 
assuming that \i{word combinations} produce new meanings, and 
that the \i{order} by which words are combines determines how 
new meanings are produced.  This notion, in turn, is allied 
with the essentially logical or propositional picture of 
signifying via doxa: the idea that inter-word relations cue up 
different logically salient transformations of an underlying 
predicate model.  Thus {\mdash} to initiate a case-study I will 
return to several times {\mdash} \i{many students} as a phrase is 
more significatorily precise than \i{students} as a word, because 
the phrase (with intimations of quantitative comparison) has 
more logical detail.  Similarly \i{many students complained}
is more logically complete because, provisioning both a verb-idea 
and a noun-idea, it represents a whole proposition.}

\p{In general, then, phrases are more complete than words because 
they pack together more elements which have some logical role, 
establishing individuals, sets, spatiotemporal setting, and 
predicates which collectively establish sufficiently 
completed propositional attitudes.  On this account the 
key role of phrase-structure is to establish phrases as 
signifying units on a logical level analogous to 
how lexemes are signifying units on a referential or conceptual 
level.  Moreover, phrases' internal structure are understood 
to be governed by rule defining \i{how} word-combinations 
draw in extra logical detail.  A link between words is not a 
random synthesis of concepts, but rather implies a certain 
logical connective which acts as a de facto \q{third party} in 
a double-word link, proscribing with orientation to predicate 
structures \i{how} the words' semantic concepts are to be 
joined.  In \i{many students} the implied connector is 
the propositional act of conceiving a certain quantitative 
scale to a conceptualized set; in \i{students complained} the 
implied connector is a subject-plus-verb-equals-proposition 
assertiveness.  Phrases acquire logical specificity by 
building up word-to-word connections into 
more complex aggregates.}

\p{One implication of this model is that phrases are semantically 
substitutable with individual lexemes that carry similar meanings, 
having been entrenched by convention to capture a multipart concept 
which would otherwise be conveyed with the aggregation of a 
phrase: consider \q{MP} for \i{member of parliament}, or
\q{primaried} for \i{subject without your own party to a primary 
challenge}.  Conversely, phrases can be repeatedly used in a 
specific context until they function as quasi lexical units in 
their own right.  These patterns of entrenchment imply that we 
hear language in term of phrases bearing semantic content; and 
insofar as we are comfortable with how we parse a sentence, each 
word sited in its specific phrasal hierarchy, we do not tend to 
consider individual words semantically outside of their 
constituent phrases.}

\p{This theory of the syntax-to-semantic relationship is a paradigmatic 
partner, at the grammatic level, to the belief that meanings are 
fixed by \q{truth makers}: that is a sentence asserts a true fact, 
this fact is its first and foremost meaning, and if not, the meaning 
is somehow the description of a possible world where it \i{is} true.  
One of my projects here is to criticize (to some extent) these
\q{truth-theoretic} paradigms.  At present I want to point out 
how this philosophical paradigm can influence our 
appraisal of phrase-structure.  Simplistic \q{logic-based} conceptions of 
phrases can be based on two concerns: first, the idea that lexemes retain some 
syntactic and semantic autonomy even within clearly defined phrases 
where they are included; and, second, that the shape of phrases 
insofar as they are perceived as holistic signifying units 
is often driven by figurative or \q{gestalt} principles rather than 
neat logical structuration.  I'll call the former the issue of
\q{phrasal isolation} (or lack thereof): syntactic and semantic effect 
often cross phrasal boundaries, even outside the overarching hierarchy 
whose apex is the whole sentence.  Both of these lines of reasoning 
{\mdash} arguably especially the second {\mdash} are developed in 
Cognitive Grammar literature.}

\p{In Ronald Langacker's \i{Foundations of Cognitive Grammar}, the sentence

\begin{sentenceList}

\sentenceItem{} \swl{itm:threetimes}{Three times, students asked an interesting question.}{sco}
\end{sentenceList}

is used to demonstrate how
grammatical principles follow from cognitive \q{construals} of the relevant situations,
those which language seeks to describe or takes as presupposed context.\footnote{For example, \cite[pp. 119 and 128]{LangackerFoundations},
discussed by \cite[p. 189]{LineBrandt}, and \cite[p. 9]{EstherPascual}.}
In particular, Langacker argues that \q{students} and \q{question} can both be either singular or
plural: syntax is open-ended here, with neither form more evidently correct.  Langacker uses this
example to make the Cognitive-Linguistic point that
we assess syntactic propriety relative to cognitive frames and conversational context.  In this
specific case, we are actually working with two different cognitive frames which are interlinked
{\mdash} on the one hand, we recognize distinct events consisting of a student asking a question, but
the speaker calls attention, too, to their recurrence, so the events can also be understood
as part of a single, larger pattern.  There are therefore two different cognitive foci, at two
different scales of time and attention, a \q{split focus} which makes both singular and plural
invocations of \q{student} and \q{question} acceptable.}

\p{Supplementing this analysis, however, we can additionally focus attention directly on
grammatical relations.  The words \i{student} and \i{question} are clearly linked as the subject and
object of the verb \i{asked}; yet, contrary to any simple presentation of rules,
no agreement of singular or plural is required between them (they can be singular and/or plural in
any combination).  Moreover, this anomaly is only in force due to the context established
by an initial phrase like \i{three times}; absent some such framing, the singular/plural
relation would be more rigid.  For example, \q{A student asked interesting questions} would
(in isolation) strongly imply \i{one} student asking \i{several} questions.  So the initial
\q{Three times} phrase alters how the subsequent phrase-structure is understood while remaining
structurally isolated from the rest of the sentence.  Semantically, it suggests a
\q{space builder} in the manner of Gilles Fauconnier or Per Aage Brandt
\cite{Fauconnier}; \cite{PerAageBrandt}, but we
need to supplement Mental Space analysis with a theory of how these spaces
influence syntactic acceptability, which would seem to be logically prior to the stage where Mental Spaces
would come in play.}

\p{The mapping of (\ref{itm:threetimes}) to a logical
substratum would be more transparent with a case like:

\begin{sentenceList}

\sentenceItem{} \swl{itm:threes}{Three students asked interesting questions.}{log}
\end{sentenceList}

(\ref{itm:threes}) is a more direct translation of the facts
which the original sentence conveys.  But this \q{more logical} example has different
connotations than the sentence Langacker cites; (\ref{itm:threetimes}) places the emphasis
elsewhere, calling attention more to the idea of something temporally drawn-out,
of a recurrence of events and a sense of time-scale.  The \q{more logical} sentence
lacks this direct invocation of time scale and temporal progression.}

\p{We can say that the \q{Three students} version is a more direct statement of fact, whereas
Langacker's version is more speaker-relative, in the sense that it elaborates more
on the speaker's own acknowledgment of belief.  The speaker retraces the steps of her
coming to appreciate the fact {\mdash} of coming to realize that the \q{interesting questions}
were a recurrent phenomenon and therefore worthy of mention.  By situating expressions
relative to cognitive processes rather than to the facts themselves, the sentence
takes on a structure which models the cognition rather than the states of affairs.
But this shift of semantic grounding from the factual to the cognitive also apparently
breaks down the logical orderliness of the phrase structure. \q{Three times}, compared
to \q{three students}, leads to a morphosyntactic choice-space which is
\q{underdetermined} and leaves room for speakers' shades of emphasis.}

\p{This is not an isolated example.  Many sentences can be provided with similar
phrase-structure complications, particularly with respect to singular/plural agreement.

\begin{sentenceList}

\sentenceItem{} \swl{}{Time after time, tourists (a tourist) walk(s) by this building
with no idea of its history.}[]{sco}[Time after time, tourists walk by this building
with no idea of its history.]

\sentenceItem{} \swl{}{The streets around here are confusing; often people (someone)
will ask me for directions.}[]{sco}[The streets around here are confusing; often someone
will ask me for directions.]

\sentenceItem{} \swl{itm:students}{Student after student came with their (his/her)
paper to complain about my grade(s).}[]{sco}[Student after student came with their
paper to complain about my grades.]

\sentenceItem{} \swl{itm:parents}{Student after student {\mdash} and their (his/her) parents
{\mdash} complained about the tuition increase.}[]{sco}[Student after student {\mdash} and their parents
{\mdash} complained about the tuition increase.]
]

On a straightforward phrase-structure reading, \i{student after student} reduces to an
elegant equivalent of \i{many students}, with the rhetorical flourish abstracted away
to a logical form.  But our willingness to accept both singular and plural agreements
(his/her/their parents, grades, papers) shows that clearly we don't simply substitute
\i{many students}; we recognize the plural as a logical gloss on the situation but
engage the sentence in a more cognitively complex way, recognizing connotations of temporal
unfolding and juxtapositions of cognitive frames.  The singular/plural underdeterminism
is actually a signification in its own right, a signal to the listener that the
sentence in question demands a layered cognitive attitude.  Here again, syntactic
structure (morphosyntactic, in that syntactic allowances are linked with
variations in the morphology of individual words, such as singular or plural form)
serves to corroborate conversants' cognitive frames rather than to model logical
form.]

\p{The contrast between the phrases \i{Student after student} and \i{Many students}
cannot be based on \q{abstract} semantics alone {\mdash} how the evident temporal implications of the
first form, for example, are concretely understood, depends on conversants' mutual recognition
of a relevant time frame.  The dialog may concern a single day, a school year, many
years.  We assume that the speakers share a similar choice of time \q{scale}
(or can converge on one through subsequent conversation). \i{Some} time-frame
is therefore presupposed in the discursive context, and the first phrase invokes
this presumed but unstated framing.  The semantics of the phrase are therefore somewhat open-ended:
the phrase \q{hooks into} shared understanding of a temporal cognitive framing without referring
to it directly.  By contrast, the second phrase is less open-ended: it is consistent with both
a more and less temporally protracted understanding of \i{many}, but leaves such details (whatever
they may be) unsignified.  The factual circumstance is designated with a level of abstraction that sets
temporal considerations outside the focus of concern.  The second (\q{\i{Many students}}) phrase 
is therefore both
less open-ended and also less expressive: it carries less detail but accordingly also relies
less on speaker's contextual understanding to fill in detail.\footnote{The examples I have used so far may also imply that a choice of phrase structure is
always driven by semantic connotations of one structure or another;
but seemingly the reverse can happen as
well {\mdash} speakers choose a semantic variant because its grammatic realization lends a useful
organization to the larger expression.  There are many ways to say \q{many},
for example: \i{a lot of},
\i{quite a few}, not to mention \q{time after time} style constructions.  Whatever their
subtle semantic variations, these phrases also have different syntactic properties:
\i{Quite a few} is legitimate as standalone (like an answer to a question);
\i{A lot of} is not, and \i{A lot} on its own is awkward.  On the other hand the \q{of} in
\i{A lot of} can \q{float} to be replicated further on: \q{A lot of students, of citizens,
believe education must be our top priority} sounds more decorous than the equivalent sentence with the
second \q{of} replaced by \q{and}.  If the cadence of that sentence appeals to the speaker, then
such stylistic preference will influence taking \q{A lot of} as the \q{many} variant of choice.
So speakers have leeway in choosing grammatic forms that highlight one or another aspect of
situations; but they also have leeway in choosing rhetorical and stylistic pitch.  Both cognitive
framings and stylistic performance can be factored when reconstructing what compels the
choice of one sentence over alternatives.}
}

\p{One consequence of these analyses is that grammar
needs to be approached holistically: the grammatic structure of phrases cannot,
except when deliberate oversimplification is warranted, be isolated from
surrounded sentences and still larger discourse units.  Semantic roles
of phrases have some effect on their syntax, but phrases are nonetheless chosen
from sets of options, whose variations reflect subtle semantic and syntactic
maneuvers manifest at super-phrasal scales.  The constituent words of phrases retain some
autonomy, and can enter into inter-word and phrasal structures with other words outside their
immediate phrase-context.  We can still apply formal models to phrase
structure {\mdash} for example, Applicative and Cognitive Grammar (\ACG{}) considers phrases
as \q{applications} of (something like) linguistic or cognitive functions,
with (say) an adjective modeled as a function applied to a noun,
to yield a different noun (viz., something playing a noun's conceptual role)
\cite{Descles2010} {\mdash} but we should not read these transformations
too hastily as a purely semantic correlation within a space of denotable concepts
\i{such that} the new concept wholly replaces the
contained parts, which then cease to have further linguistic role and effect.
Instead, applicative structures represent shifts or evolutions in
mental construal, which proceed in stages as conversants form
cognitive models of each others' discourse.  Even if phrase structure
sets landmarks in this unfolding, phrases do not wholly subsume their
constituents; the parts within phrases do not \q{vanish} on the higher scale,
but remain latent and may be \q{hooked} by other, overlapping phrases.}

\p{Consider the effect of \q{Many students complained}.  Propositionally, this appears
to say essentially that \i{students} complained; but, on hermeneutic charity, the
speaker had \i{some} reason to say \q{many}.  The familiar analysis is that
\q{many} suggests relative size; but this
is only half the story.  If the speaker chose merely \i{students complained}, we would hear an assertion
that more than one student did, but we would also understand that there were several
occasions when complaints happened.  Adding \q{many} does not just
imply \q{more} students, but suggests a mental shift away from the particular episodes.
In the other direction, saying \i{a student complained} is not just
asserting how at least one student did so, but
apparently reports one specific occasion (which perhaps the speaker wishes to
elaborate on).  In other words, we cannot really capture the singular/plural semantics,
or different varieties of plural, just by looking at the relative size of implied
sets; we need to track how representations of singleness or multitude imply
temporal and event-situational details.}

\p{Against this backdrop, \i{Student after student complained} captures both dimensions,
implying both a widespread unrest among the student body and also
temporal recurrence of complainings.
The \BlankAfterBlank{} idiom is a special case \visavis{} \q{after}, where 
the \q{argument} to
\i{after} is repeated in both positions, suggesting an unusual degree of repetition,
something frustratingly recurrent: \i{He went on and on}; \i{Car after car passed
us by}; \i{Time after time I got turned down}.
Although I have no problem
treating these constructions as idiomatic plurals, I also contend (on the
premise of phrase-overlap) that the dependent constituents in the \BlankAfterBlank{}
construction can be hooked to other phrases as well (which is why
\q{and [their/his/her] parents} can also be singular, in this case).  I dwell on
this example because it shows how type-theoretic accounts of phrase structure, 
which I will explore next section, 
can be useful even if we treat phrases more as frames which overlay linguistic
structure, not as rigid compositional isolates.  Each \q{students} variation uses
morphology to nudge cognitive attention in one direction or another, toward events or the
degree to which events are representative of some global property (here of
a student body), or both.  The pluralizing transformation {\mdash} from 
one student to many students {\mdash} is not \i{the}
morphosyntactic meaning, but instead the skeleton on which the full meaning
(via cognitive schema) is designed.}

\p{If this analysis has merit, it suggests that a Combinatory Grammar or type-logical
approach to phrases like \i{many students} or \i{student after student}
(singular-to-plural or plural-to-plural mappings) should be understood not just
as functions among Part of Speech (\POS{}) types but as adding cognitive shading, foregrounding
or backgrounding cognitive elements like events or typicality in some context.
In other words, \i{many students} is type-theoretically a pluralizing action, 
but, in more detail, it adds a kind of cognitive rider attached to the mapping which focuses
cognition in the subsequent discourse onto events (their recurrence and temporal distribution);
similarly \q{student after student} has a \q{rider} suggesting more of a temporal
unfolding.  The second form implies not only that many students complained, but that
the events of these complainings were spread out over some stretch of time.
Each such functional application (mappings between \POS{} understood as linguistic types)
produces not only a resulting \POS{} \q{type}, but also a reconfiguration of cognitive
attitudes toward the relevant situation and context.
Language users have many ways to craft a sentence with similar meanings, and arguably one
task for linguistic analysis is to model the space of choices which are available in a
given situation and represent what specific ideas and effects are invoked by one
choice over others.}

\whdecoline{}
\p{Logical semantic models build up propositional representations from the 
word level to phrase and sentence levels.  Even when we find that a logical
\i{telos} is in effect, however, at each of these levels we can 
find interpretive and situational details which may complicate, flesh out, 
or cognitively supersede a more transparent logical meaning.
So at the word-pair level, there is more going on in 
a transform like \i{rescued dog} than simply applying the 
predicate \q{rescue} to the ground \q{dog}.  The construction 
drafts a complex backstory that influences our cognizing both 
component words.  Scaling up, as I argued in the last several 
paragraphs, a phrase like \i{student after student} carries 
conceptual effects beyond just effectuating a logical 
operation of pluralization.  I will argue that similar 
effects can be identified at the sentence level {\mdash} so 
there are certain lacunae in narrowly logically-based 
reconstructions of language components at the 
distinct levels of word (transform) pairs, phrases, 
and whole sentences.  I'll sketch my arguments 
here and give a more thorough analysis later in 
Section \ref{sec:Gaps}.}

\spsubsectiontwoline{Logical Structure versus Sentence Structure}
\p{Let us grant in general that particular sentences can be 
mapped to distinct, relatively transparent propositional 
contents.  In some cases sentences expresses propositional 
attitudes to such content (requests, commands, questioning) 
rather than unadorned locutionary assertions.  To 
properly respond to speech-acts, however (even ones with 
illocutionary force) conversants need to derive the 
content which is logically conjured via the discourse,
either as the speaker's primary intent or as a condition 
for that intent.  In effect, a proposition like \i{the window 
is closed} furnishes logical content to assertions like
\i{The window is closed now} but also statements of 
belief (\i{I think the window is closed}) or 
requests or opinions (\i{The window should be closed};
\i{Please close the window}).}

\p{Philosophical \q{truth-theoretic} paradigms imply that such 
propositional contents are the \i{essential} meanings within 
language; that analyzing semantic forms via logical 
structure is the core of a rigorous theory of semantics.  
It is certainly true that many elements of language 
can be translated, or deemed as conventionalized encodings 
for, structures in predicate logic {\mdash} invocations of 
multiplicity and quantification; logical connectives between 
propositions; negations, modalities, and possibilia.  
This provides an analytic matrix wherein \i{some}
sentences' structures may be analyzed.  I will argue, 
however, that in typical cases logical forms are 
invoked only indirectly {\mdash} which calls into question 
the applicability of logical analysis as explanatory 
vehicles for \i{linguistic} analysis in itself 
(as opposed to more general cognitive/extralinguistic 
processing).}

\p{There are several cognitive operations requisite for grasping 
sentence-meanings as a logical gestalt: figuring individuals 
or multiplicities as conceptual foci (verb subjects or objects); 
establishing relationships between individuals and multiplicities 
or among multiplicities (member/part of, larger/smaller, 
overlap/disjoint); predicating properties to individuals 
or multiplicities; quantification; logical conjunction or disjunction, 
between predicates (also negation).  In some cases we can find 
these operations fairly directly encoded in explicit language form 
{\mdash} sentences which are precise in figuring multiplicities numerically, 
or through unambiguous use of determiners like \i{all} and \i{every}; 
which are structured to avoid scope ambiguities; which use transparent
semantic resources to describe verb subjects and objects; and 
so forth.  In the most recent Universal Dependencies Shared Task corpus 
we can find examples like:

\begin{sentenceList}

\sentenceItem{} \swl{itm:tumour}{It is the most common tumour found in babies, occurring in one of every 35,000 births.}{log}
\udref{en_pud}{n01051007}
\sentenceItem{} \swl{itm:Dengue}{Dengue fever is a leading cause of illness and death in the tropics and subtropics, with as many as 100 million people infected each year.}{log}
\udref{en_partut-ud-train}{en_partut-ud-1498}
\sentenceItem{} \swl{itm:TalibanKarzai}{Many Taliban living in Afghanistan voted for President Karzai.}{log}
\udref{en_ewt-ud-train}{w03002048}
\sentenceItem{} \swl{itm:Ashraf}{Most of the girls I was meeting had grown up in Mujahedeen schools in Ashraf, where they lived separated from their parents.}{log}
\udref{en_ewt-ud-train}{weblog-blogspot.com_rigorousintuition_20060511134300_ENG_20060511_134300-0112}
\sentenceItem{} \swl{itm:ChinaSpace}{Most experts believe China intends to develop a small space station of its own over the next several years.}{log}
\udref{en_ewt-ud-train}{newsgroup-groups.google.com_FOOLED_7554c5ce34a5a49e_ENG_20051012_144800-0020}
\sentenceItem{} \swl{itm:tastings}{Check out their wine tastings every Friday night!}{log}
\udref{en_ewt-ud-train}{reviews-322225-0005}
\sentenceItem{} \swl{itm:tag}{For each start tag , there is a corresponding end tag.}{log}
\udref{en_lines-ud-train}{166}
\sentenceItem{} \swl{itm:Warhol}{Each collection donated by the Andy Warhol Photographic Legacy Program holds Polaroids of well-known celebrities.}{log}
\udref{en_gum-ud-train}{GUM_news_warhol-35}
\end{sentenceList}

These sentences have straightforward logical structure, in terms of 
how they establish topical foci (\i{one of every 35,000 births};
\i{as many as 100 million people};
\i{Many Taliban}; \i{every Friday night}; \i{For each start tag}), 
and how predicates or references are bound together to create more 
precise significations (\i{the tropics and subtropics};
\i{in Mujahedeen schools in Ashraf}; \i{a corresponding end tag}).
Properties ascribed to subject foci are neatly drawn, both in 
conveying the property intended and its bearer, according to 
the sentence's terms: \i{the most common tumour found in babies};
\i{a leading cause of illness and death};
\i{China intends to develop a small space station};
\i{holds Polaroids of well-known celebrities}.  With 
aggregate foci and/or quantification, there is an unambiguous 
framing of predication and quantifier scope {\mdash} Each collection 
has its set of Polaroids; the set of Karzai voters, Dengue infections, 
birth tumours, etc., are crisply figured.}

\p{For many philosophers of language, identifying similar 
logical structuration is an intrinsic aspect of 
coming to terms with human language in general.  This paradigm 
also reinforces the goal of Artificial Intelligent Natural Language Processing, 
because computers can certainly engage in the kind of 
symbolic-logical reasoning outlining signified meanings in cases 
where language reciprocates propositional morphology very clearly.  
The problem is that language artifacts very often cloak their 
logical core, such that examples like (\ref{itm:tumour})-(\ref{itm:Warhol}) 
are not representative of language as a whole.  Logical patterns
may certainly be present, but they are not necessarily 
structurally reproduced in surface-level formations; rather a 
sentences' propositional content may depend on a subtle 
interpretive trajectory.  I will present examples throughout this 
paper, but a few further corpus items are reasonable 
case-studies:

\begin{sentenceList}

\sentenceItem{} \swl{itm:ants}{A furry black band of ants led up a cupboard door to some scrap that had flicked from a plate.}{sem}
\udref{en_lines-ud-train}{2157}
\sentenceItem{} \swl{itm:current-waiting-period}{The current waiting period is eight weeks.}{sem}
\udref{en_pud}{n01050009}
\sentenceItem{} \swl{itm:immersed}{I think that's why they immersed themselves in pattern and colour.}{sem}
\udref{en_pud}{n01087018}
\sentenceItem{} \swl{itm:princess}{With her appearance finalized, Jasmine became Disney's first non-white princess as opposed to being of European heritage.}{sem}
\udref{en_pud}{w01119076}
\end{sentenceList}

It requires a certain cognitive flexibility to understand a band of ants as \q{flurry}, or 
to parse the disjoint timeframes in \i{current waiting period}.  In (\ref{itm:immersed}), 
the presumed sense of \q{immerse} transcends any immediate, perceptual immersion, 
instead involving scholarship or engagement with artistic form; and (\ref{itm:princess}) 
depends on us understanding the meaning of temporality in Jasmine's appearance being
\i{finalized}, and also her \i{becoming} non-white.  As a fictional character, 
discourse about Jasmine can be evaluated in the time-frame of her artistic 
creation, distinct from the fictional time of her narrated world.}

\p{I think the intended propositional content in (\ref{itm:ants})-(\ref{itm:princess}) 
is no less evident than in (\ref{itm:tumour})-(\ref{itm:Warhol}); however, interpreting 
the topical foci and predicate attributions constituting such 
propositional content requires a holistic reading whose compositional 
structure is not recapitulated in the sentence-forms themselves.  
In the latter examples, then, merely notating propositional 
content in logical fashion does not yield a very informative
\i{linguistic} analysis, since it does not address the key question 
of \i{how} the sentences signify those propositions.}

\p{I propose to use the term \q{logicomorphic} for sentence in the former vein; 
in such cases, pointing out propositional content is linguistically 
useful because we can treat that content as a prototype for 
sentence organization.  That is, propositional content is not only
\i{holistically} signified but, in its logical structure, sheds 
light on pattern in the language.  The purpose of phraseology 
like \i{most common tumour}, \i{Many Taliban living in Afghanistan},
\i{Most of the girls I was meeting}, etc., is to circumscribe 
a focus or a property suitable for predication, and we can 
logically model the tools used to do so: logical superlative (\i{most common}), 
assertions of magnitude (\i{Many}, \i{Most of}), refining an multiplicity 
with some further criteria (\i{the girls I was meeting},
\i{Taliban living in Afghanistan}), and so forth.  These are
\q{logicomorphic} constructions in that we can read the logical 
structure of signified propositional content as a direct cause 
of the given phrasal morpholgy.}

\p{On the other hand, I call examples like (\ref{itm:ants})-(\ref{itm:princess})
\q{interpretive} because the sentences' propositional content, with 
its logical structure, does not explain the compositional 
rationale for the explicit linguistic form: we cannot read any 
pattern in the logic as a direct motivation for how the 
sentence is pieced together.  The spectrum between
\i{logicomorphic} and \i{interpretive} represents different 
strategies by which language is composed in anticipation of 
its cognitive reception, with the eventual goal of 
establishing a signified propositional content, but in different ways.  
On the logicomorphic side, logical form informs language directly; 
on the interpretive side, the actual rationale for 
compositional structures transcends exact predicative structure 
{\mdash} a more perceptual or indirect figuring of topical 
focus, for instance, or a more elliptical construal of 
predicate attributes, leaving the hearer to piece together the 
final propositional via some pragmatic or extralinguistic calculation.}

\p{Accordingly, the \i{logicomorphic}/\i{interpretive} distinction {\mdash} along with the 
overall contrast between linguistic and extralinguistic 
aspects of meaning {\mdash} are contrasts between sentences that become manifest 
in the compositional maxims evident at subsentence (phrasal and 
inter-word) scales.  We can apply all four criteria to 
estimate the cognitive as well as syntactic and semantic 
paradigms in effect for given inter-word pairs and phrasal structure; 
identifying sentences as logicomorphic or interpretive propagates down 
to how phrasal and interword patterns should be analyzed.  
With this in mind, having presented certain claims as to the 
holistic nature of sentences \visavis{} propositional content, 
I will now switch attention to the composition of 
sentences from the interword level upward.}

\p{}

\end{sentenceList}
}
