
\documentclass{beamer}

\usepackage{setspace}
\usepackage{varioref}

%%??\usepackage{spath3}
%\usepackage{pstricks}

\usepackage{amssymb}
\usepackage{textcomp}
\usepackage{tikz}

\newcommand{\colorbullet}[1]{{\color{#1}\ensuremath{\bullet}}}

%\usetikzlibrary{shapes,snakes}
%%?\usepackage[backslant]{aurical}

%%?\usepackage{hanging}

%\usepackage{DejaVuSerif}

\usepackage{setspace}

\usepackage[outline]{contour}



%{{\color{red!60!yellow}{\textbf{#1}}}}}}}

\usepackage{microtype}
\usepackage{hyphenat}
%%?\usepackage{dashrule}
%\usepackage[usenames,dvipsnames]{xcolor}

\usetikzlibrary{arrows, positioning, shapes}
\usetikzlibrary{backgrounds}


%%?\usepackage{wrapfig}

\usetikzlibrary{arrows, positioning, shapes}

%
%
%\usetheme{Singapore}

%\usetheme{AnnArbor}
%\usecolortheme{spruce}

\usetheme{BerkeleyAA}
\usecolortheme{spruceaa}

\def\swidth{1cm}
\setbeamersize{sidebar width left=\swidth}
%\addtobeamertemplate{frametitle}{\vspace*{6cm}}{\vspace*{0.3cm}}

\usepackage{mdframed}

\newcommand{\ft}[1]{\vspace*{.25cm}\raisebox{-.45cm}{%
\contour{MSUgreen!50!yellow}{{\protect\Huge{\protect\textbf{#1}}}}}}


%C:/dgch-dev/cpp/testdia/bulletin/reslides/beamerthemeBerkeley.sty
%:/dgch-dev/cpp/testdia/bulletin/reslides/beamerthemeAnnArbor.sty
%C:/dgch-dev/cpp/testdia/bulletin/reslides/beamerthemeAntibes.sty
%C:/dgch-dev/cpp/testdia/bulletin/reslides/beamerthemeBergen.sty

%\usecolortheme{frigatebird}
%\usecolortheme{spruce}
%
%\usetheme{Berlin}
%
%\usecolortheme{spruce}
%\usecolortheme{wolverine}

\setbeamertemplate{blocks}[rounded][shadow=true]
\setbeamertemplate{frametitle}[default][center]
\setbeamertemplate{caption}[numbered]

\setlength{\paperwidth}{10.in}
\setlength{\paperheight}{7.5in}
\setlength{\textwidth}{9.in}
\setlength{\textheight}{6.5in}

\newcommand{\curlyquote}[1]{{\fontfamily{gar}\selectfont{``}}#1%
{\fontfamily{gar}\selectfont{''}}}

\newcommand{\curlyapos}[1]{{\fontfamily{gar}\selectfont{'}}}

\newcommand{\cfbox}[2]{%
	\colorlet{currentcolor}{.}%
	{\color[rgb]{#1}%
		\fbox{\color{currentcolor}#2}}%
}


%%?\usecolortheme{spruce}

%\setbeamercolor{block title}{red}   
%\setbeamercolor{block title}{red}   

%%\usecolortheme{orchid}
%\usepackage{float}

%\setbeamerfont{itemize/enumerate body}{size=\small}
%\setbeamerfont{normaltext}{size=\}
%\setbeamerfont{block body}{size=\small}
\setbeamerfont{block title}{size=\large}
%\setbeamerfont{block body example}{size=\tiny}

\newcommand{\colortriangle}{{\color[rgb]{0.45, 0.4, 0.28}$\blacktriangleright$}}
	
\newcommand\FourQuad[4]{%
	
	\begin{minipage}[b][.45\textheight][t]{.47\textwidth}#1\end{minipage}\hfill%
	\begin{minipage}[b][.45\textheight][t]{.47\textwidth}#2\end{minipage}\\[0.5em]
	\begin{minipage}[b][.45\textheight][t]{.47\textwidth}#3\end{minipage}\hfill
	\begin{minipage}[b][.45\textheight][t]{.47\textwidth}#4\end{minipage}%
}

\newcommand\ThreeQuad[3]{%
	\hspace{2pt}\begin{minipage}[b][.23\textheight][c]{.99\textwidth}#1\end{minipage}\hfill%
	\begin{minipage}[b][.75\textheight][t]{.55\textwidth}#2\end{minipage}\hfill
	\begin{minipage}[b][.75\textheight][t]{.43\textwidth}#3\end{minipage}%
}

\newcommand\TwoQuad[2]{%
	\begin{minipage}[b][.75\textheight][t]{.55\textwidth}#1\end{minipage}\hfill
	\begin{minipage}[b][.75\textheight][t]{.43\textwidth}#2\end{minipage}%
}

\newcommand\OneQuad[1]{%
	\begin{minipage}[b][.45\textheight][t]{1.01\textwidth}#1\end{minipage}\hfill
	%\begin{minipage}[b][.75\textheight][t]{.43\textwidth}#2\end{minipage}%
}


\newenvironment{quadblock}[1]{\begin{block}{\begin{center}\Large{%
\colorbox[rgb]{0.25,0.1,0.25}{\color{yellow}{\textbf{#1}}}
}\end{center}}%
\begin{minipage}[c]{.97\textwidth}\vspace{1em}
}{
\end{minipage}
\end{block}}


\newenvironment{mpblock}[1]{\begin{block}{\begin{center}\Large{%
\colorbox[rgb]{0.25,0.1,0.25}{\color{yellow}{\textbf{#1}}}
}\end{center}}%
\begin{minipage}[c]{.97\textwidth}\vspace{1em}
\begin{textsf}}{\end{textsf}
\end{minipage}
\end{block}}

\usepackage{aurical}

%%?\usepackage{pbsi}
%\usepackage{oesch}
%\usepackage{LobsterTwo}

%\newcommand{\doubleCenteredFramebox}[1]{%
%\begin{center}%
%\framebox{%
%\begin{minipage}{0.85\textwidth}\begin{center}%
%#1\end{minipage}\end{center}}\end{center}}

%\setlength{\fboxsep}{0.15em}



\newcommand{\doubleCenteredFramebox}[1]{%
\setlength{\fboxsep}{2em}\setlength{\fboxrule}{2pt}
\begin{center}\cfbox{0.1, 0.5, 0.8}{\begin{minipage}{0.85\textwidth}\begin{center}\setlength{\fboxsep}{.25em}#1
	\end{center}\end{minipage}}\end{center}}


\newcommand{\halfPageFramebox}[1]{%
	\setlength{\fboxsep}{2em}\setlength{\fboxrule}{1pt}
	\begin{center}\cfbox{0.1, 0.5, 0.8}{\begin{minipage}{0.5\textwidth}\begin{center}\setlength{\fboxsep}{.25em}#1
				\end{center}\end{minipage}}\end{center}}

\newcommand{\sfitem}[1]{\item \LARGE{\textbf{\textsf{#1}}}}				

\renewcommand*\rmdefault{cmfib}

\newcommand{\scitem}[1]{\item {\Large{{\fontfamily{pag}\selectfont \textbf{#1}}}}}			
\newcommand{\smscitem}[1]{\item {\small{{\fontfamily{pag}\selectfont \textbf{#1}}}}}			


\newcommand{\embitem}[1]{\item \textbf{{\large #1}}}			


\newcommand{\highlightframebox}[1]{\cfbox{1,.5,1}{#1}}

%\begin{minipage}[.45\textheight][t]}{\end{minipage}\end{block}}

\newcommand{\manyasciimacron}{\textasciimacron\textasciimacron\textasciimacron%
\textasciimacron\textasciimacron\textasciimacron\textasciimacron\textasciimacron\textasciimacron%
\textasciimacron\textasciimacron\textasciimacron\textasciimacron\textasciimacron\textasciimacron%
\textasciimacron\textasciimacron\textasciimacron\textasciimacron\textasciimacron\textasciimacron%
}

\newcommand{\manyemdash}{\texttwelveudash\texttwelveudash\texttwelveudash\texttwelveudash\texttwelveudash%
 \texttwelveudash\texttwelveudash\texttwelveudash\texttwelveudash\texttwelveudash\texttwelveudash\texttwelveudash%
 \texttwelveudash\texttwelveudash\texttwelveudash\texttwelveudash\texttwelveudash\texttwelveudash\texttwelveudash%
}


\newenvironment{lightquadblock}[1]{\begin{center}\LARGE{%
			\colorbox[rgb]{.99,.99,0.66}{\color[rgb]{.14,.14,.04}{\hspace{-1em}\makebox[\textwidth]{#1}}}
		}\end{center}
		%\begin{framebox}
		\begin{minipage}[c]{.9\textwidth}\vspace{1em}
		}{
	\end{minipage}}
	
	
	\newenvironment{lightqblock}[1]{\begin{center}\LARGE{%
				\colorbox[rgb]{.90,.90,0.90}{\cfbox{.96,.94,.96}{\color[rgb]{.2,.06,.04}{\hspace{-1em}\makebox[\textwidth]{#1}}}}
			}\end{center}
			%\begin{framebox}
			\begin{minipage}[c]{.99\textwidth}\vspace{1em}
			}{
		\end{minipage}}
		

\newcommand\TwoQuadV[4]{%
	\begin{minipage}[b][.20\textheight][t]{1.04\textwidth}\begin{lightqblock}{#1}#2\end{lightqblock}\end{minipage}
	\begin{center}\begin{minipage}[t][.78\textheight][t]{.9\textwidth}%
		\begin{quadblock}{#3}\begin{minipage}{\textwidth}#4\end{minipage}\end{quadblock}\end{minipage}\end{center}%
}


%\newcommand{\conversationPatterns}{\framebox{%
%Conver\-\\
%sation\\
%Patterns}}

\newcommand{\reParser}{	
\begin{minipage}{32pt}\centering{\begin{scriptsize}\textbf{Relae\\\vspace{-6pt}Parser}\end{scriptsize}} 
\end{minipage}}

\usetikzlibrary{decorations.markings}
\usetikzlibrary{decorations.pathmorphing}

\newcommand{\q}[1]{``#1"}

\definecolor{quadgrey}{rgb}{.9,.9.,.89}
\colorlet{snakegrey}{black!80!white}
\colorlet{blbl}{blue!20!black}

\definecolor{lqboutercolor}{rgb}{.93,.91,.92}
\definecolor{lqbinnercolor}{rgb}{.98,.98,.98}

\definecolor{blGreen}{rgb}{.2,.7,.3}
\definecolor{darkRed}{rgb}{.2,.0,.1}

\definecolor{postLinkColor}{rgb}{.5,.5,.1}

\definecolor{fcBoxColor}{rgb}{.8,.6,.3}

\definecolor{BlueGreen}{rgb}{.1,.6,.4}

\colorlet{ry}{red!80!yellow}
\colorlet{rblue}{red!70!blue}
\colorlet{rb}{rblue!40!black}
\colorlet{bg}{blGreen!40!black}

\newcommand{\hc}[1]{{\Huge %
{\contour{ry}{\protect\contour{rb}{{\protect\color{blGreen!40!black}{\protect\textbf{#1}}}}}}}}

\colorlet{yy}{yellow!80!blue}
\colorlet{yo}{yellow!80!orange}
\colorlet{yw}{yellow!20!white}



%\usepackage{microtype}



\newcommand{\yhc}[1]{{ %
{\contourlength{1.2pt}
{{\protect\contour{yw!50}{\protect\contour{yo!50}{{\protect\color{yy}{\protect\textbf{%
{\protect\Huge{\protect\textls[240]{#1}}}}}}}}}}}}}


\newcommand{\cframedbox}[1]{\begin{mdframed}
[linecolor=rb!85!red,linewidth=0.4mm]#1\end{mdframed}}

\newcommand{\cframedboxyellow}[1]{\begin{mdframed}
[linecolor=rb!45!yellow,linewidth=0.4mm]#1\end{mdframed}}

\newenvironment{postfragment}{\begin{minipage}{5.4cm}\begin{small}}
{\end{small}\end{minipage}}

\newenvironment{postComment}{\begin{minipage}{8cm}\begin{small}}
{\end{small}\end{minipage}}

\newenvironment{postLongComment}{\begin{minipage}{9cm}\begin{small}}
		{\end{small}\end{minipage}}


\newenvironment{tightcenter}{%
	\setlength\topsep{0pt}
	\setlength\parskip{0pt}
	\begin{center}
	}{%
\end{center}
}

\newcommand{\tightcenterline}[1]{\begin{tightcenter}#1\end{tightcenter}}



\definecolor{postBkgColor}{rgb}{.95,.85,.95}
\definecolor{postCommentBkgColor}{rgb}{.85,.85,.95}

\definecolor{grammarArrowColor}{rgb}{.85,.85,.45}

\tikzstyle{postStyle}=[draw=yellow!120,rounded corners,fill=postBkgColor,thin,inner sep=.2cm]
\tikzstyle{postCommentStyle}=[draw=postCommentBkgColor,double,fill=postCommentBkgColor,thick,inner sep=.2cm]

\tikzstyle{sdComponent}=[double,rounded corners,draw=brown!120,fill=blue!50,thin,inner sep=.2cm]
\tikzstyle{cnvComponent}=[double,shape=diamond,rounded corners,draw=brown!120,fill=blue!50,thin,inner sep=0cm]

\tikzstyle{baseStyle}=[fill=purple!50,draw=cyan!20,ultra thick,double,shape=diamond,inner sep=.15cm]
\tikzstyle{componentStyle}=[fill=red!50,draw,shape=ellipse,inner sep=.15cm]

\tikzstyle{componentExtendedStyle}=[fill={rgb:red,10;green,10;blue,2},draw,shape=rectangle,inner sep=.15cm]

\newcommand{\emphcolor}{%
	\colorlet{currentcolor}{.}%
	\color[rgb]{.4,.4,.4}}

\newcommand{\ncolor}{%
	\color{currentcolor}}


\definecolor{slidePartHeadColor}{rgb}{0,.2,.1}

\newcommand{\slidePartHead}[1]{{\fontfamily{pnc}\selectfont\color{slidePartHeadColor}\LARGE#1}}

\newcommand{\slidePartHeadCenter}[1]{\slidePartHead{\begin{minipage}{\textwidth}\begin{center}#1\end{center}\end{minipage}}}

\newcommand{\componentLabel}[1]{\begin{minipage}{3.5cm}\begin{center}#1\end{center}\end{minipage}}
\newcommand{\baseComponentLabel}[1]{\begin{minipage}{2cm}\textbf{\begin{center}#1\end{center}}\end{minipage}}

\newcommand{\componentExtendedLabel}[1]{\begin{minipage}{10cm}\begin{center}\textbf{#1}\end{center}\end{minipage}}

\newcommand{\cstd}[1]{\textbf{{\color[rgb]{.4,.4,.1}#1}}}

\usepackage{setspace}

%\fontfamily{pag}\selectfont \textbf{
\newcommand{\raiseBoxL}[1]{\makebox{\raisebox{.1em}{{\fontfamily{put}\fontseries{sb}\selectfont #1}\hspace{1.2em}}}}
%	\begin{textsf}{\small}{}\end{textsf}\end{minipage}}

\newcommand{\doubleFrame}[1]{%
\fcolorbox{fcBoxColor!50}{gray!50}{
%{\begin{center}
\begin{minipage}{.985\textwidth}
%\begin{center}	
\hspace{2pt}\vspace*{1pt}		
\fcolorbox{fcBoxColor!90}{white}{%
\begin{minipage}{.98\textwidth}%
\begin{center}\begin{minipage}{.955\textwidth}				
\begin{spacing}{1}\vspace{1em}\part{title}#1\end{spacing}%
\end{minipage}\end{center}
\end{minipage}}
%\end{center}
\end{minipage}}
%\end{center}}
}%

\usepackage{aurical}
\usepackage[T1]{fontenc}
\usepackage{libris}
\usepackage{relsize}

\newcommand{\VersatileUX}{{\color{red!85!black}\Fontauri Versatile}%
{{\fontfamily{qhv}\selectfont\smaller UX}}}


\newcommand{\doubleFrameTwo}[2]{%
\begin{center}
\begin{minipage}{\textwidth}	
\fcolorbox{fcBoxColor!50}{gray!50}{
%	\begin{center}
\begin{minipage}{\textwidth}
	\begin{center}
			%\hspace{4pt}\vspace*{2pt}		
			\fcolorbox{fcBoxColor!90}{white}{%
				\begin{minipage}{.95\textwidth}%
\begin{center}\begin{minipage}{.91\textwidth}#1\end{minipage}\end{center}%
				\end{minipage}}\vspace*{.5em}\\
%\vspace*{2pt}
%\hspace{4pt}				
\fcolorbox{fcBoxColor!90}{white}{
		\begin{minipage}{.94\textwidth}
\begin{center}\begin{minipage}{.91\textwidth}#2\end{minipage}\end{center}%
		\end{minipage}}
\end{center}		
\end{minipage}
%
}\par 
\end{minipage}
\end{center}		
}

\newcommand{\itup}[1]{\raisebox{2pt}{{\normalsize(#1)}}}

\newcommand{\FER}{{\color[rgb]{0.1,0.25,0.6}{FER}}}
\newcommand{\BER}{{\color[rgb]{0.1,0.25,0.6}{BER}}}
\newcommand{\IR}{{\color[rgb]{0.1,0.25,0.6}{IR}}}

\newcommand{\FrontEnd}{{\color[rgb]{0.1,0.25,0.6}{Front-End}}}
\newcommand{\BackEnd}{{\color[rgb]{0.1,0.25,0.6}{Back-End}}}
\newcommand{\Intermediate}{{\color[rgb]{0.1,0.25,0.6}{Intermediate}}}

%\newcommand{\doubleFrameTwo}[2]{%
%\fcolorbox{fcBoxColor}{gray!50}{
%\begin{minipage}{.98\textwidth}
%\hspace{4pt}\vspace*{2pt}	
%\fcolorbox{fcBoxColor}{white}{%
%\begin{minipage}{.95\textwidth}#1
%\end{minipage}}#1
%\fcolorbox{fcBoxColor}{white}{%
%\begin{minipage}{.95\textwidth}#2%
%\end{minipage}}
%}


%\newenvironment{doubleFrame}{%
%\fcolorbox{fcBoxColor}{gray!50}{\begin{minipage}{\textwidth}}
%{\end{minipage}}}

%\newenvironment{doubleFrame}{% \fcolorbox{fcBoxColor}{gray!50}{%
%\fbox{\begin{minipage}{\textwidth}%
%}
%{\end{minipage}}}
%}}
\usetikzlibrary{shadows}
\definecolor{logoRed}{rgb}{.3,0,0}
\definecolor{logoPeach}{RGB}{255, 159, 102}
\definecolor{logoCyan}{RGB}{66, 206, 244}
\definecolor{logoBlue}{RGB}{4, 2, 25}

\usepackage{changepage}

\newcommand{\nodeincludegraphicsLOCTRV}[7]{
\node[anchor=south west,inner sep=0,thick,
drop shadow={top color=logoBlue!50!logoCyan,
              bottom color=logoRed!50!logoCyan,
              shadow xshift=-1pt,
              shadow yshift=-3pt,
              rounded corners}] (image) at (#1){%
\fcolorbox{logoRed}{logoPeach!40!logoRed}{\includegraphics[scale=#2,trim={#5 #4 #3 #6},clip]{#7}}};}

%\newcommand{\doubleFrame}[1]
\newcommand{\nodeincludegraphics}[2][0.8\textwidth]{
\node[anchor=south west,inner sep=0] (image) at (0,0){%
\includegraphics[width=#1]{#2}};}

\newcommand{\nodeincludegraphicsR}[2][1]{
\node[anchor=south west,inner sep=0] (image) at (0,0){%
\includegraphics[scale=#1]{#2}};}

\newcommand{\nodeincludegraphicsAS}[1]{
\node[anchor=south west,inner sep=0] (image) at (0,0){%
\includegraphics{#1}};}

%\newcommand{\nodeincludegraphicsLOC}[3]{
%\node[anchor=south west,inner sep=0] (image) at (#1){%
%\includegraphics[scale=1]{#3}};}

\newcommand{\nodeincludegraphicsLOC}[3]{
\node[anchor=south west,inner sep=0] (image) at (#1){%
\includegraphics[scale=#2]{#3}};}

\newcommand{\nodeincludegraphicsLOCTR}[5]{
\node[anchor=south west,inner sep=0] (image) at (#1){%
\includegraphics[scale=#2,trim={0 #4 #3 0},clip]{#5}};}

\newcommand{\nodeincludegraphicsTR}[3]{
\node[anchor=south west,inner sep=0] (image) at (0,0){%
\includegraphics[trim={0 #2 #1 0},clip]{#3}};}

\newcommand{\nodeincludegraphicsTRRS}[6]{
\node[anchor=south west,inner sep=0,scale=#1] (image) at (0,0){%
\includegraphics[trim={#4 #3 #2 #5},clip]{#6}};}


\newcommand{\ann}[9]{%
	\path [draw=#1,draw opacity=#2,line width=#3, fill=#4, fill opacity = #5, even odd rule] %
	(#6) ellipse(#7 and #8) ellipse(#7*#9 and #8*#9);}

\newcommand{\polyann}[9]{%
\path [draw=#1,draw opacity=#2,line width=#3, fill=#4, fill opacity = #5, even odd rule] %
	node [minimum size=#8,regular polygon,regular polygon sides=#7] at (#6) {}
	node [minimum size=#9,regular polygon,regular polygon sides=#7] at (#6) {};}
	

%\newcommand{\polyann}[9]{%
%	\path [draw=#1,draw opacity=#2,line width=#3, fill=#4, fill opacity = #5, even odd rule] %
%	path minimum size=2cm,regular polygon,regular polygon sides=6(#7 and #8) ellipse(#7*#9 and %#8*#9);}


\makeatletter
\newcommand*\getX[1]{\expandafter\getX@i#1\@nil}
\newcommand*\getY[1]{\expandafter\getY@i#1\@nil}
\def\getX@i#1,#2\@nil{#1}
\def\getY@i#1,#2\@nil{#2}
\makeatother

\newcommand{\rectann}[9]{%
	\path [draw=#1,draw opacity=#2,line width=#3, fill=#4, fill opacity = #5, even odd rule] %
	(#6) rectangle(\getX{#6}+#7,\getY{#6}+#8)
	({\getX{#6}+((#7-(#7*#9))/2)},{\getY{#6}+((#8-(#8*#9))/2)}) rectangle %
	({\getX{#6}+((#7-(#7*#9))/2)+#7*#9},{\getY{#6}+((#8-(#8*#9))/2)+#8*#9});}

\newcommand{\rectanneatdbl}[9]{%
	\path [draw=#1,draw opacity=#2,line width=#3, fill=#4, fill opacity = #5, even odd rule] %
	(#6) rectangle(\getX{#6}+#7,\getY{#6}+#8)
	({\getX{#6}+\getX{#9}},{\getY{#6}+\getY{#9}}) rectangle %
	({\getX{#6}+#7-\getX{#9}},{\getY{#6}+#8-\getY{#9}})
	;}

\newcommand{\rectanneat}[9]{%
	\path [draw=#1,draw opacity=#2,line width=#3, fill=#4, fill opacity = #5, even odd rule] %
	(#6) rectangle(\getX{#6}+#7,\getY{#6}+#8)
	({\getX{#6}+(#7/abs(#7))*#9},{\getY{#6}+(#8/abs(#8))*#9})
	rectangle({\getX{#6}+#7-(#7/abs(#7))*#9},{\getY{#6}+#8-(#8/abs(#8))*#9});}

\newcommand{\colorarr}[8]{
	\draw [#1, draw=#2,draw opacity=#3,
	fill=#4,fill opacity=#5,line width=#6] (#7) to (#8);
}

\colorlet{brred}{brown!53!red}
\colorlet{grred}{grammarArrowColor!40!red!60}


\begin{document}

%\input{cqlDescriptionSlide}



\begin{frame}{\ft{The \curlyquote{Re-PDF} Customizable}  
\ft{Interactive PDF Viewer for Real Estate}}

%{\LARGE
%\vspace{2em}{\color{darkRed}

%\colorbox{white}
%{\begin{minipage}{.75\textwidth}\centering 
					
%{\color[rgb]{.5,.25,.2}{\LARGE\fontfamily{bsi}\selectfont
% \textbf{{\it Create Interactive Desktop Apps for 
% Prospective Buyers \\-- Customized for each Realtor}}}}
%\vspace{1em}\\
\OneQuad
{
\vspace{-5em}
\begin{quadblock}
{{ {\parbox{18cm}{\centering\fontfamily{lmdh}
%\fontsize{110}{40}
\fontseries{b}\selectfont
{
\yhc{Delivering a package of}\vspace{.6em} 
\yhc{realty photos and}\vspace{.6em} 
\yhc{property information}\vspace{.7em}
\yhc{embedded in PDF files}\vspace{.6em}
}
} }}} \vspace{1em}

\vspace{2em}

\colorbox{white}{\setlength{\fboxsep}{1.5em}
   \framebox{\begin{minipage}{.85\textwidth}\centering
	  {\color{blGreen!40!blbl}\textbf{{\Huge 
	  Linguistic Technology Systems, Inc. \\
	  Amy Neustein, Ph.D., \\
	  Founder and CEO \\
	  amy.neustein@verizon.net \\
	  201-224-5096}}} \vspace{1em}\\
%	  Lead Software Architect: Nathaniel Christen
	 \end{minipage}}}
\end{quadblock}
}
\end{frame}




\begin{frame}{\ft{Loading Embedded Files}}

\doubleFrame{Re-PDF (Real Estate PDF Viewer): photos and information are 
embedded in PDF files and are automatically extracted when the document is 
opened with this special viewer, which can be customized for each realtor.}

\begin{tikzpicture}
\nodeincludegraphicsAS{screenshots/ss-extdia.png}
%\draw [solid] (12,6) ellipse(8 and 3);
%\draw [solid] (12,6) ellipse(7 and 2);

%\path [draw=red,fill=gray, even odd rule, fill opacity = 0.5] (12,6) 
% ellipse(8 and 3) ellipse(7 and 2);

\ann{BlueGreen}{0.3}{1mm}{grammarArrowColor}{0.5}{12,6}{8}{3}{0.85}
%\node [anchor=west] (note) at (-1,3) {\Large Note};
%\draw [-latex, ultra thick, red] (note) to[out=0, in=-120] (0.48,0.80);

%\path [draw=red,fill=gray, even odd rule, fill opacity = 0.5] (12,6) ellipse(8 and 3);
\end{tikzpicture}

\end{frame}




\begin{frame}{\ft{Viewing Re-PDF Files}}

\doubleFrame{An Re-PDF file serves two purposes: first, 
it embeds pictures and data about 
individual properties so this information can be 
conveniently downloaded and saved, like a zipped folder;  
second, it displays important facts about properties 
--- similar to an MLS website.}

\begin{tikzpicture}
\nodeincludegraphicsTR{6.18cm}{0cm}{screenshots/ss-loaded.png}

 \node [anchor=west,fill=brown!20!white,inner sep=7, text width=22cm]
  (longnote) at (0.5,5) {%  %{\color{rb!85!red}{
  {\cframedbox{\large \textbf{An Re-PDF file looks like 
  ordinary PDF when viewed with a conventional PDF viewer, 
  but it also packages together embedded files 
  and RETS data, for use by customized Re-PDF applications.}}}};


\end{tikzpicture}


\end{frame}




\begin{frame}{\ft{Using Embedded RETS Data}}

\doubleFrame{Each Re-PDF file is a vehicle for 
delivering a pre-selected assembly of RETS 
(Real Estate Transaction Standard) data to 
prospective buyers.  With RETS data selected 
in advance, the Re-PDF application 
can focus on presenting this data 
in the most compelling, interactive ways.}

\begin{tikzpicture}
\nodeincludegraphicsTR{6.18cm}{1cm}{screenshots/ss-loaded.png}

 \node [anchor=west,fill=brown!20!white,inner sep=7, text width=22.7cm]
  (longnote) at (0.2,7.8) {%  %{\color{rb!85!red}{
  {\cframedbox{\large 
  {%
  \begin{minipage}{21.9cm}
  {\textbf{Re-PDF can benefit both developers and 
  end-users:  For end-users, Re-PDF packages important 
  RETS data so that RETS information does not need to be 
  acquired piece-by-piece, with repeated network requests.  As 
  a result, Re-PDF presentations can be experienced offline, 
  and the user interface will not be slowed down by network latency.  
  For developers, Re-PDF allows the transformation of RETS 
  data into GUI-based presentations.  
   is divided into 
  two logically separate tasks: first, selecting RETS data to 
  package into a localized data set, and, second, designing GUI 
  code which implements a fluid, desktop-style application-level 
  view onto this data set.
  \vspace*{0.5em}
  \\
  The pre-selected data --- called 
  a \curlyquote{Portable Resource Set} --- can be compiled from 
  conventional RETS servers and also from the newer 
  \curlyquote{Semantic} models of Real Estate data which use 
  RDF (Resource Description Format) and 
  \curlyquote{Ontologies}.  The architecture of 
  an engine used to build Re-PDF Portable Resource Sets 
  from RDF data is outlined on the next slide.}} 
  
  %{\color{blGreen!40!blbl}Figure \ref{fig:relae}}, 
  %appearing on the next slide.}}
  %
  
  \end{minipage}
  }
  }}};


\end{tikzpicture}


\end{frame}




\begin{frame}{\Huge{\textbf{\VersatileUX{} and APDL: Schematic Outline}}}
		
%		{\Large 
% {\color{BlueGreen}
% 	{\fontfamily{pbk}\fontshape{sc}\selectfont \raisebox{1pt}{Schematic Outline}}
% 	}}}}}
	
\tikzstyle{nlComponent}=[double,draw=brown!120,fill=blue!50,thin,inner sep=.2cm]
\tikzstyle{sdComponent}=[double,rounded corners,draw=brown!120,fill=blue!50,thin,inner sep=.2cm]
\tikzstyle{cnvComponent}=[double,shape=diamond,rounded corners,draw=brown!120,fill=blue!50,thin,inner sep=0cm]
\tikzstyle{cfStyle}=[fill=red!50,draw,shape=ellipse,inner sep=.15cm]

\tikzset{snake it/.style={decorate, 
decoration={snake, segment length=2mm, amplitude=.4mm}, 
draw=snakegrey!60!orange, line width=.5mm}}

%
\vspace*{2em}


\OneQuad
{
\vspace{-9em}
\begin{quadblock}{Source Text or APDL \raisebox{1pt}{\Large $\rightarrow$} 
 RelaeGraph \raisebox{1pt}{\Large $\rightarrow$} TripleStore 
 {\Large $\rightarrow$} \VersatileUX}
\raisebox{4em}{\begin{minipage}{\textwidth}
\begin{figure}
\cfbox{.9,.9.,.89}{\colorbox[rgb]{.9,.9,.88}{%	
\begin{minipage}{\textwidth}\doubleCenteredFramebox{%
\colorbox{quadgrey}{ \resizebox{!}{!}
	{\begin{tikzpicture}

\tikzstyle{TextPolygonNode}=[shape=regular polygon,regular polygon sides=15,double,draw=BlueGreen,fill=yellow!50,
text opacity=1,fill opacity=.9,text centered,inner sep=0]

\tikzstyle{TextStarNode}=[shape=star,star points=4,double,draw=BlueGreen,fill=yellow!50,
text opacity=1,fill opacity=.9,text centered,inner sep=-10pt,yscale=0.5]


\path 
 node (txtNode) at (0,0) [nlComponent, text width = 4cm,
  text centered] {Source Text Input}
 node (hiddenArcLeft) [above=of txtNode] {}
 node (hiddenCenter) [right=3 cm of txtNode] {}
 node (qaNode) [sdComponent, right=3 cm of hiddenCenter, text width = 4cm, text centered] 
 {Qt Application + APDL}
 node (hiddenArcRight) [above=of qaNode] {}

 node (rpNode) [above=of hiddenCenter] {}

 node (rpTxtNode) [cnvComponent, left=1cm of rpNode] {\reParser}
 node (rgNode) [above=of rpNode] {}
 node (rgNodePt) [above=-2mm of rgNode] {}
 node (rgTxtNode) [rounded corners,double,draw=darkRed,left=-12mm of rgNode.center] {RelaeGraph}
 node (IRNode) [cfStyle, above=11mm of rgNode] {IN MEMORY}

 node (LGNode) [cnvComponent, 
 below left=11mm and 2.5cm of IRNode.south west,inner sep=.05cm, draw=pink!120, text width=2cm, text centered] 
  {\footnotesize{Customizable Grammar}}
  
node (TRNode) [double,draw=pink!120,fill=gray!50,inner sep=.5em,above left = 3.5cm and 2.5cm of IRNode]
 {{\hspace*{.5em}}Triplestore/SDL\hspace*{1em}}

node (TRTextNode) [above left= .1 cm and -22mm of TRNode.north west] 
 {{\scriptsize Third-Party Triplestore or Semantic Data Lake}} 

node (TRDescNode) [below left = .1 cm and -39mm of TRTextNode.south west, 
text width = 35mm] 
{{\scriptsize --- Lisp or C++ clients}}


node (VERNode) [TextStarNode, %double,draw=pink!120,fill=gray!50,inner sep=.5em,
above right = 2.5cm and 2.8cm of IRNode]
 {{\hspace*{.5em}}\scalebox{1}[2]{\VersatileUX}\hspace*{1em}}

node (VERTextNode) [above right  = .1cm and .4cm of VERNode.north east] 
{{\scriptsize Qt Application Front-End}} 

node (VERDescNode) [below right = .1 cm and -2.3cm of VERTextNode.south east, 
text width = 2.7cm] 
{{\scriptsize --- Or other Qt library with APDL support}} 

 node (APDLRNode) [cnvComponent, 
 below left=11mm and -1mm of VERNode.south west,inner sep=.00cm, draw=pink!120, text width=1.75cm, text centered] 
 {\scriptsize{APDL Stream Reader}}

 node (APDLRSNode) [cnvComponent, 
 above left=-6mm and 13mm of VERNode.north west,inner sep=.00cm, draw=pink!120, text width=1.75cm, text centered] 
 {\scriptsize{APDL Stream Reader}};



\draw[ ->>, draw=darkRed, fill=quadgrey, 
line width=0.5mm, >= triangle 45, shorten <= .25cm, shorten >= .25cm ]
 (IRNode.west) -|  % node %{{\manyemdash}\textit{compiles to}...} 
  (TRNode.south);
  
  

%\draw[ |-,-|, <->, thick, double equal sign distance, >= stealth, shorten <= .25cm, shorten >= .25cm ] 
%(txtNode.east) to node 
%{\raisebox{2.5em}{\begin{minipage}{4em}%
%\centering{\begin{footnotesize}\color{darkRed}{Matched Object Fields}\end{footnotesize}}
%\end{minipage}}} 
%(dsNode.west);

%\draw [thick, shorten <= .25cm, shorten >= .25cm, -> ]
% (txtNode) edge [bend left=30,looseness=1,draw=orange] (LGNode);

%\draw [thick, shorten <= .25cm, shorten >= .25cm, <- ]
%(LGNode) edge [bend left=30,looseness=1,draw=orange] (FERNode);

%\draw [thick, shorten <= .25cm, shorten >= .25cm, -> ]
%(LGNode) edge [bend left=30,looseness=1,draw=orange] (IRNode);

%\draw [thick, shorten <= .25cm, shorten >= .25cm, -> ]
%(IRNode) edge [bend left=30,looseness=1,draw=orange] (PRNode);


%\draw [thick, shorten <= .25cm, shorten >= .25cm, -> ]
%(dsNode) edge [bend right=30,looseness=1,draw=orange] (PRNode);

%\draw [thick, shorten <= .25cm, shorten >= .25cm, -> ]
%(PRNode) edge [bend right=30,looseness=1,draw=orange] (BERNode);


%\draw[ |-,-|, ->>, thin, >= triangle 45, shorten <= .25cm, shorten >= .25cm ]
% (FERNode.south) to % node %{{\manyemdash}\textit{compiles to}...} 
%  (IRNode.north);
 
%\draw[ |-,-|, ->>, thin, >= triangle 45, shorten <= .25cm, shorten >= .25cm ]
%(IRNode.north) to %node {{\manyemdash}...} 
% (BERNode.south);
 
%\draw[ |-,-|, ->, very thick, >= latex', shorten <= .15cm, shorten >= .05cm ]
% (rgNode.north) to (IRNode.south) % to node {{\manyasciimacron}\raisebox{.5em}{\textit{queries}...}}  
 %(dataModel.north)
% ;

\draw[<-, thick, >= angle 90, shorten >= .25cm ]
 (rgNodePt) edge [bend right=100,looseness=1,draw=purple] (hiddenArcLeft.west);
\draw[-, thick, shorten >= .25cm ]
 (rgNodePt) edge [bend right=100,looseness=1.05,draw=black] (hiddenArcLeft.west);
\draw[-, thick, shorten >= .25cm ]
(rgNodePt) edge [bend right=100,looseness=1.1,draw=purple] (hiddenArcLeft.west);

\draw[<-, thick, >= angle 90, shorten >= .25cm ]
 (rgNodePt) edge [bend left=100,looseness=1,draw=purple] (hiddenArcRight.east);
\draw[<-, thick, >= angle 90, shorten >= .25cm ]
(rgNodePt) edge [bend left=100,looseness=1.05] (hiddenArcRight.east);
\draw[<-, thick, >= angle 90, shorten >= .25cm ]
(rgNodePt) edge [bend left=100,looseness=1.1,draw=purple] (hiddenArcRight.east);

\draw node (qaNodeN) [right = 2mm of qaNode.north] {};

%%\draw[ ->>, thick, shorten <= .25cm, >= triangle 60 ]%, shorten <= .15cm, shorten >= 1.5cm ]
\draw[ ->, thick, >= latex', shorten <= .25cm, shorten >= .55cm, double]
 (qaNodeN) to node[cnvComponent, %below right=-4mm and 2.8 cm of IRNode.south east,
 inner sep=.00cm, draw=pink!120, text width=1.75cm, text centered] 
 {\scriptsize{APDL Stream Writer}}  (rgNode.south);
%  {\raisebox{4.5em}{\begin{footnotesize}\hspace*{.75em}%
%  \color{darkRed}{APDL}\end{footnotesize}}} 
  
 
% \node (PRNode) [cnvComponent, below right=-4mm and 2.8 cm of IRNode.south east,inner sep=.00cm, draw=pink!120, text width=1.75cm, text centered] 
% {\scriptsize{APDL Stream Writer}};
  
 
% \draw[ <->>, thick, >= triangle 60, shorten <= .15cm, shorten >= 1.5cm ]
% (dataModel) to node 
% {\raisebox{8.5em}{\begin{footnotesize}\hspace*{.75em}%
% 		\color{darkRed}{Object Oriented}\end{footnotesize}}} (dsNode);
 
% {\raisebox{8.5em}{\begin{footnotesize}\hspace*{.75em}%
%\color{darkRed}{Object Oriented}\end{footnotesize}}} (dsNode);

\draw node (TXTNodeN) [left = 1mm of txtNode.north] {};

%\draw  [ <<-, thick, >= triangle 60, shorten <= .25cm, shorten >= .25cm ]
% (dataModel) -- (TXTNodeN) node (CusGNode) [pos = 0.275, sloped, text width=2.20cm, text centered] 
%  {\color{darkRed}{\colorbox{white}{Customiable Grammar}}};


%\draw node (LAFLabelHiddenNode)
%[below = 0.75 cm of dataModel.south west]{};
%  
%\draw node (LAFLabelNode)
% [left = 9 mm of LAFLabelHiddenNode.east,
%  text width=1.5cm, align=right] {{\tiny 
%  		\colorbox{white}{Linguistic} \colorbox{white}{Annotation} \colorbox{white}{Framework}}};
 

%%%

\begin{frame}{\ft{Using Embedded RETS Data}}

\doubleFrame{Each Re-PDF file is a vehicle for 
delivering a pre-selected assembly of RETS 
(Real Estate Transaction Standard) data to 
prospective buyers.  With RETS data selected 
in advance, the Re-PDF application 
can focus on presenting this data 
in the most compelling, interactive ways.}

\begin{tikzpicture}
\nodeincludegraphicsTR{6.18cm}{1cm}{screenshots/ss-loaded.png}

 \node [anchor=west,fill=brown!20!white,inner sep=7, text width=22.7cm]
  (longnote) at (0.2,7.8) {%  %{\color{rb!85!red}{
  {\cframedbox{\large 
  {%
  \begin{minipage}{21.9cm}
  {\textbf{Re-PDF can benefit both developers and 
  end-users:  For end-users, Re-PDF packages important 
  RETS data so that RETS information does not need to be 
  acquired piece-by-piece, with repeated network requests.  As 
  a result, Re-PDF presentations can be experienced offline, 
  and the user interface will not be slowed down by network latency.  
  For developers, Re-PDF allows the transformation of RETS 
  data into GUI-based presentations.  
   is divided into 
  two logically separate tasks: first, selecting RETS data to 
  package into a localized data set, and, second, designing GUI 
  code which implements a fluid, desktop-style application-level 
  view onto this data set.
  \vspace*{0.5em}
  \\
  The pre-selected data --- called 
  a \curlyquote{Portable Resource Set} --- can be compiled from 
  conventional RETS servers and also from the newer 
  \curlyquote{Semantic} models of Real Estate data which use 
  RDF (Resource Description Format) and 
  \curlyquote{Ontologies}.  The architecture of 
  an engine used to build Re-PDF Portable Resource Sets 
  from RDF data is outlined on the next slide.}} 
  
  %{\color{blGreen!40!blbl}Figure \ref{fig:relae}}, 
  %appearing on the next slide.}}
  %
  
  \end{minipage}
  }
  }}};


\end{tikzpicture}


\end{frame}


 

\draw[ ->, thick, >= latex', shorten <= .25cm, shorten >= .25cm ]
(rpTxtNode)
to node (AIHideNode) [inner sep=0cm, pos=0.75]
{} 
(rgNode.south) [double];

%\draw node [below right = .3cm of AIHideNode.south] %{\colorbox{white}{\begin{minipage}{9em}\begin{scriptsize}\color{blGreen}
%{Application Provenance and Description Language}\end{scriptsize}
%		\end{minipage}}}; 

\node (ADPLNode) [above right = 30mm and 62mm of AIHideNode.south,
fill opacity=1,fill=blGreen!60,text width=25mm, text centered] 
{APDL};

\node (ADPLTextNode) [below=0 mm of ADPLNode.north,
fill opacity=1,draw=blGreen!60, line width=1mm, text centered, minimum width=3cm, minimum height=1cm,
text width=33mm] 
{
 \vspace{4mm}
 \\
 \scriptsize{
 Application Provenance and Description Language, for: 
 \textbf{\tiny{\\
 \colorbullet{green!70!red!20!black} Application Description Ontology 
 \colorbullet{green!20!red!50!black} Autogenerated Provenance Data
 \colorbullet{blue!70!red!20!black} Data Serialization 
 \colorbullet{blue!20!red!50!black} GUI Documentation
 \colorbullet{green!30!red!30!blue} Workflow Implementation}}}
	
};



\node (VUXNode) [above left = 49mm and 62mm of AIHideNode.south,
fill opacity=1,fill=orange!40!yellow,text width=25mm, text centered] 
{\VersatileUX};

\node (VUXTextNode) [below=0 mm of VUXNode.north,
fill opacity=1,draw=orange!40!yellow, line width=1mm, text centered, minimum width=3cm, minimum height=1cm,
text width=33mm] 
{
 \vspace{4mm}
 \\
 \scriptsize{
 Application \\Development Toolkit 
 \textbf{\tiny{\\
 \colorbullet{green!70!black} Qt-Based
 \colorbullet{red!70!black} Prioritizes Custom GUI Development
 \colorbullet{blue!70!black} Supports \curlyquote{Multi-Application} Workflows}}}
	
};



\node (RELGNode) [below right = 4mm and -12mm of VUXTextNode.south,
fill opacity=1,fill=darkRed!40!cyan,text width=25mm, text centered] 
{RelaeGraph};

\node (RELGTextNode) [below=0 mm of RELGNode.north,
fill opacity=1,draw=darkRed!40!cyan, line width=1mm, text centered, minimum width=3cm, minimum height=1cm,
text width=35mm] 
{
 \vspace{4mm}
 \\
 \scriptsize{
 \curlyquote{Regular Expression Labeled\\Annotation Engine} 
 \textbf{\tiny{\\
 \colorbullet{green!70!red!20!black} In-Memory Semantic Graphs
 \colorbullet{green!20!red!50!black} Uses Qt Collections
 \colorbullet{blue!70!red!20!black} Graph Manipulation via C++ Operators
 \colorbullet{blue!20!red!50!black} Graph Nodes Wrap C++ Pointers
 \colorbullet{green!30!red!30!blue} Customizable Parser for Graph Description Languages}}}
 
};


%\path node (IRDescNode) [above=7mm of IRNode.north, text width = 3cm, text centered] 
% {{\footnotesize Intermediate Representation (cross-platform)}} ; 

\node (NDPCloudNode) [TextPolygonNode,minimum height=2cm,minimum width=1cm,
 above right=20mm and -1.2cm of IRNode.north,
 text width=12mm,inner sep=-1mm,scale=1,text centered] 
{\textbf{\small{{\fontfamily{phv}\fontseries{b}\selectfont NDP Cloud}}} 
\small{\makebox{(C++ Http} Library)}};

\node (QtrNode) [sdComponent,minimum height=1cm,minimum width=1cm,
 above left=2mm and -6mm of IRNode.north,
 text width=5cm,inner sep=2mm,scale=1,text centered] 
{\textbf{\small{{\fontfamily{phv}\fontseries{b}\selectfont V-PRI}}} 
\small{(\VersatileUX{} \\Portable Resource Interface)}};

\draw[ ->>, draw=darkRed, fill=quadgrey, line width=0.5mm, 
>= triangle 45, shorten <= .25cm, shorten >= .25cm ]
 (TRNode.east) -|  % node %{{\manyemdash}\textit{compiles to}...} 
  (QtrNode.north);

\draw[ ->>, draw=darkRed, fill=quadgrey, line width=0.5mm,  >= triangle 45, shorten <= .25cm, shorten >= .25cm ]
 (IRNode.east) |-  % node %{{\manyemdash}\textit{compiles to}...} 
  (NDPCloudNode.south east);
  

\draw [ ->, line width=1mm, draw=orange, opacity=.8, bend left=50,looseness=1,
shorten >= 2mm, shorten <= 0mm ] 
(NDPCloudNode.north east) to (VERNode.north west); 

\draw [ % strong 
 ->, line width=1mm, draw=orange, opacity=.8, bend right=20,looseness=1,
shorten >= 2mm, shorten <= 0mm ] 
(QtrNode.east) to (VERNode.south west); 

\draw[thick, shorten <= -.15cm, shorten >= -.15cm, snake it ] 
(txtNode.north)
to (rpTxtNode);
% node {}%
%	\raisebox{3em}{%	
%		\colorbox{white}{%	
%			\begin{minipage}[t]{1.5em}	 
%			\centering{\begin{scriptsize}
%				{\begin{singlespace}	
%					\color{blGreen}{}
%					\end{singlespace}}	
%				\end{scriptsize}}
%			\end{minipage}}}}
%(rpTxtNode);

 
%\draw[->, thick, >= angle 90, shorten >= .25cm ]
%  (hiddenArcRight.east)
%   edge [bend left=100,looseness=1]  (hiddenCenter);

%\draw[ <->, thick, >= latex', shorten >= .25cm ]
% (hiddenArcLeft.north) edge [bend left=-100hiddenArc, bend right=100, looseness=1] (hiddenArcRight.north);

\end{tikzpicture}}}
}\end{minipage}}}	
\caption{\colorbox[rgb]{0.98,0.94,0.94}{{\color[rgb]{0.31,0.21,0.1}{{\textbf{Outline of Mappings to and from RDF}}}}}}\label{fig:CQLStack}
\label{fig:relae}
\end{figure}


\end{minipage}}

%\end{minipage}{\textwidth}}
\end{quadblock}
}

\end{frame}




\begin{frame}{\ft{Using Custom Viewers}}

\doubleFrame{Re-PDF data is delivered to potential 
buyers embedded in PDF files so that buyers can open 
these files even if they do not have the Re-PDF software.  
As a result, users will not be burdened with files that 
they cannot open or identify.  At the same time, 
users that \textit{do} install the Re-PDF viewer will 
experience the Re-PDF presentations interactively --- adding 
many layers to the ordinary PDF experience.}

\begin{tikzpicture}
%\nodeincludegraphics[0.88\textwidth]{screenshots/ss-norpdf.png}
\nodeincludegraphicsTR{2.5cm}{2cm}{screenshots/ss-norpdf.png}

\ann{darkRed}{0.7}{1mm}{grammarArrowColor}{0.5}{6,3.5}{6.5}{1}{0.85}
%\node [anchor=west] (note) at (-1,3) {\Large Note};
%\draw [-latex, ultra thick, red] (note) to[out=0, in=-120] (0.48,0.80);

 \node [anchor=west,fill=blue!20!yellow,inner sep=13, text width=14cm]
  (longnote) at (1.5,1.5) {%  %{\color{rb!85!red}{
  {\cframedboxyellow{\large \textbf{Re-PDF documents include links 
  to web pages where users can download each 
  realtor's customized version of the Re-PDF viewer 
  --- for users who are reading the 
  document with other PDF applications.}}}};


\end{tikzpicture}


\end{frame}




\begin{frame}{\ft{Launching Dialog Windows}}

\doubleFrame{By clicking on links in Re-PDF documents, 
users can launch dialog windows with 
special layout and features.  For 
regular PDF viewers, these links will take users to 
realtors\curlyapos{} web pages.  However, for custom Re-PDF viewers, 
these links will open adjunct viewers --- fine-tuned for 
each realtor --- such as customized dialog boxes 
and 3D or multimedia presentations.}

\begin{tikzpicture}
%\nodeincludegraphics[0.87\textwidth]{screenshots/ss-links.png}
\nodeincludegraphicsTR{2.5cm}{2cm}{screenshots/ss-links.png}

\node [anchor=west,fill=orange!8!white,inner sep=12, 
opacity=0.88, text opacity=1] (note) at (9.5,8.5) {\hc{Links}};

%\rectann{darkRed}{0.7}{1mm}{grammarArrowColor}{0.5}{2,8}{5}{-7}{0.5}
\rectanneat{darkRed}{0.7}{1mm}{grammarArrowColor}{0.5}{0,10}{9}{-7}{0.7}

%\draw [-latex, ultra thick, red] (note) to (7, 7);

\colorarr{>=latex, ->}{fcBoxColor!60!black}{0.8}{blGreen!30!red}{1}{1mm}{note}{7, 8}


 \node [anchor=west,fill=brown!20!white,inner sep=7, text width=10cm]
  (longnote) at (8.5,7) {%  %{\color{rb!85!red}{
  {\cframedbox{\large \textbf{Photo View Dialog (slides 9-12)}}}};

\colorarr{>=latex, ->}{fcBoxColor!20!black}
{0.8}{darkRed!70!blue}{2}{1mm}{longnote.west}{6.3, 7.5}


 \node [anchor=west,fill=brown!20!white,inner sep=7, text width=10cm]
  (longnote) at (8.5,5) {%  %{\color{rb!85!red}{
  {\cframedbox{\large \textbf{Virtual Tour Dialog (slides 16-17)}}}};

\colorarr{>=latex, ->}{fcBoxColor!20!black}
{0.8}{darkRed!70!blue}{2}{1mm}{longnote.west}{6.1, 6.8}



 \node [anchor=west,fill=brown!20!white,inner sep=7, text width=10cm]
  (longnote) at (8.5,3) {%  %{\color{rb!85!red}{
  {\cframedbox{\large \textbf{Room Dialog (slides 13-15)}}}};

\colorarr{>=latex, ->}{fcBoxColor!20!black}
{0.8}{darkRed!70!blue}{2}{1mm}{longnote.west}{5.9, 4.7}


%\draw [-latex, draw=darkRed,draw opacity=0.8,line width=#4] #5 to #6;


%[out=120, in=0]
\end{tikzpicture}


\end{frame}




\begin{frame}{\ft{Using Context Menus}}

\doubleFrame
}

\begin{tikzpicture}
%\nodeincludegraphics[0.87\textwidth]{screenshots/ss-contextmenu.png}

\nodeincludegraphicsTR{5.5cm}{2cm}{screenshots/ss-contextmenu.png}

\rectann{red}{0.3}{3mm}{blue!50!cyan}{0.5}{8.5,9}{7.5}{-6.5}{0.85}

\end{tikzpicture}


\end{frame}





\begin{frame}{\ft{The Photo Viewer}}

\doubleFrame{Re-PDF provides a suite of useful dialog boxes 
and a toolkit for programming customized presentations according 
to the preferences of each realtor.  For example, one 
standard format would be a color-coded property-photo viewer with 
thumbnail-sized previews of each image.  The whole set of 
thumbnail photos can then be seen together 
alongside the larger photo being viewed, rather than 
users having to scroll between groups of four 
or five thumbnails at a time.  
Users can enlarge photos by clicking on 
their thumbnails, or view photos in 
sequence via left/right arrows.}

\vspace{-1em}
\begin{tikzpicture}
%\nodeincludegraphics[0.89\textwidth]{screenshots/ss-ph1.png}
\nodeincludegraphicsTRRS{1}{3.3cm}{3cm}{2cm}{1cm}{screenshots/ss-ph1.png}

\ann{BlueGreen}{0.3}{1mm}{grammarArrowColor}{0.5}{13.5,8}{5}{4.5}{0.95}

%\node [anchor=west] (note) at (6.25,9) {\Huge {\color{blGreen!30!black}Thumbnails}};
\node [anchor=west,fill=brown!8!white,inner sep=12, 
opacity=0.88, text opacity=1] (note) at (2.95,9.1) {\Huge {\hc{Thumbnails}}};

\colorarr{>=latex, ->}{fcBoxColor!60!black}{0.8}%
{blGreen!30!red}{1}{1mm}{[xshift=2em,yshift=.4em]note.south}{8.5, 8}

%\node [anchor=west] (noteArr) at (20.15,3) {\Huge {\color{blGreen!30!black}Arrows}};

\node [anchor=west,fill=yellow!15!white,inner sep=9, 
opacity=0.5, text opacity=1] (noteArr) at (18.7,2.6) {\hc{Arrows}};


\colorarr{>=latex, ->}{fcBoxColor!60!black}{0.8}%
{blGreen!30!red}{1}{1mm}{noteArr.south}{20.15,1}

\end{tikzpicture}


\end{frame}




\begin{frame}{\ft{Photo Viewer Color Coding}}

\doubleFrame{The Photo Viewer would access embedded data structures 
which specify how photos are grouped into categories, such as 
different kinds of rooms (entrance/foyer/hall, 
kitchen/dining room, bedroom, etc.).  Colors 
are assigned to each group, with 
%When rendering the thumbnail images, 
color thumbnail used as visual cues marking each
photo's classification.  Realtors can  
modify the group names and their corresponding colors,  
according to their own preferences.}

\begin{tikzpicture}
%\nodeincludegraphics[0.9\textwidth]{screenshots/ss-ph2.png}
\nodeincludegraphicsTRRS{1}{4.6cm}{2.2cm}{0.8cm}{1cm}{screenshots/ss-ph2.png}

\rectann{darkRed}{0.7}{1mm}{grammarArrowColor}{0.5}{0.05,0.5}{9.4}{5.25}{0.9}

\node [anchor=west,fill=brown!8!white,inner sep=12, 
opacity=0.88, text opacity=1] (note) at (11,2.89) {\hc{Current Photo}};

\colorarr{>=latex, ->}{fcBoxColor!60!black}
{0.8}{blGreen!30!red}{1}{1mm}{note.west}{6.58, 7.25}


\end{tikzpicture}


\end{frame}




\begin{frame}{\ft{Photo Viewer Interactive Cues}}

\doubleFrame{Color insertions switch from horizontal 
to vertical indicating which photos have been viewed 
(enlarged).  Separate and apart from that, the 
thumbnail of the current enlarged photo 
is marked with a thick colored border.  This border will 
surround the thumbnail and the overlay both for photos being 
viewed for the first time (with a horizontal color-band) 
and for photos being viewed a second time (with a vertical 
color-band).
}

\begin{tikzpicture}
%\nodeincludegraphics[0.9\textwidth]{screenshots/ss-ph3.png}
\nodeincludegraphicsTR{5.5cm}{3.23cm}{screenshots/ss-ph3.png}

 %%\node [anchor=west] (note) at (11,11.2) {\Huge {\color{blGreen!30!black}Already Viewed}};

\node [anchor=west,fill=brown!8!white,inner sep=12, 
opacity=0.88, text opacity=1] (note) at (7,11.8) {\hc{Already Viewed (vertical color band)}};


%\ann{BlueGreen!30!red}{1}{.8mm}{blue!50!orange}{0.5}{3.61,8.95}{.4}{.6}{0.7}

\colorarr{>=latex, ->}{fcBoxColor!20!black}
{0.8}{darkRed!70!blue}{2}{1mm}{note.west}{3.7, 9}


\node [anchor=west,fill=brown!8!white,inner sep=12, 
opacity=0.88, text opacity=1] (note) at (6.3,9.7) {\hc{Not Yet Viewed (horizontal color band)}};

%\ann{BlueGreen!30!red}{1}{.8mm}{blue!50!orange}{0.5}{7.6, 7.3}{.73}{.35}{0.7}

%
\colorarr{>=latex, ->}{fcBoxColor!20!black}
{0.8}{darkRed!70!blue}{2}{1mm}{note.south}{8.2, 7.25}

%\node [anchor=west] (note) at (11,3) {\Huge {\color{blGreen!30!black}Current Photo}};

\node [anchor=west,fill=brown!8!white,inner sep=12, 
opacity=0.88, text opacity=1] (note) at (5,1.89) {\hc{Current Photo (viewed for the second time)}};

\colorarr{>=latex, ->}{fcBoxColor!60!black}
{0.8}{blGreen!30!red}{1}{1mm}{note.north west}{4.39, 4.75}


\end{tikzpicture}


\end{frame}





\begin{frame}{\ft{Photo Viewer Alternative Interface}}

\doubleFrame{For realtors who prefer showing/hiding 
thumbnails (as opposed to using color-coding to 
differentiate groups), the 
Photo Viewer can also be implemented with this alternative design, 
or the two designs can be combined.  
In general, realtors will be afforded a lot of control over the 
layout and operational design of Re-PDF interactive windows.}

\begin{tikzpicture}

\nodeincludegraphicsTRRS{1}{4cm}{0.05cm}{3cm}{1.35cm}{screenshots/ss-pha4.png}


\node [anchor=west,fill=brown!8!white,inner sep=12, 
opacity=0.88, text opacity=1] (note) at (4.5,8.2) {\hc{Visible Groups}};


\ann{BlueGreen}{1}{.8mm}{grammarArrowColor}{0.5}{3.61,5.5}{2.4}{.6}{0.7}
%\ann{BlueGreen!30!red}{1}{.8mm}{blue!50!orange}{0.5}{3.61,5.5}{2.4}{.6}{0.7}

%{BlueGreen}{0.3}{1mm}{grammarArrowColor}{0.5}

\colorarr{>=latex, ->}{fcBoxColor!60!black}
{0.8}{blGreen!30!red}{2}{1mm}{note.south}{6, 5.9}




\node [anchor=west,fill=brown!8!white,inner sep=12, 
opacity=0.88, text opacity=1] (note) at (11,3.8) {\hc{Hidden Groups}};


\ann{BlueGreen!30!red}{1}{.8mm}{blue!50!orange}{0.5}{3.61,1.7}{2.4}{.6}{0.7}

\colorarr{>=latex, ->}{fcBoxColor!20!black}
{0.8}{darkRed!70!blue}{2}{1mm}{note.west}{6, 1.9}


%\node [anchor=west,fill=brown!8!white,inner sep=12, 
%opacity=0.88, text opacity=1] (note) at (6.7,9.7) {\hc{Not Yet Viewed (horizontal color band)}};

%\ann{BlueGreen!30!red}{1}{.8mm}{blue!50!orange}{0.5}{7.6, 7.3}{.73}{.35}{0.7}

%\colorarr{>=latex, ->}{fcBoxColor!20!black}
%{0.8}{darkRed!70!blue}{2}{1mm}{note.south}{8.2, 7.25}



%\node [anchor=west,fill=brown!8!white,inner sep=12, 
%opacity=0.88, text opacity=1] (note) at (5,1.89) {\hc{Current Photo (viewed for the second time)}};

%\colorarr{>=latex, ->}{fcBoxColor!60!black}
%{0.8}{blGreen!30!red}{1}{1mm}{note.north west}{4.35, 4.75}


\end{tikzpicture}


\end{frame}





\begin{frame}{\ft{Room Highlights Dialog Box: Room-By-Room Breakdown}}

\doubleFrame{The Re-PDF viewer provides a special 
dialog box to show the room-by-room 
breakdown of architectural/decorative features/amenities.}

\begin{tikzpicture}

%\nodeincludegraphics[0.85\textwidth]{screenshots/ss-rm1.png}
\nodeincludegraphicsTR{5.5cm}{2cm}{screenshots/ss-rm1.png}

\ann{darkRed}{0.7}{1mm}{grred!30!brown}{0.5}{13,9}{10}{3}{0.8}

\node [anchor=west,fill=white,inner sep=8]
 (note) at (13,9) {\hc{Feature Descriptions}};
 
% \node [anchor=west,fill=yellow!5!white,opacity=.95,
% text opacity=1,inner sep=9, text width=6.8]
%  (longnote) at (5,4.8) {{\color{black}{Each room features, one of which is highlighted at a %time}
%  }
%  };


 \node [anchor=west,fill=yellow!5!white,opacity=.95,
 text opacity=1,inner sep=9, text width=6.8cm]
  (longnote) at (5,4.8) {{\color{brred!30!black}{%
  {\hspace*{0.8em}\parbox{6.3cm}{
  {\LARGE {\begin{centering}\begin{spacing}{1.1}\textbf{Each room description is divided into several features, 
  one of which is highlighted at a time}\end{spacing}\end{centering}}}}}}}};
  

\end{tikzpicture}


\end{frame}




\begin{frame}{\ft{Room-By-Room Tab Organization}}

\doubleFrame{The Room-by-Room breakdown features a dual tabbed 
area: on the top appear the room tabs, and to the left appear 
the feature tabs.  The combination of 
currently selected room and feature tabs 
control the highlighting of details in room descriptions.  
Realtors will have the flexibility to choose which features 
to describe and which rooms to include in the breakdown.}

\begin{tikzpicture}
\nodeincludegraphicsTR{5.5cm}{3cm}{screenshots/ss-rm2.png}

\node [anchor=west,fill=white,inner sep=8]
 (note) at (10,8) {\hc{Dual Set of Tabs}};

\ann{darkRed}{0.7}{1mm}{grammarArrowColor}{0.5}{12,10.5}{6.5}{1}{0.85}

\colorarr{>=latex, ->}{fcBoxColor!20!black}
{0.8}{blGreen!30!red}{2}{1mm}{note.north}{15, 10}

\colorarr{>=latex, ->}{fcBoxColor!20!black}
{0.8}{blGreen!30!red}{1}{1mm}{note.west}{5.3, 6.2}

 \node [anchor=west,fill=brown!20!white,inner sep=21, text width=8.9cm]
  (longnote) at (8,3.5) {%  %{\color{rb!85!red}{
  {\cframedbox{\Large \textbf{The tab to the left determines which \makebox{feature} is  
highlighted, while tab on top \makebox{determines} which room is shown.}}}};

\ann{BlueGreen}{0.3}{1mm}{grammarArrowColor}{0.5}{3,6}{3}{4.5}{0.95}

\end{tikzpicture}


\end{frame}




\begin{frame}{\ft{Room-By-Room Navigation}}

\doubleFrame{The dual-tab functionality allows users to conveniently browse 
through different rooms in sequence, particularly 
when they are interested in honing in on 
one feature in particular.   By selecting a feature 
from the left margin, those become highlighted in 
each room description so users see it immediately.  
In this screenshot, the user selected window treatment 
as a feature to be highlighted in 
each room description.}

\begin{tikzpicture}
\nodeincludegraphicsTR{2.5cm}{2.5cm}{screenshots/ss-rm3.png}

\rectanneatdbl{darkRed}{0.7}{1mm}{grammarArrowColor}
{0.5}{3,12.5}{16}{-1.5}{2,-0.2}

\ann{red}{0.7}{1mm}{blue!50!cyan}{0.5}{12,3.65}{6.5}{1}{0.85}


\node [anchor=west,fill=white,inner sep=8]
 (note) at (10,10) {\hc{Browsing room-by-room ...}};

\colorarr{>=latex, ->}{fcBoxColor!20!black}
{0.8}{blGreen!30!red}{2}{1mm}{note.north west}{7.6, 11.4}

\node [anchor=west,fill=white,inner sep=8]
 (note) at (10,5.5) {\hc{and feature-by-feature ...}};

\colorarr{>=latex, ->}{fcBoxColor!60!black}
{0.8}{darkRed!70!blue}{1}{1mm}{note.west}{5.3, 4.1}


\end{tikzpicture}


\end{frame}





\begin{frame}{\ft{Using Virtual Tours}}

\doubleFrame{Another kind of dialog box is an 
embedded WebGL viewer showing  
a property\curlyapos{}s Materport tour in a detached window}

\begin{tikzpicture}
\nodeincludegraphicsTR{2.1cm}{2cm}{screenshots/ss-vt1.png}

\node [anchor=west,fill=brown!8!white,inner sep=4, 
opacity=0.88, text opacity=1]
 (note) at (10,8) {\hc{Matterport \curlyquote{Line of Sight}}};

\colorarr{>=latex, ->}{fcBoxColor!20!black}
{0.8}{blGreen!30!red}{2}{1mm}{note.south}{14.5, 5.9}


\end{tikzpicture}


\end{frame}




\begin{frame}{\ft{Virtual Tour Windows}}

\doubleFrame{Compared to browser-based Virtual Tours,
the embedded WebGL viewers in Re-PDF are more convenient, because 
they allow the tour window to be visually juxtaposed with 
other windows, such as the room-by-room breakdown of 
descriptions and features.}

\begin{tikzpicture}
\nodeincludegraphicsTR{2.1cm}{2cm}{screenshots/ss-vt2.png}

 \node [anchor=west,fill=yellow!10!red!2!white,opacity=0.98,
 text opacity=1,inner sep=16, text width=6cm]
  (longnote) at (14,11.15) {{\color{brred!50!black}{%
  {\hspace*{0.8em}\parbox{5.7cm}{
  {\LARGE {\begin{centering}\begin{spacing}{1.1}\textbf{Room description features
  can be linked to Matterport views of each room}\end{spacing}\end{centering}}}}}}}};

\colorarr{>=latex, ->}{fcBoxColor!20!black}
{0.8}{darkRed!70!blue}{2}{1mm}{longnote.west}{12.5, 10.5}

\colorarr{>=latex, ->}{fcBoxColor!60!black}
{0.8}{blGreen!30!red}{1}{1mm}{longnote.south}{18.5, 7.5}

\end{tikzpicture}


\end{frame}




\begin{frame}{\ft{Virtual Tour Tag Posts}}

\doubleFrame{With Materport tours that employ tag-posts, 
the Virtual Tour can also be synchronized 
with other parts of the application --- for 
example, showing information about a particular room when 
users click on a tag-post hyperlink positioned 
at the center of the room as depicted in a virtual tour.
Matterport spaces often use tag-posts to 
describe property features/amenities found at 
specific points in the virtual tour. 
}

\begin{tikzpicture}

\nodeincludegraphicsTRRS{1}{0cm}{1cm}{2.8cm}{3cm}{screenshots/ss-vt3.png}

%\ann{BlueGreen!30!red}{0.8}{0.7mm}{red!30!blue}{0.5}{11,4.5}{2}{2.5}{0.95}

%\polyann{BlueGreen!30!red}{0.8}{0.7mm}{red!30!blue}{0.5}{11,4.5}{6}{3cm}{3.5cm}

%\polyann{black}{1}{2mm}{red!30!blue}{0.5}{11,4.5}{6}{3cm}{3.5cm}

%\path [draw=#1,draw opacity=#2,line width=#3, fill=#4, fill opacity = #5, even odd rule] %
%	node [minimum size=#8,regular polygon,regular polygon sides=#7] at (#6) {}
%	node [minimum size=#9,regular polygon,regular polygon sides=#7] at (#6) {};}
%\path [draw=black,line width=3mm,fill=red,even odd rule] %
%	

\node [minimum size=4.2cm,regular polygon,regular polygon sides=6,
rotate=30,draw=BlueGreen!30!red,line width=1mm] (innr) at (8.3,5.58) {};

\node [minimum size=4.8cm,regular polygon,regular polygon sides=6,
rotate=30,draw=BlueGreen!30!red,line width=1mm] (outr) at (8.3,5.58) {};

\path [fill=red!30!blue,fill opacity=0.5,even odd rule]
(outr.corner 1) -- (outr.corner 2) -- (outr.corner 3) -- 
(outr.corner 4) -- (outr.corner 5) -- (outr.corner 6)
(innr.corner 1) -- (innr.corner 2) -- (innr.corner 3) -- 
(innr.corner 4) -- (innr.corner 5) -- (innr.corner 6)
;

\node [anchor=west,fill=white,inner sep=8]
 (note) at (-0.1,6) {\hc{Tag Posts}};

\node [anchor=west,fill=white,inner sep=4, text width=5.2cm]
 (longnote) at (-0.1,2.4) {{\color{rb!80!red}{%
 {\LARGE {\begin{spacing}{0.9}\textit{Matterport Tag Posts add details to fixed 
 locations in a virtual tour.  Tag Posts can include 
 text, audio, video, or hyperlinks}\end{spacing}}}}}};


\colorarr{>=latex, ->}{fcBoxColor!70!blue}{1}
{fcBoxColor!30!red}{1}{2mm}{note.east}{7.4,4.75}



%\filldraw[fill=red]
%node [minimum size=2cm,regular polygon,regular polygon sides=6,draw=black,thick]
% (innr) at (10,4.5) {}
%node [minimum size=3cm,regular polygon,regular polygon sides=6,draw=black,thick] at (10,4.5)  %(outr) {};


%\fill[even odd rule] innr outr {};
%
%\path [draw=black,line width=3mm,fill=red,even odd rule]

%
%\path[fill=red,even odd rule]
%\draw[fill, fill=red,even odd rule]

%\node [minimum size=3cm,regular polygon,
%regular polygon sides=6,draw=black,thick] (outr) at (15,4.5) {};
%\node [minimum size=2cm,regular polygon,
%regular polygon sides=6,draw=black,thick] (innr) at (15,4.5) {};

%%,save path=innr

%\path[fill=red,even odd rule]
%restore spath=innr;



%\path[fill=red,even odd rule]
%regular polygon [minimum size=3cm,regular polygon sides=6] at (15,4.5) {} 
%regular polygon [minimum size=2cm,regular polygon sides=6] at (15,4.5) {};


%\ann{BlueGreen!30!red}{0.8}{0.7mm}{red!30!blue}{0.5}{11,4.5}{2}{2.5}{0.95}

\end{tikzpicture}


\end{frame}





\begin{frame}{\ft{Interactive VR: Hyperlinked Tag Posts}}

\doubleFrame{Materport tag posts can embed 
hyperlinks, allowing realtors (and 
other businesses, such as museums, car dealerships, 
restaurants, stores, and theaters) 
to embed links in Matterport spaces where users 
can request details about features or inventory 
at the tagged location.
For embedded WebGL, these links become 
channels of communication between VR engines 
and the Re-PDF application.}

\begin{tikzpicture}
\nodeincludegraphicsTR{2.7cm}{2.4cm}{screenshots/ss-vt5.png}

\ann{darkRed}{0.7}{1mm}{grammarArrowColor}{0.5}{13,7}{5.5}{2}{0.9}

\node [anchor=west,fill=white,inner sep=11]
%(note) at (13,10) {\Huge {\color{blGreen!30!black}Hyperlinked Tag Posts}};
 (note) at (12,10) {\hc{Hyperlinked Tag Posts}};

\colorarr{>=latex, ->}{blGreen!80!blue}{0.7}
{cyan!60!red}{1}{2mm}{note.south}{15,6.5}


\end{tikzpicture}


\end{frame}




\begin{frame}{\ft{Intercepting Hyperlinks}}

\doubleFrame{With Re-PDF, Matterport hyperlinks can 
	then be \curlyquote{intercepted} --- that is, rather 
	than directing users to a web page, 
	the Re-PDF viewer can load specialized windows 
in accordance with 
	the user's current location in the virtual tour.}

\begin{tikzpicture}
\nodeincludegraphicsTR{2.7cm}{2cm}{screenshots/ss-vt4.png}

 \node [anchor=west,fill=brown!20!white,inner sep=7, text width=14cm]
  (longnote) at (5.5,7) {%  %{\color{rb!85!red}{
  {\cframedbox{\large \textbf{By intercepting hyperlink click actions, 
  Re-PDF ascertains where the user is positioned at that point in 
  time on the virtual tour; and by doing so, 
  it can therefore respond 
  to the user's clicks with customized content.  For \makebox{example}, the 
  application could show a 3D model 
  of the object for which the user is requesting information, as 
  seen on the next slide.}}}};

\end{tikzpicture}


\end{frame}




\begin{frame}{\ft{Hyperlinks and External Applications}}

\doubleFrame{As an example of custom handlers 
	for Matterport hyperlinks, Re-PDF can export 3D 
	models assocaited with some links to MeshLab.}

\vspace{2em}
\begin{tikzpicture}

\nodeincludegraphicsLOCTR{0,5}{0.7}{0cm}{4cm}{screenshots/ss-vt6.png}
\nodeincludegraphicsLOC{14,5.5}{0.5}{screenshots/ss-vt7.png}

 \node [anchor=west,fill=brown!20!white,inner sep=11, text width=14cm]
  (longnote) at (8.5,15.6) {%  %{\color{rb!85!red}{
  {\cframedbox{\large \textbf{The relevant Re-PDF 
  applications can be designed to follow up 
  on clicks about cars for sale or museum displays (as 
  seen on the last two slides) by launching 3D viewers 
  for the objects involved.}}}};

\end{tikzpicture}


\end{frame}





\begin{frame}{\ft{Using MeshLab and Panini with the Re-PDF Viewer}}

\doubleFrame{Realtors can also opt to export their 
Matterport \curlyquote{dollhouse} to MeshLab (a 3D graphics engine) 
and their panorama-photos to Panini (a viewer for many 
panorama geometries).  Both of these applications can 
be customized and bundled 
with Re-PDF software.}

\begin{tikzpicture}
\nodeincludegraphicsTRRS{1.2}{13.16cm}{1cm}{4cm}{4cm}{screenshots/snapshot07.png}
\end{tikzpicture}

\end{frame}




\begin{frame}{\ft{Deleting Saved Files}}

\doubleFrame{After users study a property, they have the option of 
removing the local files that were extracted from the PDF 
document when it was first opened in the Re-PDF application.}

\begin{tikzpicture}
\nodeincludegraphicsTRRS{1}{3.83cm}{0cm}{3.7cm}{0cm}{screenshots/ss-cl2.png}

\ann{BlueGreen}{0.3}{1mm}{grammarArrowColor}{0.5}{12,8}{8}{3}{0.85}

\end{tikzpicture}

\end{frame}





\begin{frame}{\Huge{\textbf{Thanks!}}}
{\LARGE
\vspace{1em}
 \colorbox{white}{\setlength{\fboxsep}{1.5em}
   \framebox{\begin{minipage}{.7\textwidth}\centering
	  {\color{blGreen!40!blbl}\textbf{Please contact 
	  Linguistic Technology Systems, Inc., for 
	  more information  
	  about Re-PDF, \VersatileUX{}, APDL, or 
	  other Software Language Engineering solutions.}}
	 \end{minipage}}}}
	 
\vspace{-1em}

\begin{adjustwidth*}{-2em}{} 
\begin{tikzpicture}

\nodeincludegraphicsLOCTRV{4.8,6.6}{0.4}{1cm}{4cm}{3.5cm}{1cm}{slide21pic.png}
\nodeincludegraphicsLOCTRV{12.4,0.8}{0.4}{4cm}{2cm}{1cm}{2cm}{voronoi.png}
\nodeincludegraphicsLOCTRV{14,8}{0.3}{0cm}{4cm}{0cm}{0cm}{ecs3.png}
\nodeincludegraphicsLOCTRV{0,5.5}{0.22}{4.5cm}{1cm}{0.5cm}{0.5cm}{panda.png}
\nodeincludegraphicsLOCTRV{6.2,3.7}{0.24}{0cm}{2cm}{2cm}{0cm}{flowers.jpg}
\nodeincludegraphicsLOCTRV{0.6,0.7}{0.2}{0cm}{4cm}{0cm}{0cm}{iqMOL-5.png}

\end{tikzpicture}
\end{adjustwidth*}	 
	 
\end{frame}



\end{document}
