\documentclass{article}


\usepackage{xparse}

\usepackage{letltxmacro}
\LetLtxMacro{\hyphenHokHhere}{\-}

\edef\oldCCofEXC{\the\catcode`!}
\edef\oldCCofDASH{\the\catcode`-}
\edef\oldCCofCOL{\the\catcode`:}

\catcode`!=11
\catcode`-=11

\edef\newCCofEXC{\the\catcode`!}
\edef\newCCofDASH{\the\catcode`-}
\edef\newCCofCOL{11}

\newcommand{\oldCCforEXC}{\catcode`!=\oldCCofEXC}
\newcommand{\newCCforEXC}{\catcode`!=\newCCofEXC}

\newcommand{\oldCCforDASH}{\catcode`-=\oldCCofDASH}
\newcommand{\newCCforDASH}{\catcode`-=\newCCofDASH}

\newcommand{\oldCCforCOL}{\catcode`:=\oldCCofCOL}
\newcommand{\newCCforCOL}{\catcode`:=\newCCofCOL}

\AtBeginDocument{\newCCforEXC\newCCforDASH}

\oldCCforEXC\oldCCforDASH

%\oldCCforCOL



%% comment this to set a different p gap ... 
\newcommand{\p}[1]{

\vspace{1em}#1}
%%

%\newcounter{pcurrentjobname}



\usepackage{datetime}
\usepackage{etoolbox}

\usepackage{zref-user}
\usepackage{xifthen}

\newCCforEXC\newCCforDASH

%\newCCforCOL
%\newCCforCOL{}

%\catcode`:=11

\newlength\orig-basels % original baselineskip
\AtBeginDocument{%
\setlength\orig-basels{\baselineskip}%
} 

\newwrite\sdi-file
\immediate\openout\sdi-file=\jobname.sdi.ntxh

\newif\iftabng
\tabngfalse

\newif\ifactive-sdi-mark
\active-sdi-markfalse

\newif\ifsdi-no-ss-flag
\sdi-no-ss-flagfalse

\usepackage{letltxmacro}
\LetLtxMacro{\oldmmsemi}{\;}
\LetLtxMacro{\oldtbplus}{\+}
\LetLtxMacro{\oldtbgt}{\>}
\LetLtxMacro{\oldtblt}{\<}
\LetLtxMacro{\oldmmgt}{\+}
\LetLtxMacro{\oldmmexc}{\!}
\LetLtxMacro{\oldmmcomma}{\,}

%\oldCCforCOL{}
\catcode`:=12
\LetLtxMacro{\oldmmcolon}{\:}
%\catcode`:=11

\newcommand{\?}[1][]{\smk{#1}}
\renewcommand{\,}[1][]{\ifactive-sdi-mark%
\relax\ifmmode\oldmmcomma\else\smd{#1}\fi\else\oldmmcomma\fi}

%\renewcommand{\!}{\relax\ifmmode\oldmmexc\else\sdi-no-ss\fi}

\newcommand{\+}{\iftabng\oldtbplus\else\sss\fi}
\newcommand{\<}{\iftabng\oldtblt\else\spe\fi}

\renewcommand{\>}{\iftabng\oldtbplus\else\relax%
\ifmmode\oldmmgt\else\sse\sss\ignorespaces\fi\fi}

%\renewcommand{\>}{\sse\sss}

\renewcommand{\;}{\relax\ifmmode\oldmmsemi\else\sse\fi}
\renewcommand{\:}{\relax\ifmmode\oldmmcolon\else\sps\fi}
\renewcommand{\!}{\relax\ifmmode\oldmmexc\else\sdi-no-ss\fi}


\newcommand{\write-sdi}[1]{\immediate\write\sdi-file#1}

\newcommand{\emblink}[2]{\href{\#sdi:#1--#2}{\#sdi:#1--#2}}

\newcounter{sdi-counter}
\newcounter{sdi-counter-O}[page]

\newcounter{sdi-mark}
\newcounter{sdi-mark-O}[page]

\newcounter{sdi-counter-P}
\newcounter{sdi-counter-PO}[page]

\newcounter{sdi-counter-FJ}
\newcounter{sdi-counter-F}[sdi-counter-FJ]
\newcounter{sdi-counter-FS}[sdi-counter-FJ]

\makeatletter
\def\currentjobname{\jobname}% 
\newcommand{\sdiinput}[1]{%\def\currentjobname{#1}
\stepcounter{sdi-counter-FJ}
\def\currentjobname{#1}
%\setcounter{sdi-counter-F}{0}
\input{#1}}
\makeatother


\newcounter{sdi-counter-C}[sdi-counter-P]

\newcounter{sdi-element}
\newcounter{sdi-element-O}[page]
\newcounter{sdi-element-C}[sdi-counter-P]

\def\topt#1{\expandafter\the\dimexpr\dimexpr#1sp\relax\relax}


\newcommand{\write-sdi-start}[1]{\write\sdi-file{!/ SDI_#1_Start}%
\write\sdi-file{$n: #1:Start}}
\newcommand{\write-sdi-end}[1]{\write\sdi-file{!/ SDI_#1_End}%
\write\sdi-file{$n: #1:End}}
\newcommand{\write-sdi-val-start}[1]{\write\sdi-file{!/ SDI_#1}%
\write\sdi-file{$n: #1}}
\newcommand{\write-sdi-val-end}{\write\sdi-file{/!^^J<+>^^J}}

\newcommand{\field-mark}[1]{\expandafter$\expandafter#1%
\expandafter:\expandafter\space}

\makeatletter
\newcommand{\field-mark@i}{\field-mark{i}}
\newcommand{\field-mark@o}{\field-mark{o}}
\newcommand{\field-mark@s}{\field-mark{s}}
\newcommand{\field-mark@p}{\field-mark{p}}
\newcommand{\field-mark@e}{\field-mark{e}}
\newcommand{\field-mark@r}{\field-mark{r}}
\newcommand{\field-mark@c}{\field-mark{c}}
\newcommand{\field-mark@b}{\field-mark{b}}
\newcommand{\field-mark@k}{\field-mark{k}}
\newcommand{\field-mark@f}{\field-mark{f}}
\newcommand{\field-mark@x}{\field-mark{x}}
\newcommand{\field-mark@y}{\field-mark{y}}
\newcommand{\field-mark@j}{\field-mark{j}}


\newcommand{\sdi@check-write-bls}{%
\ifthenelse{\lengthtest{\orig-basels=\baselineskip}}{}%
{\write\sdi-file\expandafter{\expandafter\field-mark@b\the\baselineskip}}}


\newcommand{\nhv!sdi-element-start}[1]{%
\stepcounter{sdi-element}%
\stepcounter{sdi-element-O}%
\stepcounter{sdi-element-C}%
\pdfsavepos\write-sdi-start{Element}%
\write\sdi-file\expandafter{\expandafter\field-mark@i%
\the\c@sdi-element}%
\write\sdi-file\expandafter{\expandafter\field-mark@o%
\the\c@sdi-element-O}%
\write\sdi-file\expandafter{\expandafter\field-mark@c%
\the\c@sdi-element-C}%
\write\sdi-file\expandafter{\expandafter\field-mark@r%
\the\c@sdi-counter-P}%
\write\sdi-file\expandafter{\expandafter\field-mark@k#1}%
\sdi@check-write-bls%
\write\sdi-file\expandafter{\expandafter\field-mark@p%
\thepage^^J%
$x: \topt\pdflastxpos^^J%
$y: \topt\pdflastypos}%
\write-sdi-val-end}

\newcommand{\nhv!sdi-element-end}[1]{%
\pdfsavepos\write-sdi-end{Element}%
\write\sdi-file\expandafter{\expandafter\field-mark@i%
\the\c@sdi-element}%
\write\sdi-file\expandafter{\expandafter\field-mark@o%
\the\c@sdi-element-O}%
\write\sdi-file\expandafter{\expandafter\field-mark@p%
\thepage^^J%
$x: \topt\pdflastxpos^^J%
$y: \topt\pdflastypos}%
\write-sdi-val-end}

\newcommand{\smk}[1]{%
\active-sdi-marktrue%
\stepcounter{sdi-mark}%
\stepcounter{sdi-mark-O}%
\pdfsavepos\write-sdi-val-start{Mark}%
\write\sdi-file\expandafter{\expandafter\field-mark@i%
\the\c@sdi-mark}%
\if#1\empty\else\write\sdi-file\expandafter{\expandafter\field-mark@s#1}\fi%
\write\sdi-file\expandafter{\expandafter\field-mark@o%
\the\c@sdi-mark-O}%
\write\sdi-file\expandafter{\expandafter\field-mark@r%
\the\c@sdi-counter-P}%
\sdi@check-write-bls%
\write\sdi-file\expandafter{\expandafter\field-mark@p%
\thepage^^J%
$x: \topt\pdflastxpos^^J%
$y: \topt\pdflastypos}%
\write-sdi-val-end}

\newcommand{\smd}[1]{%
\active-sdi-markfalse%
\pdfsavepos\write-sdi-val-start{Mark_Secondary}%
\write\sdi-file\expandafter{\expandafter\field-mark@i%
\the\c@sdi-mark}%
\sdi@check-write-bls%
\if#1\empty\else\write\sdi-file\expandafter{\expandafter\field-mark@s#1}\fi%
\write\sdi-file\expandafter{\expandafter\field-mark@p%
\thepage^^J%
$x: \topt\pdflastxpos^^J%
$y: \topt\pdflastypos}%
\write-sdi-val-end}


\newcommand{\sdi-no-ss}{%
\pdfsavepos\write-sdi-val-start{NoSentenceFlag}%
\sdi@check-write-bls%
\write\sdi-file\expandafter{\expandafter\field-mark@r%
\the\c@sdi-counter-P^^J%
$x: \topt\pdflastxpos^^J%
$y: \topt\pdflastypos}%
\write-sdi-val-end%
\sdi-no-ss-flagtrue}

\newcommand{\sss}{%
\ifsdi-no-ss-flag\sdi-no-ss-flagfalse\else%
\stepcounter{sdi-counter-O}%
\stepcounter{sdi-counter-C}%
\stepcounter{sdi-counter-FS}%
\stepcounter{sdi-counter}%
\pdfsavepos\write-sdi-start{Sentence}%
\write\sdi-file\expandafter{\expandafter\field-mark@i%
\the\c@sdi-counter}%
\write\sdi-file\expandafter{\expandafter\field-mark@r%
\the\c@sdi-counter-P}%
\write\sdi-file\expandafter{\expandafter\field-mark@c%
\the\c@sdi-counter-C}%
\write\sdi-file\expandafter{\expandafter\field-mark@o%
\the\c@sdi-counter-O}%
\sdi@check-write-bls%
\write\sdi-file\expandafter{\expandafter\field-mark@p%
\thepage}
\write\sdi-file\expandafter{\expandafter\field-mark@f%
\the\c@sdi-counter-FS^^J%
$x: \topt\pdflastxpos^^J%
$y: \topt\pdflastypos}%
\write-sdi-val-end%
\fi}

\newcommand{\sse}{%
\pdfsavepos\write-sdi-end{Sentence}%
\write\sdi-file\expandafter{\expandafter\field-mark@i%
\the\c@sdi-counter}%
\write\sdi-file\expandafter{\expandafter\field-mark@o%
\the\c@sdi-counter-P}%
\write\sdi-file\expandafter{\expandafter\field-mark@p%
\thepage^^J%
$x: \topt\pdflastxpos^^J%
$y: \topt\pdflastypos}%
\write-sdi-val-end}

\newcommand{\sps}{%
\stepcounter{sdi-counter-F}%
\stepcounter{sdi-counter-P}%
\stepcounter{sdi-counter-PO}%
\pdfsavepos\write-sdi-start{Paragraph}%
\write\sdi-file\expandafter{\expandafter\field-mark@i%
\the\c@sdi-counter-P}%
\write\sdi-file\expandafter{\expandafter\field-mark@o%
\the\c@sdi-counter-PO}%
\sdi@check-write-bls%
\write\sdi-file\expandafter{\expandafter\field-mark@p%
\thepage}
\write\sdi-file\expandafter{\expandafter\field-mark@f%
\the\c@sdi-counter-F}
\write\sdi-file\expandafter{\expandafter\field-mark@j%
\currentjobname^^J%
$x: \topt\pdflastxpos^^J%
$y: \topt\pdflastypos}%
\write-sdi-val-end}%


\newcommand{\spe}{%
\active-sdi-markfalse%
\pdfsavepos\write-sdi-end{Paragraph}%
\write\sdi-file\expandafter{\expandafter\field-mark@i%
\the\c@sdi-counter-P}%
\write\sdi-file\expandafter{\expandafter\field-mark@o%
\the\c@sdi-counter-PO}%
\write\sdi-file\expandafter{\expandafter\field-mark@p%
\thepage^^J%
$x: \topt\pdflastxpos^^J%
$y: \topt\pdflastypos}%
\write-sdi-val-end%
\sdi-no-ss-flagfalse}
\makeatother

\newcommand{\write-sdi-ntxh-start}{\immediate\write\sdi-file{%
- Generated from LaTeX File \jobname.tex ^^J%
  \the\day/\the\month/\the\year, \the\currenthour:\the\currentminute ^^J%
.^^J%
^^J%
- i=id n=name (entity kind) k=element kind p=page r=paragraph o=order (on page) s=size^^J%
  c=count (in paragraph) j=current jobname f=in-file id  x,y=pdf coords b=baselineskip^^J%
.^^J%
^^J%
&type SDI_Sentence_Start {10}^^J%
  :n:1 :i:2 :r:3 :c:4 :o:5 :p:7 :f:8 :x:9 :y:10 :b:6 ;^^J%
^^J%
&type SDI_Sentence_End {6}^^J%
  :n:1 :i:2 :o:3 :p:4 :x:5 :y:6 ;^^J%
^^J%
&type SDI_Paragraph_Start {9}^^J%
  :n:1 :i:2 :o:3 :p:5 :j:6 :f:7 :x:8 :y:9 :b:4 ;^^J%
^^J%^^J%
&type SDI_Paragraph_End {6}^^J%
  :n:1 :i:2 :o:3 :p:4 :x:5 :y:6 ;^^J%
^^J%
&type SDI_Mark {8}^^J%
  :n:1 :i:2 :o:3 :r:4 :p:6 :x:7 :y:8 :b:5 ;^^J%
^^J%
&type SDI_Mark_Secondary {6}^^J%
:n:1 :i:2 :p:4 :x:5 :y:6 :b:3 ;^^J%
^^J%
&type SDI_Element_Start {10}^^J%
:n:1 :i:2 :o:3 :c:4 :r:5 :k:6 :p:8 :x:9 :y:10 :b:7 ;^^J%
^^J%
&type SDI_Element_End {7}^^J%
:n:1 :i:2 :o:3 :p:4 :x:5 :y:6 ;^^J%
^^J%
&type SDI_Info {2}^^J%
:n:1 :global-baseline-skip:2 ;^^J%
^^J%
^^J%
&/^^J%
^^J%
^^J%
!/ SDI_Info^^J%
$n: Info^^J%
$global-baseline-skip: \the\orig-basels^^J%
^^J%
/!^^J%
<+>^^J%
^^J%
^^J%
}}

\newcommand{\write-sdi-ntxh-end}{\write\sdi-file{%
^^J%
/&^^J%
^^J%
}}

\AtBeginDocument{{\write-sdi-ntxh-start}}
\AtEndDocument{{\write-sdi-ntxh-end}}

\usepackage{embedfile}
{\message{^^J^^JSDI: Trying to embed \jobname.sdi.zip ...} 
\IfFileExists{\jobname.sdi.zip}
{\embedfile{\jobname.sdi.zip}\message{OK^^J^^J}}
{\message{File not found^^J^^J}}}

\message{File}

\oldCCforEXC\oldCCforDASH



\newcommand{\cq}[1]{{\fontfamily{gar}\selectfont ``}#1{\fontfamily{gar}\selectfont ''}}

\newcommand{\dq}[1]{"#1"}

\newCCforEXC\newCCforDASH

\newcommand{\cq-tech}[1]{%
\nhv!sdi-element-start{quote-tech}\cq{#1}%
\nhv!sdi-element-end{quote-tech}}

\newcommand{\dq-tech}[1]{%
\nhv!sdi-element-start{quote-tech}\dq{#1}%
\nhv!sdi-element-end{quote-tech}}


\newcommand{\cq-title}[1]{%
\nhv!sdi-element-start{quote-title}\cq{#1}%
\nhv!sdi-element-end{quote-title}}

\newcommand{\dq-title}[1]{%
\nhv!sdi-element-start{quote-title}\dq{#1}%
\nhv!sdi-element-end{quote-title}}


\newcommand{\cq-qual}[1]{%
\nhv!sdi-element-start{quote-qual}\cq{#1}%
\nhv!sdi-element-end{quote-qual}}

\newcommand{\dq-qual}[1]{%
\nhv!sdi-element-start{quote-qual}\dq{#1}%
\nhv!sdi-element-end{quote-qual}}


\newcommand{\cq-attr}[1]{%
\nhv!sdi-element-start{quote-attr}\cq{#1}%
\nhv!sdi-element-end{quote-attr}}

\newcommand{\dq-attr}[1]{%
\nhv!sdi-element-start{quote-attr}\dq{#1}%
\nhv!sdi-element-end{quote-attr}}


\newcommand{\cq-intr}[1]{%
\nhv!sdi-element-start{quote-intr}\cq{#1}%
\nhv!sdi-element-end{quote-intr}}

\newcommand{\dq-intr}[1]{%
\nhv!sdi-element-start{quote-intr}\dq{#1}%
\nhv!sdi-element-end{quote-intr}}



\oldCCforEXC\oldCCforDASH




\AtEndDocument{\immediate\write18{cd /media/mint/MainVolume/gits/brtest/ntxh/wip-facs/ar/code/cpp/qmake-console/projects/gtagml/ngml-sdi-console; ./run-with-args.sh /media/mint/MainVolume/gits/brtest/ntxh/wip-facs/ar/dev/consoles/gtagml/sdi/t1/t1.gt.tex /media/mint/MainVolume/gits/brtest/ntxh/wip-facs/ar/dev/consoles/gtagml/sdi/t1/t1.gt.sdi.ntxh /media/mint/MainVolume/gits/brtest/ntxh/wip-facs/ar/dev/consoles/gtagml/sdi/t1/t1.gt.sdi-prelatex.ntxh; cd -;}}



\p{\:\+Taking this further, we can define a \i{channel} as a list of carriers which, by
inter-carrier transfers, signify (or orchestrate) the passage of data into and out
of function bodies (note that this usage varies somewhat from process calculi,
where a channel would correspond roughly to what is here called a single carrier\svoid{}
I assume channels in the general case are composed of multiple carriers).\;
I'll use the notation \opTransfer{} to represent inter-carrier transfer\svoid{} let
\carrOne{} and \carrTwo{} be carriers, then \carrOneOpTransferTwo{} is a
transfer \q{operator} (note that \opTransfer{} is non-commutative\svoid{} the
\q{transfer} happens in a fixed direction), marking the
logical moment when a value is moved from code-point to code-point.\;
The \opTransfer{} is intended to model several 
scenarios, including \q{forced coercions} where the associated
value is modified.\; Meanwhile, without further details a \q{transfer} can be
generalized to \i{channels} in multiple ways.\;
If \carrOne{} and \carrTwo{} are carriers which belong to two channels
(\chanOne{}, \chanTwo{}), then \carrOneOpTransferTwo{} elevates to
a transfer between the channels
\ndash{} but this needs two indices to be concrete\svoid{} the notation has to
specify which carrier in \chanOne{} transfers to which carrier in \chanTwo{}.\;
For example, consider the basic function-composition \fofg{}\svoid{}
\fDotOfGX{} \svoid{} \fOfGx{}. \> The analogous \q{transfer} notation
would be, say, \gOpTransferOneOneF{}\svoid{}
here the first carrier in the \returnch{} channel
of \funG{} transfers to the first carrier in the \lambdach{} channel of \funF{}
(the subscripts indicate the respective positions).\;\<
}

\p{\:\+Most symbols in a function body (and corresponding nodes in a
semantic graph) accordingly represent carriers, which
are either passed in to a function or lexically declared
in a function body. \> Assume each function body corresponds
with one lexical scope which can have subscopes
(the nature of these scopes and how they fit in
graph representation will be addressed later in this section).\;
The \i{declared} carriers are initialized with
values returned from other functions (perhaps the
current function called recursively), which can
include constructors that work on literals (so, the
carrier-binding in source code can look like
a simple assignment to a literal, as in \intieqzero{}).\;
In sum, whether they are passed \i{to} a  function or
declared \i{in} a function, carriers are only initialized
\ndash{} and only participate in the overall semantics
of a program \ndash{} insofar as they are passed to other functions
or bound to their return values.\;\<
}

\p{\:\+Furthermore, both of these cases introduce associations between
different carriers in different areas of source code. \> When a carrier
is passed \i{to} a function, there is a corresponding carrier
(declared in the callee's signature) that receives the
former's value\svoid{} \q{calling a function} means
transferring values between carriers present at the site of
the function call to those present in the function's
implementation. \> Sometimes this works in reverse\svoid{} a function's
return may cause the value of one of its carriers to be
transferred to a carrier in the caller (whatever
carrier is bound to the caller's return value).\;\<
}
