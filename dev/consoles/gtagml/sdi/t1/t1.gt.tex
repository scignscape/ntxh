
\p{\:\+As.\;{\fiatses}\footnote{\+There is a website.\;\<
}
}

\p{\:\+Type-theoretic semantics can also apply Ontological tropes to unpack the overlapping mesh of word-senses,
like \i{material object} or \i{place} or \i{institution}.
\> This mode of analysis is especially well illustrated when competing senses
collide in the same sentence. \> Slightly modifying two examples:\;{\sssg}\footnote{
\cite[p. 40]{ChatzikyriakidisLuo} (former) and
\cite[p. 4]{MeryMootRetore} (latter).
}
{\srsg}\begin{sentenceList}
\sentenceItem{} \swl{}{The newspaper you are reading is being sued.}{lex}
\sentenceItem{} \swl{itm:Liverpool}{Liverpool, an important harbor, built new docks.}{lex}
\end{sentenceList}
\+Both have a mid-sentence shift between senses, which is analyzed
in terms of \q{type coercions} (see also \cite{ZhaohuiLuo}
and \cite{ZhaohuiLuoSignatures}).
\> The interesting detail of this treatment
is how it correctly predicts that such coercions are not guaranteed to
be accepted:\;
\begin{sentenceList}
\sentenceItem{} \swl{}{The newspaper fired a reporter and fell off
the table.}[(?)]{lex}
\sentenceItem{} \swl{}{Liverpool beat Tottenham and built new docks.}[(?)]{lex}
\end{sentenceList}
(\+again, slightly modifying the counter-examples). \> Type coercions are
\i{possible} but not \i{inevitable}. \> Some word-senses \q{block} certain coercions
\mdash{} that is, certain sense combinations, or juxtapositions, are disallowed.
\> These preliminary, motivating analyses carry to more
complex and higher-scale types, like plurals (the plural of a type-coercion
works as a type-coercion of the plural, so to speak).
\> As it becomes structurally established that type rules at the
simpler levels have correspondents at more complex levels, the use of
type notions \i{per se} (rather than just \q{word senses} or other
classifications) becomes more well-motivated.\;\<
}
