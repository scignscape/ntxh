

\p{\:\+The question of whether computers can be programmed to understand 
language may be philosophical, but it overlaps with 
broad methodological bifurcations: after all, linguists 
\i{are} programming computers to \q{understand} language, at 
least to some approximation. \> Given that computational 
linguistics is now a well-established practice, we can 
consider how this program for investigating the nature of 
language orients into linguistics as a whole: to what degree 
are the computers really \q{understanding} their linguistic 
input? \> How much does \i{behavior} consistent with language-understanding 
suggest actual understanding? \> Is linguistic competence mostly a 
behavioral phenomenon, or something more holistic and (inter-) subjective? \> 
Are the imperfections of automated Natural Language Processing 
inevitable, and if so, does that foreclose the possibility of 
\NLP{} engines being considered truly linguistic? \> That is, 
should we treat flawed and oversimplistic (but practically useful) 
\NLP{} software \ndash{} or \q{personas} driven by this software, like 
\q{digital assistants} \ndash{} as bonafide (if rather primitive) 
participants in the world of human language? \> Or are they merely 
machines that simulate linguistic behavior without manifesting 
real linguistic behavior, as a computer simulation of a 
celestial galaxy is not a real galaxy?\;\<  
}

\p{\:\+These are methodological as well as thematic questions. \> There is a 
wide swath of formal and computational linguistics, for instance, for 
which the measure of a theory is its chance of being operationalized 
on \NLP{} terms and within \NLP{} tools, yielding automated systems whose accuracy and/or 
computational efficiency competes favorably with other systems. \> 
Faithfulness to how \i{humans} process language is at most a secondary 
concern. \> Conversely, there is a broad literature in Cognitive Linguistics 
and the Philosophy of Language for which uncovering the cognitive and 
interpretive registers through which \i{we} understand, produce, and 
are affected by language is the main goal. \> For scholars chasing 
that telos, failure to encode theoretical models in mathematical 
or software systems is not \i{prima facie} an explanatory limitation 
\ndash{} conversely, we might take this as evidence that cognitive models 
are addressing the deep, subtle realities of language that are 
opaque to computer simulation.\;\<   
}

\p{\:\+Then there is hybrid work, like attempts to formalize 
Cognitive Grammar (Matt Selway \cite{}, 
Kenneth Holmqvist \cite{}, \cite{}), or
other branches of Cognitive Linguistics 
(cf. Terry Regier's influential \cite{}), 
or Conceptual Space Theory as initiated by 
Peter \Gardenfors{} (which has seen several attempts at 
mathematical-computational formalization, such as 
Frank Zenker, Martin Raubal, and Benjamin Adams's 
metascientific perspectives \cite{}, 
\cite{}, \cite{}, and more recent 
Category-Theoretic structures linked to mathematicians 
such as Bob Coecke and David Spivak \cite{}). \> 
To this list we could add research that extends beyond 
language alone to broader cognitive-perceptual and 
conceptual themes, like formal descriptions rooted in 
Husserlian analyses by phenomenologists whose methods 
encompass some computer-scientific techniques, like 
Barry Smith (as in \cite{}) and 
Jean Petitot (see \cite{}); we can see 
these accounts as generalizing cognitive-linguistic 
theories by noting the phenomenological basis 
of linguistic phenomena, as articulated by (say) Olav 
K. Wiegand (\cite{}, \cite{}) 
and Jordan Zlatev (\cite{}). \> 
In each of the works just cited (prior anyhow to the 
last three)  
we can find formal/computational models whose 
rationale is, in large part, to 
shed light on human cognitive processes 
(albeit not necessarily translating to practical 
\NLP{} components in any straightforward way).\;\< 
}

\p{}

\p{}
