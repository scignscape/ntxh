
\p{\:\+As.\;{\fiatses}\footnote{\+There is a website.\;\<
}
}

\p{\:\+In any case, these variations present some potentially
informative characteristics about individual types.
\> Given type \tTyp{}, we can ask questions like:\;{\fiatses}\begin{enumerate}

\item{} \+Which instances of \tTyp{}, if any, can be the result of
a trivial constructor?\;  
\item{} \+Does \tTyp{} have a \i{default value} whish is the result of
a trivial constructor?\;  
\item{} \+Which instances of \tTyp{} can be the result of a
literal constructor?\;  
\item{} \+Which instances of \tTyp{}, if any, can be the result of
a reversible constructor?\;  
\item{} \+If we have values which \i{are} the result of
a reversible constructor, is there an efficient way
to \q{un-construct} the value to support pattern matching?\;  
\item{} \+Does \tTyp{} have co-constructors (i.e., they
are not literal) which also are neither trivial nor reversible?\;  
\end{enumerate}

\+Notice that a trivial constructor does not necessarily produce
a default value. \> For example, a type meant to represent
days of the week could default to whichever day is current
when a constructor is called: if an application is run on
Tuesday, the day-of-week trivial constructor would return
the value for Tuesday. \> So a type may have \i{more
than one} trivial-constructed values. \> But
if a type has \i{exactly one} trivial-constructed value,
this is \i{usually} a \q{default} value.\;\<
}
