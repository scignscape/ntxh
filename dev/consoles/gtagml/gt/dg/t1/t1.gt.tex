\p{The phrase \i{latest dogs} carries implications in its own
right; we assume the neighbor had owned other dogs before.
Of course \q{latest} implies some temporal order, but the
understood time-scale depends on context.
If we hear talk about a \i{vet}'s two latest dogs, we would presumably
interpret this in terms of patients the vet has seen over the course of
a day:

\begin{sentenceList}
\sentenceItem{} \swl{itm:vet}{We have to wait until after the vet's two latest dogs.}{;;
\sentenceItem{} \swl{itm:organization}{I'm concerned for the rescue organization's
two latest dogs.}{sem}
}

Understanding the relevant time-frame depends on understanding the relation
between the dogs and the possessive antecedent.  In (\ref{itm:latest})
the neighbor (in a typical case) actually owns the dogs, so the
situational context grounding the modifier \i{latest} would be understood
against the normal time-scale for dog ownership (at least several years).
In (\ref{itm:vet}), the vet only \q{possesses} the dogs in the sense
of endeavoring to examine them, a process of minutes or hours.
In (\ref{itm:organization}), the implication of the \i{organization's}
possessive \visavis{} rescued dogs is that the group endeavors
to rehabilitate and find permanent homes for the rescuees.  So in each
case \i{latest}\end{sentenceList}
}
