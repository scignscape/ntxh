

\p{\:\+According to Miriam Fried and Jan-Ola \"{O}stman,

\begin{quote}The primary motivation for Construction Grammar is the insight that
the juxtaposition of two or more forms seldom results in a simple
concatenation of the meanings those forms might have in isolation.
\> Consequently, Construction Grammar sees linguistic units as particular
associations between form and meaning that must be represented as such,
rather than leaving such associations to the operation of a set of rules for how
to combine individual forms.\;
\cite[page 3]{FriedOstman}
\end{quote}

\+I would qualify this paradigm merely by saying that we should
not discount the compositionality of language merely on the
basis of linguistic form not seeming to reciprocate
\i{logical} compositions, as I have stressed numerous times here.\;\<
}
