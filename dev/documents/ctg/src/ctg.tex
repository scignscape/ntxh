
\AddToShipoutPicture{%
  \AtPageLowerLeft{%
    \hspace*{3pt}
 \rotatebox{90}{%
        \begin{minipage}{\paperheight}
   \centering
   {\color{codegr!65}\textcopyright ~\today{} Nathaniel Christen}
        \end{minipage} %
      }
    } %
  }%

\begin{document}

\title{From \cq{Naturalizing Phenomenology} to 
Formalizing Cognitive Linguistics (I): Cognitive Transform Grammar}
\author{Nathaniel Christen}
\newsavebox{\qboxi}
\newsavebox{\qboxii}
\begin{lrbox}{\qboxi}
\begin{frquote}On conna\^{\OldI}t la c\'{e}l\`{e}bre affirmation de Claude L\'{e}vi-Strauss: 
\q{les sciences humaines seront structurales ou ne seront pas}.  Nous aimerions lui en
adjoindre une autre: \q{les sciences humaines seront des sciences naturelles ou ne seront pas}. 
Evidemment, sauf \`{a} en revenir \`{a} un r\'{e}ductionnisme dogmatique, une telle
affirmation n'est soutenable que si l'on peut suffisamment g\'{e}n\'{e}raliser le concept
classique de \q{naturalit\'{e}}, le g\'{e}n\'{e}raliser jusqu'\`{a} pouvoir y faire droit, 
comme \`{a} des ph\'{e}nom\`{e}nes naturels, aux ph\'{e}nom\`{e}nes d'organisation structurale.
\\ \longdash{} Jean Petitot, \cite[p. 1]{PetitotSyntaxe}
\end{frquote}
\end{lrbox}	
\begin{lrbox}{\qboxii}
\begin{frquote}The nature of any entity, I propose, divides into three aspects or facets, which we may call its
	form, appearance, and substrate.  In an act of consciousness, accordingly, we must distinguish
	three fundamentally different aspects: its form or intentional structure, its appearance or
	subjective \q{feel}, and its substrate or origin.  In terms of this three-facet distinction, 
	we can define the place of consciousness in the world.
\\ \longdash{} David Woodruff Smith, \cite[p. 11]{DavidWoodruffSmith}
\end{frquote}
\end{lrbox}	
\twocolumn[\begin{@twocolumnfalse}
\maketitle{}
\begin{abstract}\end{abstract}
\begin{flushright}\usebox{\qboxi}
\usebox{\qboxii}
\end{flushright}
\decoline{}
\vspace{3em}
\end{@twocolumnfalse}]
\addcontentsline{toc}{section}{From \dq{Naturalizing Phenomenology} ...}


\sdiinput{intro.gt}

\sdiinput{section1.gt}
\sdiinput{section2.gt}
\sdiinput{section3.gt}
\sdiinput{section4.gt}




\begin{thebibliography}{99}
\phantomsection \label{References}
\addcontentsline{toc}{section}{References}
{\fontfamily{lmtt}\selectfont\scriptsize



\bibitem{AsudehGiorgolo}
Ash Asudeh and Gianluca Giorgolo,  
\bibtitle{Enriched Meanings: Natural Language Semantics with Category Theory}.
Oxford, 2020. 


\bibitem{AsherLuo}
Nicholas Asher and Zhaohui Luo,  
\cq{Formalization of Coercions in Lexical Semantics}.
\biburl{https://semanticsarchive.net/sub2012/AsherZhaoHui.pdf}


\bibitem{JerryTBall}
Jerry T. Ball,  
\cq{Double R Grammar: The Grammatical 
Encoding of Referential and Relational Meaning in
English}.
\biburl{https://citeseerx.ist.psu.edu/document?repid=rep1&type=pdf&doi=34c258564f72b6dc6455aeaac6e6be1a4a07d262}


\bibitem{BarkerShan}
Chris Barker and Chung-chieh Shan,  
\bibtitle{Continuations and Natural Language}.
Oxford, 2014. 


\bibitem{LawrenceWBarsalou}
Lawrence W. Barsalou,  
\cq{Grounded Cognition}.
\biburl{http://matt.colorado.edu/teaching/highcog/spr10/readings/b8.pdf}



\bibitem{HansCBoas}
Hans C. Boas, 
\cq{Cognitive Construction Grammar}.
\biburl{https://sites.la.utexas.edu/hcb/files/2011/02/CCxG-October22-2010.pdf}


\bibitem{GregCarlson}
Greg Carlson,
\cq{Thematic Roles and the Individuation of Events}
\biburl{https://www.sas.rochester.edu/lin/people/faculty/carlson_greg/assets/pdf/them-roles-events.pdf}


\bibitem{LucasChampollionC}
Lucas Champollion,
\cq{Covert Distributivity in Algebraic Event Semantics}
\intitle{Semantics \& Pragmatics}, Volume 9 (2016), pages 1-65. 
\biburl{https://semprag.org/article/view/sp.9.15}


\bibitem{ChampollionDissertation}
\_\_\_\_\_,
\cq{Parts of a Whole: Distributivity as a Bridge Between Aspect and Measurement}
Dissertation, University of Pennsylvania, 2010.
\biburl{https://repository.upenn.edu/cgi/viewcontent.cgi?article=2117&context=edissertations}



\bibitem{RobinCooper}
Robin Cooper,
\cq{Perception, Types and Frames}
\biburl{https://link.springer.com/chapter/10.1007/978-3-030-50200-3_8}


\bibitem{CopleyHarley}
Bridget Copley and Heidi Harley,
\cq{A Force-Theoretic Framework for Event Structure}
\biburl{http://heidiharley.com/heidiharley/wp-content/uploads/2016/09/CopleyAndHarley2015Published.pdf}






\bibitem{DeWitBrisard}
Astrid De Wit and Frank Brisard,
\cq{A Cognitive Grammar Account of the Semantics of the English Present Progressive}
\biburl{https://www.researchgate.net/publication/326960775_Nominal_Grounding_Elements_in_English_A_Domain-based_Account}



\bibitem{DuziFait}
Marie Du\v{z}\i{\OldI} and Michal Fait,
\cq{Integrating Special Rules Rooted in Natural Language Semantics into
the System of Natural Deduction}
\biburl{https://www.scitepress.org/Papers/2020/93696/93696.pdf}



\bibitem{MullerWechsler}
Stefan M\"uller and Stephen Wechsler,
\cq{Lexical Approaches to Argument Structure}
\biburl{https://hpsg.hu-berlin.de/~stefan/Pub/arg-st.pdf}



\bibitem{MustafaFatah}
Kobeen Raouf Mustafa and Azad Hasan Fatah,
\cq{Nominal Grounding Elements in English:
A Domain-based Account}
\biburl{https://www.researchgate.net/publication/326960775_Nominal_Grounding_Elements_in_English_A_Domain-based_Account}


\bibitem{JeanMarkGawron}
Jean Mark Gawron,
\cq{Circumstances and Perspective:
The Logic of Argument Structure}
\biburl{https://escholarship.org/content/qt7sd5987t/qt7sd5987t_noSplash_14d78590a8657a71ed9ad46dc6e69064.pdf}


\bibitem{JeanMarkGawronPaths}
\_\_\_\_\_,
\cq{Paths and the Language of Change}
\biburl{https://gawron.sdsu.edu/paths_change.pdf}


\bibitem{ChrisGenovesi}
Chris Genovesi,
\cq{Metaphor and What is Meant: Metaphorical content, what is
said, and contextualism}
\biburl{https://www.tcd.ie/slscs/assets/documents/Clinical-speech/genovesi_2019_metaphor.pdf}



\bibitem{RobertFHadley}
Robert F. Hadley,
\cq{A Default-Oriented Theory of
Procedural Semantics}
\biburl{https://onlinelibrary.wiley.com/doi/pdf/10.1207/s15516709cog1301_4}


\bibitem{DanielWHarris}
Daniel W. Harris,
\cq{Speech Act Theoretic Semantics}.
Dissertation, CUNY, 2014.
\biburl{https://academicworks.cuny.edu/cgi/viewcontent.cgi?article=1221&context=gc_etds}


\bibitem{DarrinLouisHindsill}
Darrin Louis Hindsill,
\cq{Its a Process \i{and} an Event:
Perspectives in event semantics}.
Dissertation, University of Amsterdam, 2007.
\biburl{https://eprints.illc.uva.nl/id/eprint/2058/1/DS-2007-03.text.pdf}



\bibitem{KennethHolmqvist}
Kenneth Holmqvist,
\cq{Conceptual Engineering: Implementing cognitive semantics}, 
in Jens Allwood and Peter \Gardenfors, eds., 
\intitle{Cognitive Semantics}, pp 153 - 171, Amsterdam,
Philadelphia: John Benjamins, 1999.

\bibitem{KennethHolmqvistDissertation}
Kenneth Holmqvist, 
\cq{Implementing Cognitive Semantics: Image schemata, 
valence accommodation, and valence suggestion for  
AI and computational linguistics}.
PhD thesis, Dept. 


\bibitem{HornPeirce}
Laurence R. Horn and Istvan Keckes Peirce,
\cq{Pragmatics, Discourse, and Cognition}
\biburl{http://www.albany.edu/faculty/ikecskes/files/Horn\%20and\%20Kecskes\%2065-45-R1026-Horn.pdf}


\bibitem{WernerKallmeyer}
Werner Kallmeyer,
\cq{Verbal Practices of Perspective Grounding}
\biburl{https://d-nb.info/120819738X/34}


\bibitem{KyleJohnson}
Kyle Johnson,
\cq{What Makes a Theta-role}
\biburl{https://www.qmul.ac.uk/sllf/media/sllf-new/department-of-linguistics/hagit-borer-celebration/Johnson.pdf}


\bibitem{MarkJohnson}
Mark Johnson, \cq{The Body in the Mind: 
The Bodily Basis of Meaning, Imagination, and Reason}.
University of Chicago Press, 1990


\bibitem{LakoffJohnson}
George Lakoff and Mark Johnson, 
\bibtitle{Philosophy in the Flesh: the Embodied Mind and its Challenge to Western Thought}.
New York: Basic Books, 1999.


\bibitem{LakoffJohnsonMetaphors}
George Lakoff and Mark Johnson, 
\bibtitle{Metaphors We Live By}.
University of Chicago, 1980.


\bibitem{LangackerIntro}
Ronald Langacker, 
\bibtitle{Cognitive Grammar: A Basic Introduction}.  
Oxford, 2008.

\bibitem{LangackerCCxG}
Ronald Langacker, 
\cq{Cognitive (Construction) Grammar}.  
\biburl{https://www.degruyter.com/document/doi/10.1515/COGL.2009.010/html?lang=en}


\bibitem{LangackerConstructions}
\_\_\_\_\_, 
\cq{Constructions in Cognitive Grammar}.  
\biburl{https://www.jstage.jst.go.jp/article/elsj1984/20/1/20_1_41/_pdf/-char/en}

\bibitem{LangackerConceptualization}
\_\_\_\_\_, 
\bibtitle{Grammar and Conceptualization} 
de Gruyter, 1999). 

\bibitem{LangackerClause}
\_\_\_\_\_, 
\cq{How to Build and English Clause}.  
\biburl{https://oaji.net/articles/2016/3124-1458506162.pdf}


\bibitem{LangackerEvidentiality}
\_\_\_\_\_, 
\cq{Evidentiality in Cognitive Grammar}.  
\biburl{https://benjamins.com/catalog/pbns.271.02lan}


\bibitem{LangackerPresent}
\_\_\_\_\_, 
\cq{The English Present: 
Temporal coincidence vs. epistemic immediacy}.  
\biburl{https://benjamins.com/catalog/hcp.29.06lan}


\bibitem{LebaniLenci}
Gianluca E. Lebani and Alessandro Lenci,
\cq{A Distributional Model of Verb-Specific
Semantic Roles Inferences}
\biburl{https://colinglab.humnet.unipi.it/wp-content/uploads/2012/12/Lebani-Lenci-2018.pdf}


\bibitem{JosephLehmann}
Joseph Lehmann, \i{et. al.},
\cq{Age-Related Hearing Loss, Speech 
Understanding and Cognitive Technologies}.
\biburl{https://link.springer.com/article/10.1007/s10772-021-09817-z}




\bibitem{SanderLestrade}
Sander Lestrade, 
\cq{The Space of Case}.
\biburl{https://repository.ubn.ru.nl/bitstream/handle/2066/82611/82611.pdf?sequence=1&isAllowed=y}



\bibitem{ClaudiaMaienborn}
Claudia Maienborn,
\cq{Event Semantics}.
\biburl{https://www.researchgate.net/publication/236898346_Event_semantics}




\bibitem{ErwanMoreau}
Erwan Moreau,
\cq{From Link Grammars to Categorial Grammars}
\biburl{https://hal.archives-ouvertes.fr/hal-00487053/document}


\bibitem{NeusteinChristen}
Amy Neustein and Nathaniel Christen, 
\bibtitle{Covid, Cancer, and Cardiac Care}. Elsevier, 2022. 



\bibitem{PeterPelyvas}
P\'eter Pelyv\'as, 
\cq{Epistemic Modality: A choice 
between alternative cognitive models}.
\biburl{https://litere.uvt.ro/publicatii/BAS/pdf/no/bas_2011_articles/22\%20247-259.pdf}


\bibitem{PeterPelyvasDeontic}
\_\_\_\_\_, 
\cq{On Epistemic and Deontic Grounding}.
\biburl{https://argumentum.unideb.hu/2019-anyagok/special_issue_I/pelyvasp.pdf}


\bibitem{PeterPelyvasDevelopment}
\_\_\_\_\_, 
\cq{The Development of the Grounding Predication: Epistemic Modals and Cognitive Predicates}.
\biburl{https://brill.com/display/book/9780585474267/B9780585474267_s010.xml}


\bibitem{JeanPetitot}
Jean Petitot, \cq{The Morphodynamical Turn of Cognitive Linguistics}.
\biburl{https://journals.openedition.org/signata/549}


\bibitem{PetitotDoursat}
Jean Petitot and Ren\'e Doursat, 
\bibtitle{Cognitive Morphodynamics:
Dynamical Morphological Models of Constituency in Perception and Syntax}.
Peter Lang, 2011


\bibitem{JamesPustejovsky}
James Pustejovsky, 
\cq{The Syntax of Event Structure}.
\biburl{https://www.cs.rochester.edu/u/james/Papers/Pustejovsky-event-structure.pdf}.




\bibitem{ShimEpstein}
Jae-Young Shim and Samuel David Epstein,
\cq{Two Notes on Possible Approaches
to the Unification of Theta Relations}
\biburl{https://jaeyoungshim.weebly.com/uploads/1/2/3/1/12312623/linguistic_analysis_two_notes_jae-young_shim.pdf}


\bibitem{GeroldSchneider}
Gerold Schneider, 
\cq{A Linguistic Comparison of Constituency, Dependency and Link Grammar}.
Zurich University, Diploma, 2008.%
\hspace{-.6em}\biburl{https://files.ifi.uzh.ch/cl/gschneid/papers/FINALSgeroldschneider-latl.pdf}



\bibitem{SleatorTamperley}
Daniel D. Sleator and Davy Tamperley, 
\cq{Parsing English with a Link Grammar}.
\biburl{https://www.link.cs.cmu.edu/link/ftp-site/link-grammar/LG-IWPT93.pdf}


\bibitem{MichaelSprangerIncremental}
Michael Spranger, 
\cq{Incremental Grounded Language Learning in
Robot-Robot Interactions}.
\biburl{http://www2.ece.rochester.edu/projects/rail/mlhrc2015/papers/mlhrc-rss15-spranger.pdf}



\bibitem{MichaelSprangerRobots}
Michael Spranger, 
\cq{Procedural Semantics for Autonomous Robots -- A Case Study in
Locative Spatial Language}.
\biburl{https://www.researchgate.net/publication/280830547_Procedural_Semantics_for_Autonomous_Robots_-_A_Case_Study_in_Locative_Spatial_Language}



\bibitem{MichaelSprangerEvolution}
\_\_\_\_\_, 
\cq{The Evolution of Grounded Spatial Language}.
\biburl{https://langsci-press.org/catalog/book/53}


\bibitem{MichaelSpranger}
Michael Spranger, \i{et. al,},
\cq{Open-ended Procedural Semantics}.
\biburl{https://martin-loetzsch.de/publications/spranger12openended.pdf}





\bibitem{SteffensenFill}
Sune Vork Steffensen and Alwin Fill,
\cq{Ecolinguistics: The state of the art and future horizons}
\biburl{https://findresearcher.sdu.dk/ws/files/87218550/Steffensen_2014_Language_Sciences.pdf}


\bibitem{ChukwualukaMichaelUyanne}
Chukwualuka Michael Uyanne, \i{et. al.},
\cq{Ecolinguistic Perspective: Dialectics of Language
and Environment}
\biburl{https://ezenwaohaetorc.org/journals/index.php/ajells/article/viewFile/5-1-2014-010/61}





\bibitem{OrlinVakarelov}
Orlin Vakarelov,
\cq{Pre-cognitive Semantic Information}.
\biburl{https://link.springer.com/article/10.1007/s12130-010-9109-5}

\bibitem{VakarelovAgent}
\_\_\_\_\_, 
\cq{The Cognitive Agent: Overcoming
informational limits}
\biburl{https://philarchive.org/archive/VAKTCAv1}


\bibitem{VepstasGoertzel}
Linas Vepstas and Ben Goertzel, 
\cq{Learning Language from a Large (Unannotated) Corpus}
\biburl{https://arxiv.org/pdf/1401.3372.pdf}


\bibitem{JordanZlatev}
Jordan Zlatev, 
\cq{Embodiment, Language, and Mimesis}
\biburl{https://lucris.lub.lu.se/ws/portalfiles/portal/4514085/1044802.pdf}


\bibitem{JordanZlatevPhenomenology}
\_\_\_\_\_, 
\cq{Phenomenology and Cognitive Linguistics}
\biburl{https://citeseerx.ist.psu.edu/document?repid=rep1&type=pdf&doi=ebc99d9442497d36be7477f393dcc38d2d53ab5b}


}
\end{thebibliography}









\end{document}
