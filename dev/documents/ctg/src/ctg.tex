
\AddToShipoutPicture{%
  \AtPageLowerLeft{%
    \hspace*{3pt}
 \rotatebox{90}{%
        \begin{minipage}{\paperheight}
   \centering
   {\color{codegr!65}\textcopyright ~\today{} Nathaniel Christen}
        \end{minipage} %
      }
    } %
  }%

\begin{document}

\title{From \cq{Naturalizing Phenomenology} to 
Formalizing Cognitive Linguistics (I): Cognitive Transform Grammar}
\author{Nathaniel Christen}
\newsavebox{\qboxi}
\newsavebox{\qboxii}
\begin{lrbox}{\qboxi}
\begin{frquote}On conna\^{\OldI}t la c\'{e}l\`{e}bre affirmation de Claude L\'{e}vi-Strauss: 
\q{les sciences humaines seront structurales ou ne seront pas}.  Nous aimerions lui en
adjoindre une autre: \q{les sciences humaines seront des sciences naturelles ou ne seront pas}. 
Evidemment, sauf \`{a} en revenir \`{a} un r\'{e}ductionnisme dogmatique, une telle
affirmation n'est soutenable que si l'on peut suffisamment g\'{e}n\'{e}raliser le concept
classique de \q{naturalit\'{e}}, le g\'{e}n\'{e}raliser jusqu'\`{a} pouvoir y faire droit, 
comme \`{a} des ph\'{e}nom\`{e}nes naturels, aux ph\'{e}nom\`{e}nes d'organisation structurale.
\\ \longdash{} Jean Petitot, \cite[p. 1]{PetitotSyntaxe}
\end{frquote}
\end{lrbox}	
\begin{lrbox}{\qboxii}
\begin{frquote}The nature of any entity, I propose, divides into three aspects or facets, which we may call its
	form, appearance, and substrate.  In an act of consciousness, accordingly, we must distinguish
	three fundamentally different aspects: its form or intentional structure, its appearance or
	subjective \q{feel}, and its substrate or origin.  In terms of this three-facet distinction, 
	we can define the place of consciousness in the world.
\\ \longdash{} David Woodruff Smith, \cite[p. 11]{DavidWoodruffSmith}
\end{frquote}
\end{lrbox}	
\twocolumn[\begin{@twocolumnfalse}
\maketitle{}
\begin{abstract}\end{abstract}
\begin{flushright}\usebox{\qboxi}
\usebox{\qboxii}
\end{flushright}
\decoline{}
\vspace{3em}
\end{@twocolumnfalse}]
\addcontentsline{toc}{section}{From \dq{Naturalizing Phenomenology} ...}


\sdiinput{intro.gt}

\sdiinput{section1.gt}
\sdiinput{section2.gt}
\sdiinput{section3.gt}
\sdiinput{section4.gt}

\begin{thebibliography}{99}
\phantomsection \label{References}
\addcontentsline{toc}{section}{References}
{\fontfamily{lmtt}\selectfont\scriptsize


\bibitem{KyounghoAn}
Kyoungho An, \i{et. al.},
\cq{Cloud Computing for Cyber Physical Systems:
Reliability and Security Challenges and Solutions}.
\biburl{http://www.dre.vanderbilt.edu/~gokhale/WWW/papers/CC4CPS13_FT.pdf} 


\bibitem{AneesAra}
Anees Ara, \i{et. al.},
\cq{A Secure Service Provisioning Framework for Cyber-Physical 
Cloud Computing Systems}.
\biburl{https://arxiv.org/abs/1611.00374} 


\bibitem{BerghoferUrban}
Stefan Berghofer and Christian Urban,
\cq{A Head-to-Head Comparison of
de Bruijn Indices and Names}.
\biburl{https://www.sciencedirect.com/science/article/pii/S1571066107002319} 


\bibitem{YiCai}
Yi Cai, \i{et. al.},
\cq{Sensor Data and Information Fusion to Construct Digital-Twins Virtual
Machine Tools for Cyber-Physical Manufacturing}.
\biburl{https://www.researchgate.net/publication/318291645_Sensor_Data_and_Information_Fusion_to_Construct_Digital-twins_Virtual_Machine_Tools_for_Cyber-physical_Manufacturing/link/596140e8aca2728c11e0b5fe/download} 


\bibitem{CardoneHindley}
Felice Cardone and J. Roger Hindley,
\cq{History of Lambda-Calculus and
Combinatory Logic}.
\biburl{http://www.users.waitrose.com/~hindley/SomePapers_PDFs/2006CarHin,HistlamRp.pdf} 




\bibitem{NathanielChristenCyberphysical}
Nathaniel Christen, 
\cq{Hypergraph Type Theory for 
Specifications-Conformant Code and 
Generalized Lambda Calculus}.
\intitle{Advances in Ubiquitous Computing: Cyber-Physical Systems, Smart Cities, and Ecological Monitoring}, Amy Neustein, ed., Elsevier, 2019. 



\bibitem{LironCohen}
Liron Cohen, \i{et. al.},
\cq{The Effects of Effects on Constructivism}.
\biburl{https://www.cs.bgu.ac.il/~cliron/pubs/Effects.pdf} 

\bibitem{ReneDavid}
Ren\'e David, \i{et. al.},
\cq{A Direct Proof of the Confluence of Combinatory Strong Reduction}.
\biburl{https://arxiv.org/abs/0905.2545} 


\bibitem{ParastooDelgoshaei}
Parastoo Delgoshaei, \i{et. al.},
\cq{A Semantic Framework for Modeling and Simulation of Cyber-Physical Systems}.
\biburl{https://user.eng.umd.edu/~austin/reports.d/IARIA2014-PD-MA-AP-Journal-Paper.pdf} 


\bibitem{GianantonioHonsell}
Pietro Di Gianantonio and Furio Honsell,
\cq{An Abstract Notion of Application}.
\biburl{https://users.dimi.uniud.it/~pietro.digianantonio/papers/copy_pdf/ana.pdf} 


\bibitem{AlessandraDiPierro}
Alessandra Di Pierro, \i{et. al.},
\cq{On Reversible Combinatory Logic}.
\biburl{https://www.sciencedirect.com/science/article/pii/S1571066106000843} 


\bibitem{JeanHGallier}
Jean H. Gallier,
\cq{Realizability, Covers, and Sheaves: Application to the Simply-Typed Lambda-Calculus}.
\biburl{https://repository.upenn.edu/cgi/viewcontent.cgi?article=1280&context=cis_reports} 


\bibitem{MasahitoHasegawa}
Masahito Hasegawa,
\cq{The Internal Operads of Combinatory Algebras}.
\biburl{https://www.kurims.kyoto-u.ac.jp/~hassei/papers/2022_mfps2022_pre.pdf} 


\bibitem{JRogerHindley}
J. Roger Hindley,
\cq{Axioms for Strong Reduction in Combinatory Logic}.
\biburl{https://www.cambridge.org/core/journals/journal-of-symbolic-logic/article/abs/axioms-for-strong-reduction-in-combinatory-logic/9B7CBE781BAE43FF5148ABE720D4517A} 





\bibitem{HindleySeldin}
J. Roger Hindley and Jonathan P. Seldin,
\cq{Lambda-Calculus and Combinators, an Introduction}.
Cambridge University Press, 2008.
\biburl{https://www.cin.ufpe.br/~djo/files/Lambda-Calculus%20and%20Combinators.pdf} 


\bibitem{HonsellSannella}
Furio Honsell and Donald Sannella,
\cq{Pre-logical Relations}.
\biburl{https://homepages.inf.ed.ac.uk/dts/pub/prelogrel-report.pdf} 




\bibitem{PetrJanda}
Petr Janda, \i{et. al.},
\cq{Implementation of the Digital Twin Methodology}.
\biburl{https://www.daaam.info/Downloads/Pdfs/proceedings/proceedings_2019/072.pdf} 

\bibitem{SamueleMaschio}
Samuele Maschio,
\cq{On Posetal and Complete
Partial Applicative Structures}.
\biburl{https://arxiv.org/pdf/2211.11326.pdf} 


\bibitem{MitchellMoggi}
John C. Mitchell and Eugenio Moggi,
\cq{Kripke-Style Models for Typed Lambda Calculus}.
\biburl{https://person.dibris.unige.it/moggi-eugenio/ftp/kripke.pdf} 


\bibitem{TiagoMuck}
Tiago M\"uck, \i{et. al.},
\cq{CHIPS-AHOy: A Predictable Holistic Cyber-Physical Hypervisor
for MPSoCs}.
\biburl{https://par.nsf.gov/servlets/purl/10119096} 


\bibitem{NeusteinChristen}
Amy Neustein and Nathaniel Christen, 
\bibtitle{Covid, Cancer, and Cardiac Care}. Elsevier, 2022. 


\bibitem{BorjaBordelSanchez}
Borja Bordel S\'anchez, \i{et. al.},
\cq{Enhancing Process Control in Industry 4.0 Scenarios using
Cyber-Physical Systems}.
\biburl{http://isyou.info/jowua/papers/jowua-v7n4-3.pdf} 



\bibitem{VivekKumarSehga}
Vivek Kumar Sehga, \i{et. al.},
\cq{A Comparative Study of Cyber Physical Cloud, Cloud
of Sensors and Internet of Things: Their Ideology,
Similarities and Differences}.
\biburl{https://ieeexplore.ieee.org/document/6779411?arnumber=6779411} 

\bibitem{JonathanPSeldin}
Jonathan P. Seldin,
\cq{The Search for a Reduction in Combinatory Logic Equivalent to
$\lambda\beta$-reduction}.
\biburl{https://core.ac.uk/download/pdf/82407499.pdf} 



\bibitem{MurraySinclair}
Murray Sinclair, \i{et. al.},
\cq{The Identification of Knowledge Gaps in the Technologies of Cyber-Physical Systems with Recommendations for Closing These Gaps}.
\biburl{https://incose.onlinelibrary.wiley.com/doi/10.1002/sys.21464} 


\bibitem{SzymonSzominski}
Szymon Szomi\'nski, \i{et. al.},
\cq{Development of a Cyber-Physical System
for Mobile Robot Control using Erlang}.
\biburl{https://annals-csis.org/Volume_1/pliks/246.pdf} 


\bibitem{ValentiniaVialeb}
Silvio Valentinia and Matteo Vialeb,
\cq{A Binary Modal Logic for the Intersection Types of Lambda-Calculus}.
\biburl{https://www.sciencedirect.com/science/article/pii/S0890540103000890} 


\bibitem{AlexanderVodyaho}
Alexander Vodyaho, \i{et. al.},
\cq{Model Based Approach to Cyber–Physical Systems Status Monitoring}.
\biburl{https://www.mdpi.com/2073-431X/9/2/47/htm} 


\bibitem{WestParmer}
Richard West and Gabriel Parmer,
\cq{A Software Architecture for Next-Generation Cyber-Physical
Systems}.
\biburl{https://www.cs.bu.edu/~richwest/papers/west_cps.pdf} 


\bibitem{YuZhang}
Yu Zhang, \i{et. al.},
\cq{High Fidelity Virtualization of 
Cyber-Physical Systems}.
\biburl{http://web.cecs.pdx.edu/~xie/pubs/IJMSSC13.pdf} 







}
\end{thebibliography}




\end{document}
