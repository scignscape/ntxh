

\usepackage{graphicx}
\usepackage{xcolor}

\setlength{\columnsep}{8.5mm}

\usepackage[letterpaper, left=.85in,right=.85in,top=.5in,bottom=.5in, 
paperheight=12.6in,paperwidth=9.5in]{geometry}

%\usepackage[letterpaper, left=.925in,right=.925in,top=.56in,bottom=.56in, 
%paperheight=12.5in,paperwidth=9.5in]{geometry}


\usepackage{enumitem}

\setlist[description]{leftmargin=4pt}

\setlength{\footskip}{20pt}


\newcommand{\mdash}{---}
\newcommand{\q}[1]{{\fontfamily{phv}\selectfont ``}#1{\fontfamily{phv}\selectfont ''}} 
\newcommand{\sq}[1]{{\fontfamily{phv}\selectfont `}#1{\fontfamily{phv}\selectfont '}} 

\newcommand{\JaneJohn}{\i{Jane}\hspace{4pt}and\hspace{3pt}\i{John}}


\colorlet{codegr}{black!80!blue}


\let\OldFootnoteSize\footnotesize
\renewcommand{\footnotesize}{\scriptsize}

\newif\iffootnote
\let\Footnote\footnote
\renewcommand\footnote[1]{\begingroup\footnotetrue\Footnote{#1}\endgroup}

\newcommand{\AcronymText}[1]{{\iffootnote\begin{footnotesize}{\textsc{#1}}\end{footnotesize}%
\else\begin{OldFootnoteSize}{\textsc{#1}}\end{OldFootnoteSize}\fi}}

\newcommand{\visavis}{vis-\`a-vis}

\newcommand{\acronymResizeBox}[1]{#1}

\newcommand{\Cpp}{\AcronymText{C++}}

\newcommand{\MRI}{\AcronymText{MRI}}
\newcommand{\fMRI}{f\AcronymText{MRI}}

\newcommand{\CPU}{\AcronymText{CPU}}

\newcommand{\lIFC}{\AcronymText{{\large{I}}FC}}

\newcommand{\IFC}{\AcronymText{IFC}}
\newcommand{\AEC}{\AcronymText{AEC}}

\newcommand{\CAD}{\AcronymText{CAD}}

\newcommand{\ThreeD}{\AcronymText{3D}}
\newcommand{\TwoD}{\AcronymText{2D}}

\newcommand{\NLP}{\AcronymText{NLP}}

\newcommand{\AAC}{\AcronymText{AAC}}
\newcommand{\GUI}{\AcronymText{GUI}}

\newcommand{\communique}{communiqu\'e}

\newcommand{\HRI}{\AcronymText{HRI}}




\newcommand{\qquadrantcodeagainst}{\q{quadrant\_code\_against}}

\newcommand{\quadrantcodeagainstimpl}{\begin{alltt}u1 quadrant_code_against(const ty##size##s& rhs) const { return (*this - rhs).spaceship_mask().plus((*this - rhs).zeros_mask()).floor(0).times({2, 1}).inner_sum(); }\end{alltt}}

\newcommand{\xypair}{\ensuremath{x, y}}
\newcommand{\xypluss}{\ensuremath{x + s, y + s}}
\newcommand{\abpair}{\ensuremath{a, b}}


\newcommand{\sval}

\newcommand{\xyplusab}{\ensuremath{x + a, y + b}}
\newcommand{\xplusy}{\ensuremath{x + y}}

\newcommand{\inum}{\ensuremath{i}}

\newcommand{\topbarleft}{\b{top} $|$ \b{left}}

\newcommand{\bardash}{\ensuremath{|-}}
\newcommand{\dashbar}{\ensuremath{-|}}




%fRed


\renewcommand{\b}[1]{\textbf{{\small{}#1}}}


\newcommand{\qb}[1]{\q{\b{#1}}}

\newcommand{\qtt}[1]{\q{\texttt{#1}}}

\let\OldI\i
\renewcommand{\i}[1]{\textit{#1}}


\newcommand{\Schutzenberger}{Sch\"utzenberger}


\newcommand{\communique}{communiqu\'e}

\newcommand{\codetext}[1]{{\small{}\textcolor{codegr}#1}}

\newcommand{\lPACS}{\AcronymText{{\large{P}}ACS}}


\newcommand{\INSERT}{\codetext{INSERT}}
\newcommand{\UPDATE}{\codetext{UPDATE}}

\newcommand{\gid}{\b{gid}}

\newcommand{\thisSlashSelf}{\b{this}/\b{self}}



%\newcommand{\anondefin}[1]{\begin{definition}#1\end{definition}}

\usepackage{amsthm}
\theoremstyle{definition}

%\usepackage{ntheorem}
%\theorembodyfont{\upshape}
%\newtheorem{definition}{Definition}

\newtheorem*{definition*}{Definition}
\newtheorem{definition}{Definition}
\newtheorem{theorem}{Theorem}

\newtheorem{lemma}{Lemma}

\usepackage{setspace}

\usepackage{changepage}


\newtheorem*{observation*}{Observation}


\newcommand{\anonobservation}[1]{\vspace{12pt}%
\begin{observation*}
\lmargspac{5pt}{1.1}{}{}{%
\small{#1}}
\end{observation*}
%\vspace{-30pt}
}

\newcommand{\observationproof}[1]{\vspace{-10pt}%
\lmargspac{5pt}{1}{}{}{\noindent{\small{}\textit{\textbf{Proof}} #1}}}



\newcommand{\spac}[2]{{\begin{spacing}{#1}#2\end{spacing}}}

\newcommand{\lmargspac}[5]{\begin{adjustwidth}{#1}{0pt}\spac{#2}{%
\noindent\hspace*{-#1}#3\hspace{#4}#5}\end{adjustwidth}}

\newcommand{\lemmastatement}[1]{\vspace{10pt}%
\begin{lemma}%\spac{1.1}{\small{#1}}
\lmargspac{3pt}{1.1}{}{}{%
\small{#1}}
\end{lemma}
\vspace{-10pt}}

\newcommand{\theoremstatement}[1]{\begin{theorem}#1\end{theorem}}

\newcommand{\lemmaproof}[1]{\lmargspac{3pt}{1.1}{\textit{\textbf{Proof}}}{16}{%
\small{#1}}}


\newcommand{\theoremproof}[1]{\noindent\textit{\textbf{Proof}} #1}


%\newcommand{\anondefin}[1]{\begin{definition*}#1\end{definition*}}

\newcommand{\anondefin}[1]{\vspace{8pt}%
\begin{definition*}
\lmargspac{5pt}{1.1}{}{}{%
\small{#1}}
\end{definition*}
%\vspace{-30pt}
}

\newcommand{\defin}[2]{%\vspace{2pt}%
\begin{definition*}
\lmargspac{5pt}{1.1}{}{}{%
\small{\textit{#1}: #2}}
\end{definition*}
%\vspace{-30pt}
}

\newcommand{\redbluegreen}{\{\textbf{red}, \textbf{blue}, \textbf{green}\}}

\newcommand{\huesaturationvalue}{\{\textbf{hue}, \textbf{saturation}, \textbf{value}\}}


\newcommand{\Csharp}{C\#}

\newcommand{\ThreeSixty}{360\textdegree{}}

\newcommand{\INTMAX}{{\small{}\texttt{INT\_MAX}}}
\newcommand{\SHRTMAX}{{\small{}\texttt{SHRT\_MAX}}}
\newcommand{\INTMIN}{{\small{}\texttt{INT\_MIN}}}

\usepackage{enumitem}

\let\OldDescription\description

\renewenvironment{description}
  {\begin{OldDescription}[style=unboxed]}
  {\end{OldDescription}}


\newcommand{\makeboxq}[1]{\makebox{\q{#1}}}


\newcommand{\LatinOne}{\makebox{\texttt{Latin1}}}

\usepackage{array}
\usepackage[originalparameters]{ragged2e}
\newcolumntype{M}{>{\RaggedRight\setstretch{.5}}p{.2\textwidth}}

\usepackage{longtable}


%\usepackage{listings}


%\newcommand{\defin}[2]{\begin{definition}\textit{#1}: #2\end{definition}}


