

\renewcommand{\p}[1]{

\vspace{.7em}#1}

\usepackage{graphicx}
\usepackage{xcolor}

\setlength{\columnsep}{8.55mm}

\usepackage[letterpaper, left=.85in,right=.85in,top=.5in,bottom=.5in,
paperheight=13.6in,paperwidth=10.8in]{geometry}


\usepackage{enumitem}

\setlist[description]{leftmargin=4pt}

\setlength{\footskip}{20pt}


\newcommand{\mdash}{---}
\newcommand{\q}[1]{{\fontfamily{phv}\selectfont ``}#1{\fontfamily{phv}\selectfont ''}} 
\newcommand{\sq}[1]{{\fontfamily{phv}\selectfont `}#1{\fontfamily{phv}\selectfont '}} 

\newcommand{\JaneJohn}{\i{Jane}\hspace{4pt}and\hspace{3pt}\i{John}}


\colorlet{codegr}{black!80!blue}


\let\OldFootnoteSize\footnotesize
\renewcommand{\footnotesize}{\scriptsize}

\newif\iffootnote
\let\Footnote\footnote
\renewcommand\footnote[1]{\begingroup\footnotetrue\Footnote{#1}\endgroup}

\newcommand{\AcronymText}[1]{{\iffootnote\begin{footnotesize}{\textsc{#1}}\end{footnotesize}%
\else\begin{OldFootnoteSize}{\textsc{#1}}\end{OldFootnoteSize}\fi}}

\newcommand{\visavis}{vis-\`a-vis}

\newcommand{\acronymResizeBox}[1]{#1}


\newcommand{\XCSD}{\AcronymText{XCSD}}

\newcommand{\RGBthirtytwo}{\AcronymText{RGB32}}

\newcommand{\RGB}{\AcronymText{RGB}}

\newcommand{\Cpp}{\AcronymText{C++}}

\newcommand{\GUI}{\AcronymText{GUI}}
\newcommand{\HSV}{\AcronymText{HSV}}


\newcommand{\TwoD}{\AcronymText{2D}}
\newcommand{\ThreeD}{\AcronymText{3D}}



\newcommand{\CCpp}{\AcronymText{C}/\AcronymText{C++}}

\newcommand{\ASCII}{\AcronymText{ASCII}}

\newcommand{\Qt}{\AcronymText{Qt}}

\newcommand{\VM}{\AcronymText{VM}}



%\newcommand{\ttr}[1]{\hspace*{0pt}\raisebox{-1pt}{\texttt{\Large{#1}}}\hspace{0pt}}
\newcommand{\ttr}[1]{\raisebox{-.5pt}{\resizebox{7pt}{!}{{\texttt{{#1}}}}}}

\newcommand{\vx}{\ttr{v}}
\newcommand{\f}{\ttr{f}}
\newcommand{\g}{\ttr{g}}


\newcommand{\nary}{\ensuremath{n}-ary}

\newcommand{\arity}{\ensuremath{a}}

\usepackage{textcomp}

\newcommand{\arityPrime}{\ensuremath{a{\textquotesingle}}}
\newcommand{\arityprime}{\ensuremath{a{\textquotesingle}}}

\newcommand{\arityprimelessthanarity}{\ensuremath{\arityPrime{} < \arity{}}}

\newcommand{\onearityprime}{\ensuremath{1 \ldots{} \arityPrime{}}}

\newcommand{\arityprimeplusone}{\ensuremath{\arityPrime{} + 1}}

\newcommand{\nlightblue}{\textcolor{cyan!60!black}{\ensuremath{n}}}
\newcommand{\ndarkblue}{\textcolor{blue!50!black}{\ensuremath{n}}}

\newcommand{\emsm}[1]{\resizebox{!}{6pt}{\ensuremath{#1}}}



\newcommand{\fRed}{\textcolor{red!60!black}{\ttr{f}}}
\newcommand{\fOneTwoThree}{\fRed\textcolor{blue!50!black}{\emsm{123}}}
\newcommand{\fOneTwoThreeGFour}{\fOneTwoThree\textcolor{black!50}{\emsm{4}}}
\newcommand{\fOneTwoThreeGFourFive}{\fOneTwoThree\textcolor{black!50}{\emsm{45}}}

\newcommand{\enmlightblue}[1]{\textcolor{cyan!60!black}{\emsm{#1}}}
\newcommand{\enmdarkblue}[1]{\textcolor{blue!50!black}{\emsm{#1}}}


\newcommand{\fOneTwoOne}{\fRed\enmlightblue{1}\enmdarkblue{2}\enmlightblue{1}}
\newcommand{\fOneTwoTwo}{\fRed\enmdarkblue{1}\enmlightblue{2}\enmlightblue{2}}



\newcommand{\goff}{\ensuremath{g \circ{} f}}

\newcommand{\foff}{\ensuremath{f \circ{} f}}
\newcommand{\fofg}{\ensuremath{f \circ{} g}}

\newcommand{\None}{\ensuremath{N_1}}
\newcommand{\Ntwo}{\ensuremath{N_2}}

\newcommand{\fxgx}{\ensuremath{fx(gx)}}
\newcommand{\fxgy}{\ensuremath{fx(gy)}}
\newcommand{\fgxy}{\ensuremath{f(gx)y)}}
\newcommand{\fgxgx}{\ensuremath{f(gx)(gx)}}
\newcommand{\fgxgy}{\ensuremath{f(gx)(gy)}}

\newcommand{\fxy}{\ensuremath{fxy}}
\newcommand{\fyx}{\ensuremath{fyx}}

\newcommand{\fxytofyx}{\ensuremath{\fxy{} \rightarrow{} \fyx{}}}

\newcommand{\fetcxetc}{\ensuremath{f_1, ..., x_1, ...}}

\newcommand{\Cfetcxetc}{\ensuremath{\Ccomb{}f_1\cdots{}x_1\cdots{}}}


\newcommand{\combinator}[1]{\raisebox{-1pt}{\textbf{#1}}}

\newcommand{\Ccomb}{\combinator{C}}

\newcommand{\Scomb}{\combinator{S}}
\newcommand{\Bcomb}{\combinator{B}}
\newcommand{\Ycomb}{\combinator{Y}}

\newcommand{\reduces}{\makebox{\ensuremath{\rangle{}\hspace*{-2pt}}{---}}}

\newcommand{\Sfgx}{\ensuremath{\Scomb{}fgx}}
\newcommand{\fxparengx}{\ensuremath{fx(gx)}}

\newcommand{\Bfgx}{\ensuremath{\Bcomb{}fgx}}
\newcommand{\fparengx}{\ensuremath{f(gx)}}

\newcommand{\YgreducgYg}{\ensuremath{\Ycomb{}g \reduces{} g\Ycomb{}g}}
















%fRed


\renewcommand{\b}[1]{\textbf{{\small{}#1}}}


\newcommand{\qb}[1]{\q{\b{#1}}}

\newcommand{\qtt}[1]{\q{\texttt{#1}}}

\let\OldI\i
\renewcommand{\i}[1]{\textit{#1}}


\newcommand{\Schutzenberger}{Sch\"utzenberger}


\newcommand{\communique}{communiqu\'e}

\newcommand{\codetext}[1]{{\small{}\textcolor{codegr}#1}}

\newcommand{\lPACS}{\AcronymText{{\large{P}}ACS}}


\newcommand{\INSERT}{\codetext{INSERT}}
\newcommand{\UPDATE}{\codetext{UPDATE}}

\newcommand{\gid}{\b{gid}}

\newcommand{\thisSlashSelf}{\b{this}/\b{self}}



%\newcommand{\anondefin}[1]{\begin{definition}#1\end{definition}}

\usepackage{amsthm}
\theoremstyle{definition}

%\usepackage{ntheorem}
%\theorembodyfont{\upshape}
%\newtheorem{definition}{Definition}

\newtheorem*{definition*}{Definition}
\newtheorem{definition}{Definition}
\newtheorem{theorem}{Theorem}

\newtheorem{lemma}{Lemma}

\usepackage{setspace}

\usepackage{changepage}


\newtheorem*{observation*}{Observation}


\newcommand{\anonobservation}[1]{\vspace{12pt}%
\begin{observation*}
\lmargspac{5pt}{1.1}{}{}{%
\small{#1}}
\end{observation*}
%\vspace{-30pt}
}

\newcommand{\observationproof}[1]{\vspace{-10pt}%
\lmargspac{5pt}{1}{}{}{\noindent{\small{}\textit{\textbf{Proof}} #1}}}



\newcommand{\spac}[2]{{\begin{spacing}{#1}#2\end{spacing}}}

\newcommand{\lmargspac}[5]{\begin{adjustwidth}{#1}{0pt}\spac{#2}{%
\noindent\hspace*{-#1}#3\hspace{#4}#5}\end{adjustwidth}}

\newcommand{\lemmastatement}[1]{\vspace{10pt}%
\begin{lemma}%\spac{1.1}{\small{#1}}
\lmargspac{3pt}{1.1}{}{}{%
\small{#1}}
\end{lemma}
\vspace{-10pt}}

\newcommand{\theoremstatement}[1]{\begin{theorem}#1\end{theorem}}

\newcommand{\lemmaproof}[1]{\lmargspac{3pt}{1.1}{\textit{\textbf{Proof}}}{16}{%
\small{#1}}}


\newcommand{\theoremproof}[1]{\noindent\textit{\textbf{Proof}} #1}


%\newcommand{\anondefin}[1]{\begin{definition*}#1\end{definition*}}

\newcommand{\anondefin}[1]{\vspace{8pt}%
\begin{definition*}
\lmargspac{5pt}{1.1}{}{}{%
\small{#1}}
\end{definition*}
%\vspace{-30pt}
}

\newcommand{\defin}[2]{%\vspace{2pt}%
\begin{definition*}
\lmargspac{5pt}{1.1}{}{}{%
\small{\textit{#1}: #2}}
\end{definition*}
%\vspace{-30pt}
}

\newcommand{\redbluegreen}{\{\textbf{red}, \textbf{blue}, \textbf{green}\}}

\newcommand{\huesaturationvalue}{\{\textbf{hue}, \textbf{saturation}, \textbf{value}\}}


\newcommand{\Csharp}{C\#}

\newcommand{\ThreeSixty}{360\textdegree{}}

\newcommand{\INTMAX}{{\small{}\texttt{INT\_MAX}}}
\newcommand{\SHRTMAX}{{\small{}\texttt{SHRT\_MAX}}}
\newcommand{\INTMIN}{{\small{}\texttt{INT\_MIN}}}

\usepackage{enumitem}

\let\OldDescription\description

\renewenvironment{description}
  {\begin{OldDescription}[style=unboxed]}
  {\end{OldDescription}}


\newcommand{\makeboxq}[1]{\makebox{\q{#1}}}


\newcommand{\LatinOne}{\makebox{\texttt{Latin1}}}

\usepackage{array}
\usepackage[originalparameters]{ragged2e}
\newcolumntype{M}{>{\RaggedRight\setstretch{.5}}p{.2\textwidth}}

\usepackage{longtable}

\newcommand{\qXCSD}[1]{\q{\XCSD}}


\def\?{/\hspace{0pt}}

\newcommand{\labelbox}[1]{\vspace*{5pt}\parbox{0.94\textwidth}{{\fontfamily{phv}\fontsize{10}{11}\selectfont #1}}}

\usepackage{subfig}
\usepackage{graphicx}

\usepackage{microtype}

\newcommand{\biburl}[1]{ {\fontfamily{gar}\selectfont{{\scriptsize \textls*[-70]{\url{#1}}}}}}

\usepackage[colorlinks=true]{hyperref}

\colorlet{urlclr}{red!10!magenta!70!orange}

\hypersetup{
 urlcolor = urlclr!50!black,
 urlbordercolor = cyan!60!black,
 linkcolor = red!30!black,
 citecolor = orange!30!black,
 citebordercolor = yellow!30!black,
} 


\raggedbottom

%\newcommand{\defin}[2]{\begin{definition}\textit{#1}: #2\end{definition}}

\newcommand{\bibtitle}[1]{{\footnotesize{}\textit{#1}}}
\newcommand{\intitle}[1]{{\hspace{3pt}\textls*[-80]{\texttt{\textit{#1}}}}\hspace{-1pt}}


\renewcommand{\b}[1]{{\small\textbf{#1}}}

\let\OldI\i
\renewcommand{\i}[1]{\textit{#1}}



\newcommand{\rsb}[1]{\raisebox{.5pt}{{\footnotesize\textbf{#1}}}}}


\usepackage{textcomp}

\newcommand{\ThreeSixty}{360\textdegree{}}


%\usepackage{listings}


%\newcommand{\defin}[2]{\begin{definition}\textit{#1}: #2\end{definition}}


