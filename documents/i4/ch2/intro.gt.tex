\p{\:\+The previous chapter discussion Virtual Machine
architecture in a rather abstract, mathematical
sense. \> This chapter will try to make the
analysis more concrete by considering
how \VM{} technology could facilitate
software engineering in the specific
context of image-processing and Computer
Vision. \> That is, I will use image-processing
algorithms as a case-study in the sorts of
workflow modeling and quantitative operations
which might reasonably be incorporated in
\VM{} designs, albeit a \VM{} specialized for
a specific computational domain.\;\<
}

\p{\:\+More general-purpose \VM{}'s might be more
\q{minimal} in the sense of providing relatively
sparse instruction-sets, relying on higher-level
code libraries (compiling to those instructions) for
for specific domains, such as image processing.
\> However, the discussion in these chapters generally
envisions \VM{}s which are in a sense both narrower
and more expansive: encapsulating functionality
targeted at specific concerns (e.g., image analysis)
but exposing potentially a large set of functions
as primitive operations that could be expressed
within \VM{} instructions directly.\;\<
}

\p{\:\+In the case of image-processing, there are literally
hundreds of algorithms that might be run to
transform, or extract information from, a
\TwoD{} image (not to mention image-series, videos, or \ThreeD{}
point clouds) so mapping each algorithm to kernel
\VM{} instructions is of course impractical. \> Obviously,
even an image-oriented \VM{} will have to provide
some foreign-function interface where various algorithms
could be registered and accessible through \VM{} code;
the \VM{} runtime could accordingly be provided with
foreign-function pointers through the
application which hosts and initializes the \VM{}
itself. \> On the other hand, there might be some
operations related to image-processing which are
so fundamental that they should be recognized
by a domain-specific \VM{} automatically, without
needing an extra step of loading \q{foreign}
functions (providing the relevant capabilities).\;\<
}

\p{\:\+This chapter will use image-processing as a domain to
examine some of the relevant issues along these
lines. \> At the same time, extended consideration of
Computer Vision mathematics is outside the scope of the
chapter; I will attempt to orient the discussion to
quantitative functionality that is relatively
simple from an algorithmic point of view. \> In effect,
I will concentrate on image-analysis techniques which
are not necessarily representative of the most
sophisticated strategies for Computer Vision, but they
do present case-studies for the primary concerns of
the current chapter, with respect to
\VM{} design and integration with query engines and
functional-reactive event handlers.
\> For the sake of discussion, this chapter will base some examples
on a novel, rather idiosyncratic image format that
I call \q{ \XCSD{}} (the acronym stands for \q{Extensible
Channel System} and \q{Subdivision Indexing}, to be
clarified later). \> This is a format optimized for
certain image-processing operations related to
approximate segmentation and color-based (more than
shape-based) background/foreground separations. \> I will
orient the exposition around hypothetical \VM{}s for
which \XCSD{} supplies the \q{native} image format,
and which might be used with image databases
(e.g., for query evaluation) storing content in this format.\;\<
}

\p{\:\+Preliminary to discussing algorithms directly related
to images, I will review several themes arising in that
context but applicable more broadly, within the
overarching subject-matter of reasonable scope
and operation-sets for (relatively high-level) \VM{}s.
\> This initial discussion will highlight type-theoretic
issues and primitive mathematical functionality
that go beyond the ubiquitous integers, floats, and
fundamental arithmetic capabilities that one would
expect from any \VM{} (however broad or narrow in scope,
and high-level or low-level in design). \> Whether or
not these specific types and operations are appropriate
for a specific \VM{}, given its goals and motivations,
these examples will hopefully suggest how the question
of what are legitimately \q{fundamental} or \q{low-level}
types and calculations is open-ended. \> Numerous constructions
usually associated with high-level languages that compile
to \VM{}s or intermediate representations (perhaps only in a
temporary sense in the course of being compiled to
actual machine code) may prove to be worthy of
direct support and implementation at the \VM{} level
\mdash{} aside simply from convenience (insofar as certain
high-level operations do not need to be repeatedly
translated down to multiple \VM{} instructions)
the specific conditions and rationales for the
relevant high-level types can intersect with \VM{}
opsets in ways that justify accommodating them
directly. \> Again, this chapter will illustrate
such points with concrete examples. \> Subsequently,
I will situate the types and operations reviewed in the
preliminary discussion in the context of image-processing,
so that first analysis will extend into Computer Vision
as a case-study for (what one might call)
\q{domain-specific} \VM{} design.\;\<
}
