
\section{Type-Theoretic Constructions at the Virtual Machine Level}
\p{\:\+The types which lay at the foundation of almost any
computer system's type hierarchy are integer and floating
point values with different byte lengths \mdash{} often 1, 2, 4, and 8
byte integers (distinguishing signed and unsigned types)
and 4 or 8 byte floating-point numbers (decimal approximations).
\> It is less common for relatively low-level environments,
such as conventional Virtual Machines, to model \i{pairs} of
integers or decimals as \q{base} types, assuming that such values
would more likely be implemented as types in a high-level
language (e.g., \b{std::pair} in \Cpp{}). \> However, number-pairs
are such an essential component of image-processing operations
that a \VM{} specifically targeting this domain would have
reason to recognize pairs as equally fundamental to
single values; since any pixel or location in an image
needs two independent numbers to be identified, and since
image-locations are the building blocks around which
Computer Vision algorithms are constructed, such two-number
pairs are if anything more \q{foundational} than single values.\;\<
}

\p{\:\+For similar reasons, one could equally argue that several
\i{different} number-pair types are foundational.
\> One consideration for image processing is that different
Computer Vision libraries utilize different conventions
about how positions within an image are to be notated.
\> One detail which varies across platforms concerns where
location-axes have their origin point: one could have
the top-left corner correspond to the point (0, 0)
\mdash{} with axes increasingly down and to the right \mdash{}
or alternatively the origin at the bottom-right,
or even the image center. \> Another question is whether
the horizontal or vertical coordinates are notated first
in a number-pair. \> These are not complex details, of course,
but care must be taken to ensure that analytic procedures
are called with the proper location-encoding, i.e., that
there is alignment between libraries' and calling procedures
with respect to location-coordinate conventions.\;\<
}

\p{\:\+Ensuring such alignment can be promoted via strong-typing: assuming that
number-pair types are further narrowed to reflect coordinate
conventions, only locations encoded according to the proper
rules can be passed to analytic procedures. \> Such policies can
include how coordinates are names: one common vocabulary for
pairs, as in tables or spreadsheets, are to identify horizontal
\q{lines} as \i{rows}, and vertical as \i{columns}. \> Moreover,
row/column pairs usually notate the \i{row} coordinate first.
\> Therefore, constraining a pair to be specifically \i{row/column}
documents that the expected interpretation of the first coordinate
is vertical (i.e., a vertical distance from the origin) and
the second is horizontal. \> Conversely, a different popular
convention (reflecting mathematical norms) is to label
horizontal values as \b{x} and vertical as \b{y}. \> In
this case, normally the \b{x} value comes \i{first};
which illustrates possible confusion, because a number-pair
would be interpreted differently depending on whether
it is understood to be row/column (vertical then horizontal)
or x/y (horizontal then vertical).\;\<
}

\p{\:\+To add further confusion, some procedure accept inputs in terms
of \i{lengths}, e.g. width and height, where similar procedures
would accept location-coordinates. \> So, without full context
information, it is ambiguous whether a pair designates a coordinate
or two length measures. \> Consider a procedure to calculate
the average color or dominant color of some rectangular
image-region, where the inputs must define a rectangle spanning
the focus of the computation. \> That rectangle might be described
via a pair of locations \mdash{} top-left and bottom-right corners, for
instance \mdash{} or alternatively via \i{one} coordinate-pair
and one length-pair, e.g., top-left \i{corner} plus \i{width}
and \i{height}. \> In other words, the second number-pair
might be a \i{location} (x/y or row/column) or might be
\i{dimensions} (width/height). \> Confusion can be avoided
by having a width/height type type distinct from
location-types, analogous to distinguishing row/column
from x/y location coordinates. \> At the same time, note
also that some procedures might take number-pairs
which do not model geometric quantities such as position
or length, where terms like \q{width} or \q{column} could
be misleading \mdash{} in the generic \Cpp{} \b{std::pair},
for example, the first value is called \q{first} and the
second \q{second}. \> For such \q{non-geometric} pairs
one might indeed prefer still another pair-type
with coordinates labeled as first/second, say.
\> Indeed, code for the \XCSD{} format (for example)
supports numerous variations on pair-types,
including options based on start/end, top/bottom,
left/right, horizontal/vertical, and above/below
(e.g., if a pair's \Cpp{} type is, say,
\b{tl2} \mdash{} meaning top/left, two-byte values \mdash{}
the first coordinate would be accessed by a \b{.top}
member and the second by \b{.left}).\;\<
}

\p{\:\+Even were a virtual machine to natively model number-pairs,
theorists preferring relatively minimalist \VM{} design
might think to provide one such pair-type (or perhaps
several distinguished only by byte-length), anticipating
that higher-level distinctions between conceptually
distinct but structurally isomorphic types
(like row/column, x/y, width/height, and first/second).
\> Ensuring that strong pair-types are never mismatched
would then be the responsibility of the compiler emitting
\VM{} code, not the \VM{} runtime. \> Consider, however, that
pair-labeling could reasonably be an issue within
special contexts, such as query-evaluation: an image-database
could, for example, allow for queries which
invoke the rectangle-focused procedure mentioned
above (along the lines of \q{find images whose central third
rectangle color-averages} to nearly some given shade).
\> Assuming type-level distinctions such as x/y vs. width/height
are recognized by the query \i{language}, then query
\i{evaluators} would of course at some stage ensure that
an x/y coordinate is not passed to a rectangle \q{constructor}
which expects width/height dimensions, say. \> Moreover,
the \q{compilers} that would bridge \i{query} code to
\VM{} instructions might be significantly different than
those working on code from other genres, such as scripts
(with notions of types, scopes, and variables closer
in spirit to compiled programming languages). \> These
circumstances could argue for stronger type distinctions
being recognized even at the \VM{} level, at least to
some degree of detail, to help guarantee that
type-enforced conventions are applied in similar
ways across all environments that user the \VM{},
notwithstanding their distinct compilation
models (database queries compared with scripts,
for instance).\;\<
}

\p{\:\+From this perspective, then, assume that a \VM{} is
indeed designed to model \i{several} number-pair types.
\> This decision then immediately has implications
for the \i{operations} supported through
the \VM{}, because some operations seem
intrinsic to managing paired quantities; for
instance, transpose (switching the first and
second values) and arithmetic operations
(which might involve a single value \mdash{} altering
both numbers by the same amount \mdash{} or a second
number-pair, applying the relevant operation between
the first and seconds pairwise). \> Here is a more complete
list of some operations which might be used often
when working with number-pairs, and would be candidates
for native encoding in a \VM{} context:

\begin{description}
\item[(strictly) Ascending, descending, spaceship]  As boolean operators,
return true if the second value is larger (respectively, smaller)
than (or equal to, for the non-strictly versions) the first. \> The
\q{spaceship} version (borrowing the terms from many
high-level programming languages) would return 1, 0 or -1
depending on whether the second value is greater, equal, or less than the first.

\item[Zeros mask, nonzeros mask, nonnegative mask, spaceship mask]  Return pairs
whose components are one or zero depending on whether
the two input pair-values are zero (respectively, are not zero; and, are at least zero).
\> Similarly, \q{spaceship mask} is a plausible name for a function
that applies a unary spaceship operator to each coordinate in a
pair: yielding a new pair, each member either 1, -1, or 0, depending
on whether the respective number in the original pair is positive, negative, or zero.

\item[Area, ratio]  Product of the two numbers (or perhaps
absolute value of the product); respectively,
ratio of the second to the first (or perhaps a version which
transposes descending pairs before taking the ratio, so the
result is always at least one, or vice-versa, so the result is at \i{most} one).

\item[Binary merge]  Combines the two numbers into a single
value occupying different bit-positions: as a unary function, takes
an argument indicating how many bits to shift the first number,
before adding (or performing a binary \b{or}) with the second.

\item[Maximum, minimum, difference, positive-integer
difference, gap]  Greater or lesser of the
two numbers; or, return the second subtracted from the first
(the \q{positive integer} variant would result always in a non-negative
output, mapping negative differences to zero) ,
or \q{gap} between the two numbers as an absolute value (i.e.,
maximum minus minimum).

\item[Duplicate, \q{end at}]  Given one number, constructs a
pair by setting the second value with a desired gap greater
than the first (or, allowing this parameter to default to
zero, form a pair with two equal numbers); alternatively,
modify the second number in an already-existing pair to
conform to a gap.

\item[Linear intersections]  Given two pairs,
form a new pair which takes one value from each pair,
analogous to the \dashbar{} or \bardash{} operators
in diagram languages such as \b{tikz} (in effect, define a
third point as the middle vertex of a right-triangle
whose hypotenuse is the given two points, thus the
intersection of two lines, one horizontal and one vertical,
including the two input points;
there can be two versions for such a function,
corresponding to the two extra corners of a
rectangle if the given two points are one diagonal,
or equivalently which input has the horizontal
line and which the vertical).
\> As a variant form, take one pair and then two
other pairs (defining a line) and calculate the
nearest distance of the former to the latter.

\item[As arithmetic base]  Treat the two numbers
as components of a two-digit number in some base
(provided as a parameter): multiply the first
number by that input, and add the second. \> This
operation can be extended to value-triples
recognized as distinct data types analogous to pairs,
where the first first number would be multiplied
by the square of the input.

\item[Floor, ceiling, fit in range]  Modifies a pair's components
so they are at least (respectively, at most) some value.
\> Similarly, given two pairs, modify the
first pair (as needed) to satisfy range-restrictions
expressed by the second pair. \> For example, alter
each number (of the first pair) so that it is no
less than the first coordinate of the second pair,
and no greater than its second coordinate.

\item[Binary insert]  Treating the first
number as a bitset, encode the second number
in a designated bit-sequence within it
(as a unary operator, simply shift the second
number left by some count of bits, applying
binary or of the result to the first; as a
two-argument function, the second input
could specify an \i{end} bit, so in effect
the second number would be truncated of rightmost
bits as needed and then copied into the
first using a bitmask-guarded \b{or} operation
which only alters bits in the specified range.\;\<

\end{description}
}

\p{\:\+Most of the aforementioned operations are used in the
demo code associated with this chapter
(largely for calculations related to the
\XCSD{} image format), so this code can
furnish examples of these operations
employed in concrete code. \> One benefit of
providing a relatively thorough suite of
operators defined for number-pairs is
encouraging programmers to think of
algorithms in terms of operation-sequences
on number pairs as integral units, not as
sequences applied to the two pair-elements
in isolation. \> Usually pairs designate some
logically connected entity (e.g., a geometric
point) and operations have an interpretation
where it is conceptually accurate to consider
the pair as one value in a two-coordinate
space, rather than two separate numbers.
\> As an example, \XCSD{} uses a function
called \qquadrantcodeagainst{} which
is defined (within a \Cpp{} macro) as a
series of pair-operators: \quadrantcodeagainstimpl{}
That code may not be more readable than an alternative
which works with each coordinate separately,
but it fits within a paradigm where we are
encouraged to think of pairs as self-contained
mathematical entities wherever possible.\;\<
}

\p{\:\+Although the names chosen for some of these operations
might make more sense in some types' contexts
than other (\q{area} as a label for the inner-product
absolute value fits width/height better than
row/column, for example), it seems reasonable
to provide operators for all versions of structurally
similar types absent compelling reasons to make
specific exceptions. \> Moreover, distinctions
such as row/column versus width/height
(and x/y, first/second, top/left, etc.) are orthogonal
to signed/unsigned, integer/float, and byte lengths for the components,
so we can derive a fairly large number of \q{primitive}
pair-types, reflecting different combinations of
coordinate names, signed/unsigned, 1/2/4/8 bytes, and
some floating-point-based types (albeit some operations
applicable to integers, such as ones based on bitsets
or binary encodings, would not carry over).
\> On the principle that most operations available
for one pair-type should be available for others, this
results in quite a few duplicate operations differing
only by input types. \> Assuming that high-level coding
techniques such as templates are not available for
\VM{} implementations, such a scenario seems to
demand brute-force coding of multiple function-versions
(perhaps aided by code generators). \> It is not
difficult to maintain a function-pointer table with
hundreds of entries if needed, so supporting this
diversity of \q{primitive} operations does not
impose a major overhead for a \VM{} runtime.
\> However, once code-generators (and perhaps more
complex grammars) are involved, the development
environment for \i{implementing} a \VM{} runtime
would become relatively more complex, and would
need to be set up properly.\;\<
}

\p{\:\+Since we are not constrained by existing \VM{} or intermediate representations,
there's room for creativity in how opsets are designed,
in terms of their textual labels (or even their numeric opcodes)
frmo the point of view of people reading \q{virtual assembly} code
for the \VM{}, as well as runtime engineering
(opcodes might have bit-flags carrying information about
their respective operators, for example), with the
idea of organizing groups of operations differing only
by (e.g.) input/return types. \> Likewise, a well-designed
\VM{} infrastructure can allow designers of high-level languages
targeting the \VM{} some creativity as well.
\> The pair-operator code for \XCSD{} mentioned above
(which expands via macros to \Cpp{}, not a \VM{}, but
could migrate to other contexts) is constrained by
\Cpp{} syntax and operator-overloading, but other
\VM{}-specific languages would be free to, for instance,
create new binary operators for pairs. \> I alluded
above to one example: the \b{tikz} \dashbar{} and \bardash{}
could be used as syntactic atoms in a high-level language,
compiling to line-intersection functions (analogous
to the \Cppspaceship{} spaceship recently being
added to the \Cpp{} operator set, in current specifications for
that language). \> Similar comments could
be made for other operators which seem common enough
that they might warrant operator-like syntax (for readability)
rather than function-call notation \mdash{} examples might be min/max,
absolute value (applying the \abs{} operator to both components),
floor/ceiling, ranges, and so forth.\;\<
}

\p{\:\+Also, a comparable
custom-syntax scenario might apply to conventional
arithmetic (and boolean/binary) operators, where
it is necessary to distinguish an operation between a
\i{pair} and a \i{single} value (e.g., multiplying both
coordinates by the same scale) or another pair
(e.g., vector multiplication\footnote{In the
sense of Hadamard product; cf. Julia's
\b{.*} operator.}), or \i{internally} within one pair.
\> For instance, there are three important
versions of, say, addition in a number-pair context:
given \xypair{} there's \xypluss{} for some scalar
\sval{}, or \xyplusab{} for pair \abpair{},
or \xplusy{} (the \q{inner} sum). \> These
alternatives would similarly exist for
almost all \q{primitive} numeric operators
for most coding languages, which would include
arithmetic functions and also boolean
calculations both in the true-false sense
(e.g., treating positive or non-zero values
as \b{true} and zeros as \b{false}) and bitwise
manipulation, plus binary operations such as
left- and right-shift. \> A high-level language
might notate these variations via compound
tokens combining one part or character
representing the operation itself and
another character (or characters)
indicating a specific variant
(as a hypothetical example, a bare \b{+} might
represent addition \i{between} pairs, while
\b{.+} represents \q{inner sum} and \b{.+}
encodes addition with a duplicated scalar).\;\<
}

\p{\:\+Apart from overloading functions for many structurally
similar types, there are further dimensions to the
overlap between \VM{} operations, types, and code-management,
some of which are indirectly posed by the previous
examples. \> Consider the process of casting a pair
from signed to unsigned. \> Plausibly, this might
be achieved in several ways: first, take absolute values;
second, map negatives to zero (essentially using a \q{floor}
operator with parameter zero); or, third, simply
reinterpret any negative value as the binary
encoding of a positive value (so -1, say, would become
the maximum integer of the corresponding unsigned type).
\> Each of these could potentially be expressed as
operators between pairs as well as type-cast logic;
as such, one consideration when formulating a
\VM{} underlying type system is to define operators
which provide the calculations for type-casts
(in case someone wants to achieve type-cast-like
effects via a function-call, or perhaps to
enable scenarios where the \VM{} could be configured
to perform type casts in different ways by specifying
a desired operator). \> Mire generally, it's worth
noting how type-casts intersect with operator-application:
consider an \q{inner difference} function which
subtracts the second pair-value from a first.
\> Assuming we sustain the principle that unsigned
pairs should by default perform operations yielding
other unsigned pairs/values, we might want the
\i{non-negative} version of inner-difference,
which in turn implies casting a signed result
to unsigned. \> This is analogous to
a \q{nonnegative-inner-difference} procedure
which applies subtraction but with a floor
of zero \mdash{} however, such a variation
might \i{also} be useful in a signed
(including a floating-point) environment,
or as the nullary version of a unary operator
which takes an inner-difference raised
(as needed) to a floor which might be
something \i{other} than zero, passed as a parameter.\;\<
}

\p{\:\+Similar ideas apply in the context of addition with
possible overflow: should be consider addition
involving one-byte integers to be capped at 255, or
to \q{wrap around} past zero? \> The latter option
does not make much sense in the case of colors, say
(it would almost certainly be an error
if an algorithm considered
\q{pure black} zero to be one unit greater than \q{pure white}
255), and moreover could be a notorious source of
infinite-recursion bugs (cf. tests for \lteq{255},
which might never revert to false, depending on the
type involved). \> Types related to colors
(in conventional one-byte-per-component encodings)
are good examples of types for which basic arithmetic
operations should almost always be guarded against
overflow (addition maxed at 255 without circling
around to zero, and subtraction floored at zero
without negative-overflow back to 255). \> Conversely,
in some contexts integer-maximum wraparound effects
are exploited deliberately; analogously,
casting an integer to a single byte
(typically in the context of merging multiple
values via bitmasks) performs a function
analogous to zeroing out bits with index
9 and greater.\;\<
}

\subsection{Issues with overflow/underflow and loop termination}
\p{\:\+The overarching theme here reflects different conceptual
roles played by numeric types. \> In the case of one-byte
integers, for example, in some contexts the
full 0-255 range represents meaningful values (e.g.,
color), whereas elsewhere only a subset of this
range is actually used. \> A ready example is enumerations,
which might use \b{unsigned char}s as the
base or \q{carrier} type but only define, say,
ten values. \> Most instances of the underlying
type therefore do not match any enumerated value;
the \b{char} (i.e., 8-bit) type is used simply
because most systems do not have types smaller
than one byte (one cannot take the memory address
of \q{half a byte}, for instance). \> At the
same time, the numeric details of how enumerations
are encoded can sometimes become consequential
even for application-level code. \> Consider
the relatively common situation where an
\b{enum} (together with some other data) are
encoded into a single (4-byte, say) integer where
the enum is assigned a specific bit-range (if there
are ten enumerated labels, say, the bitmask must
allow four bits for the enum). \> A programmer would
immediately know to left-shift the \b{enum}s
value (cast to an integer as needed) by the
requisite number of bits, but could easily
forget that the left-shift applied to
\b{char} could result in \i{another} \b{char} and
therefore truncate bits on the left; the
\b{enum} instead must be cast to \i{int} (or whatever
type will encoded to shifted bits) even if the original
base type is, say \b{unsigned char}.\;\<
}

\p{\:\+Similarly, consider using one-byte integers (say)
in contexts where the underlying mathematics reflects
modular arithmetic. \> An example is encoding directions
(such as for labeling adjacency steps for inter-pixel
color comparisons): a common option is to encode
the eight options from top-left, top, top-right,
etc., through bottom-right, via integers 0-8.
\> On the other hand, these directions can also be
derived via coordinate subtraction, spanning the
eight pairs (-1, -1) through (1, 1) (excluding
(0, 0)), so procedures may need to interconvert
the pair-based and mod-8-based encodings. \> Still another
possibility is to label directions such as
\q{northwest} or \q{top-left} and so forth via a
distinct enumeration class, where the enumeration
labels would be mapped to corresponding numbers
in the 0-8 ramge, but descriptive terms such as \q{left}
might be more readable than numeric codes. \> A variation
on this example would be that of notating directions between
image-locations on a larger scale via 16 direction
possibilities, incorporating those between strict
orthogonal, horizontal, and vertical. \> Actually,
the \XCSD{} code (mentioned earlier) in some
contexts recognizes 24 different directions (to accommodate
variations in even/odd pixel or region counts, where the center
in an image-partition might correspond to a line of
regions or to the gap between two regions). \> Such direction-codes
are used in several calculations (mostly related to the
proximity of regions to an image center) where arithmetic
operations are performed on the underling numeric
value (e.g., mapping a direction to its nearest strict diagonal
counter-clockwise). \> Insofar as numeric codes represent
directions, then \q{rotations} (via modular addition or subtraction)
and modular wraparounds become mathematically significant.\;\<
}

\p{\:\+Modular integer-ranges can potentially be supported directly
via a strong type system allowing programmers to explicitly
define a (distinct) type for integers modulo 24, say.
\> Alternatively, one might adopt something like \q{modular guarded}
arithmetic (and binary, e.g. shift) operations, where each
operation can be given an extra parameter to serve as a
modular-arithmetic context (not that equally relevant
for image-analysis and other geometric contexts might be
variations on modular arithmetic where zero is
sometimes replaced by the modulus, or results
are \i{subtracted from} the modulus). \> Some of these
arithmetic details could just be encoded relatively
inelegantly by if-then branches and the like
(\q{if(result > 24) ...} and so forth) but we have
more flexibility on the \VM{} side to identify
various forms of modular guards (perhaps via
runtime flags that can be attached to ordinary
integers) and of modular arithmetic in general
(e.g., zero-to-modulus swaps).\;\<
}

\p{\:\+The main point here is to identify coder's rationales
for employing specific types and to recognize situations
where algorithms need to satisfy conditions more
precise than the types alone express (an integer
encodes mod-24 values, and enumeration is mapped
to numeric quantities for geometric computations,
such as mapping named directions under rotations/reflections,
and so forth). \> Whereas these conceptual details
might be merely implicit (or consigned to
comments) in high-level source code, \VM{}s can
treat such conditions more rigorously by
annotating types, or flagging runtime values,
or providing variants on common operations
(e.g., a modular variant on arithmetic functions).
\> Consider again integer-overflow situations: if a
programmer is looping \i{up to} an integer maximum
(say, 255) they might select a type for the
looping variable to \i{avoid} overflows; so
if a number \inum{}, say, is to represent
one color-component and be used for a loop,
the algorithm might declare \inum{} as a
\i{short} (or a \i{signed short} to avoid
negative wraparound) since 256 (and -1)
then become expressible values,
preserving the loop-termination. \> At this
point however we have a \i{short} encoding a
color-component whose only \i{conceptually}
meaningful values are in the \b{char} ranges;
if the \inum{} were used in a color-related
computation it would need to be cast
to its conceptually accurate \b{unsigned char}
(aside from an extra coding step, consider
how this could trip up code-review or compiler-warning
sensitive processes). \> A more fluent solution might
be to keep \inum{} itself as one-byte but cast
it to \b{short} for the loop-test. \> This is an
example of where a \VM{} could provide extra functionality
beyond the reach of compilers targeting smaller-scoped
assembly languages: consider a variation on inequality
operators which would combine a (temporary) type-cast
and boolean test into a single instruction. \> Having that
form of test available in a \VM{} opset might
inspire language designers to support it with
distinct syntax, so that a pattern such as
\q{cast-then-compare} \mdash{} which is more conceptually
precise than using artificially larger types to avoid
overflow/underflow conditions \mdash{} is directly supported
by high-level code, potentially making it clearer
and more widely adopted.\;\<
}

\p{\:\+In general, \VM{} engineers have opportunities to identify
computationally similar but conceptually distinct
operations. \> Consider a basic increment (\b{++})
operation: this may occur in a context where the range of
possible loop end-values (e.g., the size of a list) is
well below the maximum expressible integer of a relevant
type; but it may also occur (as in color-components)
where we may be looping up to that maximum
value, which itself serves as the loop-terminator
(this in turn is the kind of situation where
overflow errors due to \q{perpetual-truth} can occur \mdash{}
comparisons of \b{unsigned char} against 256, say,
never fail). \> The former use of the increment
occurs in a context where there is an \i{a priori} range
setting outliers on the loop (from the first to the last
index in a list, say), whereas in the second case
the type itself (e.g., all possible values for one
color-component, in a format like \RGBthirtytwo{})
sets a range-span. \> One can argue that the two
uses of increment (and analogously decrement,
especially in the context of unsigned types
when looping down to zero)
are conceptually different enough to warrant to two
different \VM{} instructions, especially considering
that extra care is appropriate for the second case
(due to wraparound bugs).\;\<
}

\p{\:\+Similar attentiveness to operations' \q{conceptual}
significance applies to functions on number-pairs
as discussed above, especially in contexts where
pairs represent logically integral values (such as
locations within an image). \> When operations have a
geometric interpretation (not just arithmetic)
it is possible that they may be naturally
presented in several different versions for
different contexts, in the same way
that modular arithmetic has multiple interpretations
(including geometric ones related to directions and/or
rotations). \> A \VM{} instruction set might
therefore recognize these distinctions,
providing a more flexible compilation-target
and perhaps cueing higher-level language engineers
to similarly acknowledge the relevant
distinctions (e.g., via specialized syntax).
\> Moreover, operation-variants and the type which
they work on are conceptually interrelated,
so engineers at both higher and lower levels
should be attentive to type-system concerns
interrelated with conceptually-distinguished
low-level operations. \> This chapter thus far
has presented examples in contexts
such as modulus, integer overflow, and number-pair
types; another set of examples derives
from enumerations and their encoding strategies,
which I address next.\;\<
}

\subsection{Different variations on enumeration types}
\p{\:\+In the simplest sense, enumeration types
represent sets of named values, which may have
numeric codes (for the sake of storing instances
in memory) but whose numeric representation has
no particular significance. \> This setup, however
\mdash{} where enums' actual encoding carries no semantic
weight \mdash{} is only accurate in some contexts;
elsewhere, there are various possibilities for
enums bound more tightly to their numeric basis.
\> For example, nominals representing days of the week,
or months of the year, have a fixed sequence,
which in turn is reflected in their encoding.
\> These are examples of types where both named
labels and a limited-range, modular-arithmetic
representation have conceptual merit
(consider the problem of \q{adding} some number
of days or months, or iterating around a week, or year,
cycle). \> Or, consider a collection of named color-values:
each of these would be identified by a label, but may
also be mapped to a numerically-encoded color space,
so that quantitative inter-color operations could be
implemented (measuring the distance between two
named colors, for example, or mapping them to grayscale).
\> Another case of consequential numeric encoding
which can be illustrated in the color-values would be
using enumerators as array indices: consider enums which
labels the components of color-encodings according to
different color spaces, such as \RGB{} (the enum labels
would obviously be something like \redbluegreen{})
or \HSV{} (hue, saturation, value). \> If colors
(for these three-part models) are represented with
three-valued arrays, then the enumerated component
labels should map to 0, 1, or 2 so as to form array indices.\;\<
}

\p{\:\+Note that one use-case for enums is simply to introduce a
collection of related labels into a programming environment,
without necessarily implying that there is a single
semantic or conceptual iterations which covers them
all. \> For instance, it is plausible that an enum
employed in the context of managing calendars (not just
raw dates but schedules, appointments, and so on)
would have labels for days (Sunday-Saturday) but
\i{also}, say, \i{weekday} and \i{weekend}
(in contexts such as \q{open weekdays} or \q{every weekend}).
\> In this case, the numeric encoding for days-of-the-week
would operate on two levels: most labels would be
related sequentially (Sunday, Monday, etc.) but
\q{weekday} and \q{weekend} would be grouping-labels
instead, signifying collections (Monday to Friday and
Saturday/Sunday respectively). \> A plausible implementation
for such a type would be via bitmasks: assign the numbers
1-64 (increasing over powers of two) to the seven
weekdays (i.e., each code has exactly one bit set,
in positions 1-7) and then assign
the codes 31 (the bitmask of all five weekdays, assuming
the lowest value is \q{Monday}) and
96 (the bitwise combination of Saturday and Sunday)
to \q{weekday} and \q{weekend}, respectively.
\> Presumably, this kind of encoding would be
most natural in a contexts where bitmasks could
be employed to represent various combination
of days (e.g., coding \q{Monday and Wednesday} as the
bitwise merger of the Monday and Wednesday codes).
\> Of course, this is only feasible in contexts
where enumerations can be cast to integers
(whether explicitly or implicitly).\;\<
}

\p{\:\+Mainstream programming languages evince different patterns
with respect to mapping enums to numeric values.
\> In \Cpp{}, for example, we can specify the
\i{byte length} of an \b{enum} type by requesting a
particular underlying type (e.g., \b{enum class e : unsigned char}
forces each enum value to be represented by only 8 bits) but
enums (at least \q{strong} \b{enum class} ones) do not
automatically cast to integers. \> They can, however, be manually
cast; or, integer-style arithmetic semantics can be
added \i{to} enums via operator overloading. \> For other languages
(Java, for instance) utilizing enums' numeric values is more
difficult, whereas elsewhere (e.g., \Csharp{}) conversion to
integers is automatic. \> This variation between languages
reflects inconsistency across underlying semantics:
enumerated nominals have different degrees of conceptual
connection to an underlying arithmetic space in different contexts.
\> There are, indeed, at least six possible options (I'm proposing
non-standard terms for the final three items
here, but hopefully the exposition will furnish
rationales for the vocabulary chosen):

\begin{description}
\item[Raw nominals]  Collections of named labels (\q{nominals} in the
statistics sense) which may be
assigned numeric codes simply for purpose of computational representation,
but where the actual numbers have no special meaning
(an exception could be allowed for a label used as a \q{default}
or fallback when a specific value is missing, which may be
coded with a zero to coincide with defaults in other settings).

\item[Cyclical or sequential enumerators]  This would cover cases
where labels correspond to entities which have an implicit
order (\q{ordinals} in statistics) so we can speak of one label
coming \q{before} or \q{after} another in the sequence as
conventionally understood. \> In some contexts we can also
recognize \q{differences} (as in, Friday is four days past Monday)
and wraparound (after Sunday we return to Monday, for instance),
which in turn implies modular arithmetic.

\item[Enumerators as \i{de facto} numeric constants]  A different
use-case applies to labels which have a specific numeric code
because they are intended to give a recognizable name to an actual numeric value.
\> Of course, in some cases important numeric values which
can be assigned descriptive names are independently introduced as
constants, which does not involve a specific enumerator list
\mdash{} for instance, \CCpp{} has macros such as
\SHRTMAX{}, \INTMAX{}, \INTMIN{}, and the like which
define maximum (respectively, minimum) values for different
integer types; these are global constants which have some
obvious logical interrelationship, but there does not appear
to be any benefit in grouping together as a distinct \i{type}
whose specific purpose is to identify just these values.
\> On the other hand, consider a situation where we want to
create a list of Latin1 codes for characters which have
specific meanings in some formal context (e.g., the
list of whitespace glyphs: space, tab, new line, form-feed,
carriage-return). \> Here it is plausible that a procedure
might accept a type which expresses a member of such a
list (implementing actions for a code-generating grammar, say)
while also each label needs to embody a specific constant
value (we want to enumerator to translate directly to
a corresponding \ASCII{} and Unicode number), which in
a sense combines the sorts of contexts where global
constants would be preferred and those more
amenable to proper enums; if indeed an enum is chosen
to represent the constants as a \i{type},
the numeric values would of course have semantic
significance (on the other hand, combinations
of such values, e.g. via bitmasks, would \i{not}
in general have meaningful semantics, so
arithmetic operations between two labels
might be suppressed in these cases.

\item[Isolated bitmask enumerators]  Names which do not
have intrinsic mathematical interpretations but which are
assigned numeric codes built around powers of two,
for the sake of notating combinations of such nominals via a single
number. \> This kind of encoding is common for labels which
represent \q{flags}, with the possibility of merging multiple
flag-values or \q{activating} multiple flags at once. \> For a
geometric-flavored example, suppose we want to specify alignment
options relative to a bounding rectangle: the actual
labels recognized by the enum might be \i{left}, \i{top}, \i{right},
\i{bottom}, and \i{center}, but these labels could be
used in combinations such as \q{top-left}, using the bitwise-or
merger of codes for \i{top} and \i{left} to indicate alignment
according to both of these directions. \> I call enums in this
case \q{isolated} to suggest that \i{within the label-list itself}
each nominal has a numeric code with no intrinsic meaning; it is
only for creating \i{numeric} representations of
combinations (like \topbarleft{} as a mathematical construction)
that encodings' binary structure become significant.

\item[\q{Fusional} bitmask enumerators]  I propose
this terminology to express cases where some enumeration
labels represent combinations of other labels. \> The difference
between this case and the prior one is that here combinations
are not only numeric \i{expressions} which can be interpreted as
simultaneously declaring multiple nominals at once,
but instead labels which are part of the enumerative collection
itself can be assigned mathematical values which translate
to sets of other labels. \> The weekday/weekend case would
be one example, given a bitmask encoding such as proposed
in item (2); or, suppose for item (3) we extend the \q{alignment}
example such that \b{TopLeft} and other combinations are
actual parts of the \b{enum}. \> Types in these scenarios
have the ability to form labels for commonly-used collections
of other nominals; consider, say, the type \b{Qt::TextInteractionFlag}
(part of the \Qt{} application-development framework) which defines
settings for how a \GUI{} control displaying natural-language
texts allows for user actions (selecting, copying, following
hypertext links, and so forth). \> In addition to
several independent/isolated labels there are two
bitmasks which are part of enumerator list: \b{TextBrowserInteraction}
(which defines the most common fusion of interactions
appropriate for controls playing the role of text browsers) and
\b{TextEditorInteraction} (which, similarly, sets the defaults
for text editors \mdash{} which differ from browsers in that text may
be modified within the control, while it is being displayed).
\> Of course, other flag-combination may also be used; but
providing labels for the \i{most common} combinations
helps to clarify the specific conceptual status of
those fusions specifically.

\item[Free-form or arithmetized enumerators]  This
case would cover situations where numeric values associated
with enum labels are significant, but for multiple or
intersecting reasons not covered by the above items.
\> Relevant examples can include cases where mathematical
functions are applied to enum-values in some
algorithmic contexts. \> Earlier I mentioned
a 24-valued \q{direction} code employed internally
by \XCSD{}; were these codes to be expressed via
an enum, they would need a specific mapping to
integers so as feed the enums directly into
certain \XCSD{}-specific algorithms (related, for
example, to the process of ordering image-regions
based on proximity to the center and, secondarily,
by clockwise progression \i{around} the center).
\> For technical reasons (the details are not
especially relevant here) these directions
are coded by \i{signed} values in the range
-1 through 22; there are specific computational
reasons for this range in particular, indicating
that the numeric codes are intrinsically
associated with corresponding enum labels
(it's not a matter of assigning numbers essentially
at random, or merely in an increasing sequence to
honor ordering and/or cyclicality). \> Other \q{free form}
examples would come from situations where codes
may be assigned with an eye to bitwise fusions,
but not all labels fit that binary pattern.
\> The above \q{alignment} examples can have variations
which fit this last case; in \Qt{}, for example,
the \b{Qt::AlignmentFlag} enum gathers into one
type labels for vertical (\b{AlignTop}, \b{AlignBottom},
\b{AlignVCenter}, and \b{AlignBaseline}) and
horizontal (\b{AlignLeft}, \b{AlignRight},
\b{AlignHCenter}, \b{AlignJustify}); and also
for \i{one} named combination (\b{AlignCenter},
which fuses \b{AlignVCenter} and \b{AlignHCenter}).
\> In addition to these labels, the list includes
a special flag which is only used for text in
natural languages with a right-to-left writing system,
and \q{mask} labels for convenience \mdash{} specifically,
a mask which extracts the \i{horizontal} and another
for the \i{vertical} component of an alignment
code; those masks would be used mathematically
to split a single code into two labels (indicating
that the labels need specific numeric values to
make the binary operations work properly).\;\<
\end{description}
}

\p{\:\+As the \Qt{} alignment example shows, sometimes
enumeration types have labels serving
different conceptual roles: the \Qt{} type
essentially combines horizontal and vertical
options, plus a special supplemental flag, and a
bitmask for convenience, into a single label-set.
\> Proper numeric encoding is needed to orchestra the
interplay of concepts and values involved here.
\> More to the point, this example shows that
expressing multiple concepts through a single
label-collection increases the likelihood
that numeric values will be significant.
\> The \Qt{}.\;\<
}

\p{\:\+The various alternatives outlined here reflect a duality
in the \q{conceptual} role of enums: in effect,
enum types fulfill two distinct purposes, namely
the use of mnemonic or descriptive \i{labels}
(instead of raw numbers), on the one hand, and
(on the other) isolating a specific range or group
of numbers as a distinct \i{type}. \> These two roles
can be combined in different ways for different
enumerations, which in turn raises questions about the
proper type-theoretic protocols appropriate
for enums. \> For example, how should procedures
whose signatures indicate parameters of an
enum type work with inputs whose
numeric values do not match any nominal
label? \> Should compilers reject
code where there is no algorithm to prove that a
certain input conforms to the specific list
of values associated with enum labels? \> Or should
it allow values that can be formed from
those matched to labels via boolean \b{or} operations
(merging their respective bits) but reject others?
\> Or, perhaps, map anomalous values to one \q{default} label?
\> Moreover, we can distinguish cases where a compiler
can verify that a value will \i{fail} via either
of those constraints, as opposed to cases where
the details are uncertain \mdash{} e.g., the enum
may be obtained via serialization, or some
other external source which may (or may not)
endeavor to restrict values to \q{meaningful}
options. \> These issues are also reflected
by how enums are used by procedures which
take them: if they are fed to \b{switch} statements,
for example, then a \b{default} execution path
could catch any values not matching named labels.\;\<
}

\p{\:\+In short, the enumeration mechanism
is actually a combination of multiple semantic
patterns, and these differences tend to get
reflected at the point where procedures
take inputs of enum types (rather than
their underlying numeric types).
\> In the \VM{} context, procedures could
therefore indicate what \q{flavor} or
enum type they expect through some sort
of flag or annotation; instead of a single
enum protocol, compilers (or a \VM{} runtime)
could then fine-tune enum handling based
on procedures' specifications. \> For example,
a procedure could indicate that it should
\i{only} be called with an integer
value that matches an enum value, or
that (as a runtime feature) anomalous
values get mapped to one specific label.
\> Moreover, enum \i{types} could be equipped with
different varieties of integer-casts, distinguished
between contexts in which coversions would be
applied. \> In the most restrictive case, enums
would never be cast to their underlying
values except for a narrow set of low-level
operations (such as serialization via binary packs);
in the least restrictive, enums could be freely
utilized as quantitative magnitudes, subject to
a full suite of arithmetic, binary, and boolean
operations. \> In between those extremes, enum
types could be annotated to accept casts
in certain circumstances but not others,
or to support a proper subset of mathematical functions.\;\<
}

\p{\:\+This discussion has set forth some of the
conceptual issues associated with enumerations,
and before that with number-pair types.
\> These examples serve primarily as case-studies;
ultimately, I intend to ground the analysis
in \VM{} engineering in particular.
\> As such, I will transition to a more
\VM{}-oriented (and, indeed, image-processing oriented)
focus in the following section.\;\<
}
