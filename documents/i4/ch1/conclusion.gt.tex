\section{Conclusion}

\p{\:\+My discussion at the end of last section centered
on \GUI{} design/implementation, and \mdash{} given
this books overarching industrial and \AI{} themes
\mdash{} it seems appropriate to conclude the chapter
by pointing out the overlap between the former
and latter subject-areas. \> In terms of \q{Industry 4.0},
one plausible anticipation \mdash{}reprising also
the trends I mentioned last paragraph \mdash{} is an
increasing emphasis on \i{multi-model} User
Experience. \> People will likely interact
with software in an expanding variety of
ways, such as voice and touch alongside
existing popular interaction modes.\;\<
}

\p{\:\+It is not uncommon in recent years to read suggestions
that technologies such as touch-screens \mdash{}
whether on phone apps or \i{in-situ} interactive
displays (checking tranit info from a large-screen
display at the station, say) \mdash{} are rendering
(now-) traditional \q{mouse-and-keyboard} interaction
modes \q{obsolete}. \> My comments about
multi-modality are \i{not} intended to reprise
these sorts of claims, because I think they
miss an important point: the interactions associated
with desktop applications in many ways
represent excellent technological design.
\> The mouse-and-keyboard setup fits comfortably
into the physical space around personal computers
and permits intuitive, precise gesture. \> Compared
to mouse pragmatics, touchscreens (for example)
actually feel clunky and coarse-grained.
\> In this sense \q{mouse-and-keyboard} is not
so much outdated as a benchmark: it is true
that this specific configuration is impractical
for many \i{in situ} cases where people will be
using software in a transit hub, say,
or a construction site, or smart-home control
screen. \> The mouse/keyboard paradigm
is old-fashioned at least in the sense that
it is bound to a certain form of computer
equipment and does not obviously translate
to future computing scenarios where software
is built in to our ambient environments.
\> The problem however is to design new
equipment which has a comparable acuity
and usability \i{to} mouse-and-keyboard
setups, and arguably to emulate these
as best as possible; i.e., the fact
that desktop-computer pragmatics does
not translate to \i{in-situ} modalities
should be considered a deficit to overcome,
not a feature relegating desktop computers to obscurity.\;\<
}

\p{\:\+Multi-modality, in short, is still an evolving
field of technology. \> Ideas such as \q{stylus}
pens may emulate the exactitude of mouse-clicks
and on-screen cursors \mdash{} allowing
for dense, feature-rich front-ends \mdash{} but they apparently feel
for many users difficult to haptically
manipulate; touch-gestures may feel more
natural, but they are fuzzier (forcing
simplified displays). \> An intriguing
form of multi-modality involves voice commands
(eschewing manual gestures entirely), but
it is not obvious how to avoid users' sense
of having to give extra thought into
vocalizing commands when their intentions
can be signaled by a tiny movement
(e.g., a mouse click), leaving aside the
question of ensuring sufficiently accurate
Speech Technology for seamless
\HCI{}. \> Discussions about speech-based
\HCI{} sometimes give the impression that
the impediments are solely in this latter
\NLP{} dimension, but even if we have
almost perfect computational capabilities
to transcribe and parse speech, I find
it hardly self-evident that pragmatics
where we \i{talk to} computers will be
(in general cases) a more intuitive
style of \HCI{} than physical gestures.
\> Psychologically, we comport to spoken
language in the guise of sentences and
speech-turns, unfolding multiple ideas
in sequential patterns. \> Language has a
different structure than fine-motor
skills, and most \HCI{} gesture lack
language's temporality and strucuturation;
instead, we interact with computers
optimally via multiple tiny
gestures in isolation, like clicking a
mouse, then moving it a short distance,
and clicking again. \> We should question
whether this highly discrete environment
for performing purposeful gestures
translates readily to a spoken environment.\;\<
}

\p{\:\+When contemplating future multi-modal \HCI{},
one reasonable question is how aggressively
\AI{} can be harnessed in this context. \> The
example of spoken-command interfaces
point to how \AI{} \i{could potentially}
engender new multi-modal capabilities, because in
principle we can bring (via \AI{}) much
more nuanced interpretations of user
actions than would be possible otherwise.
\> On the other hand, for reasons I just sketched
\AI{} at least in the \NLP{} context
isn't guaranteed to improve on more
mechanical \HCI{} options. \> We can, of course,
imagine other scenarios \mdash{} robots with
interactive displays, for example, that reconfigure
their shape to make gesture-based interactions
most convenient for users (adjusting based on someone's
height, for example), perhaps deploying
\AI{} to anticipate user preferences.
\> It is equally possible, however, that the
key to optimal \HCI{} \q{in the field}
will come from new interactive devices that
have features resembling (say) a mouse,
or trackballs, but adapted to more free-form
utilization in more physically open-ended
spaces. \> In short, within Industry 4.0 we
can expect an increasing emphasis on software
\i{in situ} \mdash{} digital tools that are coextensive
with sites where we perform activities, using
applications as enhancements to real-world
places and equipment rather than sequestered
in computer terminals \mdash{} but we need
to figure out how to implement
\HCI{} in such free-form contexts in ways that
do not minimize applications' \i{own} value.
\> Usability from the mechanical/enactive point of
view is not the same as application design
optimized from the perspective of making
many software features available to users
in a convenient manner. \> There are many
application paradigms \mdash{} context menus,
tooltips, keyboard modifiers and shortcuts,
mouseover effects and other pragmatic cues
\mdash{} which help organize software
functionality so that applications can
expose many capabilities while still
being intuitive and learnable from the
user's point of view. \> Replicating
this kind of feature-availability
\i{in situ} in industrial, architectural, or
immersive settings is a difficult problem, and
a different one than merely creating
interactive modes (like voice-activation
or touch-screens) that feel efficient if
we consider only the effort to a single
command, as opposed to using applications
for multiple and varied functionality.\;\<
}

\p{\:\+I will not speculate here on optimal
physical designs for future multi-modal systems.
\> My point is rather that the technology
should be considered still very open-ended,
so we cannot predict \i{a priori} how
application-development paradigms will
need to evolve in consort with multi-modal
\HCI{}. \> Rather than focus on any one
specific interactive mode, then, a
useful strategy would be to
advance application-level software engineering
so that new \HCI{} ideas and devices can
be integrated with existing application
code and convention as readily as possible.
\> I would argue that this goal, in turn,
should direct us to optimize the
building-block levels which are
essential to software implementation in
whatever \HCI{} guise \mdash{} database architecture,
data sharing, \GUI{} modeling, Requirements
Engineering, and so forth; in this
chapter I have focused on \VM{} engineering
as a means to this end.\;\<
}

\p{\:\+As a final comment about \AI{}, concerns I raised
about how best to incorporate \AI{} capabilities
into \HCI{} pragmatics potentially
reveal larger questions about modeling
the connection between \AI{} and computing
environments in general. \> A lot of
discussion about \AI{} seems to take place
in a vacuum, as if there mere existence
of new \AI{} or Machine-Learning driven
capabilities automatically translates
to software improvements. \> However, the benefits
of \AI{} depend on insights gleaned
from \AI{}-related technology being
presented to \i{people} in informative,
productive ways. \> This, in turn, calls
for interrogating how \AI{} capabilities
become situated relative to applications:
how are data derived from \AI{} visualized
and documented? \> How can we double-check
\AI{}? \> How can users initiate computational
processes where \AI{}: is one of or the
main driving force?\;\<
}

\p{\:\+These questions are phrased from users'
point of view, but analogous concerns
present themselves around programming
itself: how should applications
be \i{developed} in the context
of \AI{} integration? \> How should
\AI{} components be invoked operationally?
\> How should the parameters, initial
conditions, input data, or anticipated
goals for an \AI{} process be declared
ahead of time? \> How should \AI{} results
be encoded, along with information
\i{about} the \AI{} process, such as
degree of certainty, and how should
applications guide and fine-tune
\AI{} processes underway (e.g., request to
optimize either for time or for accuracy)?\;\<
}

\p{\:\+These are broad questions in the context
of \AI{} proper, but we can consider
some of the details more specifically
by narrowing to certain capabilities
\i{within} \AI{}, such as Computer Vision,
which will be a theme of the next
chapter. \> I do not intend to
insinuate that image-processing
in general is a canonical example
of \AI{} overall, or even that everything
broadly conceived in the scope of
quantitative image-analysis should be
considered \q{part} of \AI{}, but
in any case image-processing serves as either a
useful proxy or case-study for
\AI{} in the application-integration context.
\> This topic, then, will be revisited in
the next chapter.\;\<
}
