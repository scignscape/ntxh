
\section{Hypergraph Data Modeling and Virtual Machines}

\p{\:\+I've suggested so far that flexible and expressive data modeling
is a good thing in principle, but haven't presented any
arguments \visavis{} how that could be achieved.
\> Without sufficient care, the general topic of data-representation
paradigms can easily remain at the level of semi-formalized
\q{design patterns} or (concrete programming) best practices
\mdash{} subject matter for developer's message boards and the like
but lacking a rigorous technical firmament to be
analyzed in depth. \> I believe that investigation of data-representations
can be a scientific topic befitting in-depth analysis, not
just a semi-formal cadre of observations orbiting coding practice.
\> I do not expect that claim to be taken for granted, however;
anyone inclined to such a perspective should justify their
interest in data (meta)models by construction a sufficiently
rigorous environment in which to analyze them.\;\<
}

\p{\:\+My earlier comments regarding application-level expressivity
perhaps suggest part of my rationale for stressing metamodels
\mdash{} application-development criteria break apparent
equivalence between disparate representational paradigms
which may encode comparable data-spaces, but with divergent
levels of \q{convenience} from the application vantage point.
\> That is to say, one (at least potentially formal or formalizable)
criterion for comparing representational schema is the space
of data structures which competing schema can unambiguously
represent; one schema is richer (or more expressive) than another
if there are structures (with requisite concrete data and
associated semantics) that can be encoded (without collision
between nonidentical structures) in the first but not
the second. \> I'd argue this amounts to a variation on the
notion in formal language theory that two languages
are analogous if anything which is a valid string in one
is valid in the other (respectively, invalid) and vice-versa;
except that when discussing \q{data structures} we are
working with a different universe of building-block elements
than \q{strings}, which still need to be described
(as will become clear, I propose to articulate these
building-blocks in terms of hypergraphs). \> Accepting
graphs in a generic sense as a theoretical \i{a priori},
we could say that two representational systems are
functionally analogous if anything which is a valid
graph-construction in one is valid in the other
(respectively, invalid) and vice-versa. \> One would
then have to clarify what are \q{graph constructions}.\;\<
}

\p{\:\+From an \q{application-level} perspective, however,
representational paradigms can be compared not only
via structures they encode in an abstract sense,
but more concretely by the degree to which
applications can be extended with capabilities to
share data according to protocols derived form their
representational tactics. \> This idea seems intuitively
well-motivated, but it rests on notions which perhaps do
not have an obvious formalization (akin to, say, equivalence
\visavis{} a space of valid strings/graphs). \> So, I will take a
moment to flesh it out.\;\<
}

\p{\:\+Prima facie, it should be uncontroversial to say that
convenience in application-development is desirable. \> This
section of the book will entertain several Civil Engineering
case-studies concerning reconstruction in Ukraine. \> At this point
I'll just point out one contention to be addressed in
following chapters, that the scale of Ukrainian
destruction allows for rebuilding initiatives to be
carried out on an exceptionally large and holistic
scale. \> This reconstruction calls for an interdisciplinary
mindset attuned to environmental impact, quality of life, and
cultural heritage alongside conventional architectural/engineering
concerns. \> This is intriguing case-study insofar as
it concretizes scenarios where existing data-integration
methodology in Civil Engineering Informatics (CEI) \mdash{} how
\BIM{} marries architectural design and structural engineering,
for example \mdash{} can be extended to urban planning, sustainability,
even cultural preservation. \> The computational infrastructure
leveraged when assessing structural integrity
of (digitized) architectural plans, that specific
analytic pipeline, can similarly be extended to
metrics of environmental impact, energy use, carbon
footprint, and myriad cultural and quality-of-life
targets.\;\<
}

\p{\:\+Although such inter-disciplinary efforts wherein the
scope of CEI integration expands further \mdash{} e.g., emphasizing
eco-friendly optimizations \mdash{} have certainly
been embraced by the construction and architectural-design
industries in recent years, Ukraine points to how
such trends could still accelerate. \> A widening
circle of parameters, from architecture and engineering
to ecology, culture, and civic design, will ideally be
recognized \mdash{} and digitally prototyped, incorporated
into simulations and \q{digital twins} \mdash{} by
technologies utilized toward reconstruction designs
(and later, potentially, by \q{smart city} data curation).
\> As of now this is hypothetical, of course, but
working on the assumption that such broad integration
is at least \i{desirable} we can derive something
like a concrete case-study, because we are contemplating an
expansion in specific applications' data-sharing capabilities
to accommodate newer, more multi-faceted information
spaces.\;\<
}

\p{\:\+In other words, this is a hypothetical but in its own
way concrete working example: consider the suite of
software tools which are employed in construction-related
industries, and their existing interoperability
(via \IFC{} objects, for instance). \> Now consider the
goal of augmenting this functionality to support
thematically richer information profiles covering
civic, ecological, and cultural data-points alongside
(say) architecture and structural engineering.
\> We then have (modulo a certain imaginative projection
presupposed by the whole thought-exercise)
real-world implementational questions to consider:
How readily can applications be updated to support
the relevant new data-sharing protocols? \> How can
these protocols be standardized (and likewise
application code-bases be engineered) in anticipation
of quickly adopting to more ambitious interoperability
specifications? \> What about supporting tools
(parsers, validators, reference implementations,
client libraries, even application-neutral \GUI{} components)
that might be provided in conjunction with new data standards
to facilitate application-level adoption?\;\<
}

\p{\:\+These are the kind of contexts I had in mind a short while
ago when suggesting that application-integration is a functional
criterion for assessing data-representation formats, alongside
more \q{mathematical} criteria such as the space
of information structures (unambiguously) modeled via a
given format. \> I'd like to keep case-studies
such as Ukrainian reconstruction present in the background
as a motivating example for following discussions
about data structures and representations.\;\<
}

\subsection{Hypergraphs and Virtual Machines}
\p{\:\+To extract a theoretical core from these hypothetically-concrete
case-studies, we can say that valid criteria for
assessing data metamodels derive in part from the
perspective of engineering application-level support
for new (or newly expanded) interoperability protocols.
\> That is, representational systems which engender
computational artifacts (parsers, for syntax, and
validators, for semantics, let's say) that can be
implemented, embedded, and leveraged by applications
as their capabilities are refined to support new
protocols are most effective when the amount of
effort consumed by the requisite programming
is minimized. \> Data-sharing protocols should be
engineered to pass through application-integration
phases as quickly as possible, but likewise
applications should be architected to accommodate
new data-sharing initiatives without
substantial re-coding.\;\<
}

\p{\:\+Generically, we can describe a metamodel as
relatively \q{expressive} to the degree that
data structures and their concomitant semantic
paradigms can be transparently encoded
according to modeling system's rules.
\> Expressivity might imply a level of
redundancy, or at least superficial
redundancy when viewed from the
sole perspective of data encoding.
\> For example, the \IFC{} attribute/property
distinction might seem locally superfluous
in that asserting a given data-point via a
property rather than an attribute
(or vice-versa) does not appear to
convey greater information \mdash{} the
information is borne by the
field's value, not by its property-or-attribute
classification. \> However, the attribute/property
is of course not semantically vacuous;
its significance emerges in the larger
scale of schema-standardization and validation.
\> A metamodel which supports this larger semantic
context is therefore sufficiently expressive
to properly encode \IFC{} data; a less expressive model
(one which collapses attributes and properties
into a generic notion of \q{fields}, say), would
fall short, at least without compensating
by leveraging its own formal resources
(e.g., classifying fields as attributes or
properties via \q{meta-data} fields).
\> Here we can appeal again to application-level
concerns: the paucity of less-expressive
systems would become evident insofar
as reifications, indirections, and other ad-hoc
solutions to representational blind-spots
can render application-integration more complex
and time-consuming.\;\<
}

\p{\:\+Metamodels are more expressive to the degree that
they offer a greater range of representational parameters
with which structural and semantic conventions
might be communicated. \> For example, \JSON{}
(JavaScript Object Notation) can be deemed more
expressive than primitive list-based encodings
(e.g., \CSV{}, or comma-separated values) because
\JSON{} distinguishes arrays (which are like lists)
from \q{objects} (i.e., associative arrays,
lists indexed by field-names rather than numbers).
\> In this sense \XML{} is more expressive than
\JSON{} insofar as elements' data can be asserted
via attributes or via nested
elements (for sake of argument, treating \XML{}
as a de-facto superset of \JSON{} wherein arrays
correspond to sequences of similarly-tagged child
nodes). \> As that last parenthetical comment suggests,
there is a level of imprecise inevitable when
discussing issues such as the relative
expressiveness of real-world data formats;
these are concrete technological artifacts rather
than mathematical constructions, and therefore
do not necessarily lend themselves to a
quasi-mathematical meta-language. \> Still,
we can analyze distinct
representation systems with some degree
of systematicity (sort of in the
register of \q{philosophy-of-information},
not \q{information-as-mathematical-system}).\;\<
}

\p{\:\+In the case of property graphs and/or hypergraphs,
data fields can be associated with an \q{object}
(in the sense of an integral data structure)
via properties (attributes on a node) or via
node-to-hypernode relations (sometimes called
\q{projections}), a duality reminiscent
of property/attribute in \IFC{}. \> Graphs,
of course, have the further stipulation
that any two nodes (or hypernodes) may be
linked by edges (themselves equipped
with labels, label-namespaces, controlled
vocabularies, and potentially constraints/axioms
enforced by \q{Ontologies}). \> Formats such as
\XML{} and \JSON{} which are more \q{syntactically}
oriented tend to be hierarchical, in that
any specific value in a \JSON{} object or array may
itself be another object/array (rather than atomic
value) and likewise \XML{} elements may have other
elements as children. \> By contrast, formats
such as \RDF{} and property graphs which are
more \q{semantically} focused tend to be
graph-like and encode semantic relations via
edges across nodes (rather than hierarchical nesting).
\> Hypergraphs potentially combine both styles of
representation, with hypernodes containing
nodes hierarchically \i{and also} edges
between two (or two-or-more) nodes and/or hypernodes
(the precise rules as to which constructions are
possible will vary from one system to another, but
these are reasonable approximations).\;\<
}

\p{\:\+Any representational system, as these examples point out,
provides a certain \q{tableau} of parameters that
can be pressed into service when formalizing a protocol
for encoding specific kinds of data via structures
conformant to the specific system. \> More expressive
systems have a wider arsenal of parameters; for
instance, along the lines of the above gloss, property
hypergraphs (with potentially iterative node/hypernode
relations) are more expressive than either property
graphs or document-style hierarchical trees
alone, because properties-on-nodes (as in \XML{} attributes),
nested hierarchies (hypernodes containing child nodes akin
to child \XML{} elements), and inter-node connections
(as in Semantic Web labeled edges) are all potential
representational devices. \> This chapter will accordingly
focus on property-hypergraphs as a general-purpose
metamodel, but the relevant point for the
moment is that expressivity can be \q{measured} via
the range of distinct representational parameters afforded by
the system, at least intuitively.\;\<
}

\p{\:\+This intuitive point can be made more precise, insofar as
I have deferred any rigorous definition for
\q{representational parameters}. \> That is to say, for a
reasonably systematic analysis of metamodels we should
specify building-blocks of constructions
recognized through any metamodel. \> The overall
concept may be clear enough \mdash{} in general,
the parameters of a modeling system are the full set
of structural elements that might potentially
be employed in fully describing any given
structure covered by the system \mdash{} but one would
like a still tighter definition. \> For this chapter,
I approach this problem from the
perspective of Virtual Machines.\;\<
}

\p{\:\+The correlation between Virtual Machines and
representational paradigms should be clear: suppose
we take any structure instantiating a particular
metamodel. \> Presumably, such a structure
can be assembled over multiple stages. \> Insofar
as node-hypernode inclusion is a constructional parameter,
for instance, then a structure can be modified
by a including a node within the scope of a hypernode.
\> Similarly, insofar as inter-node relations (via directed
edges or hyperedges) are significant constructions,
then a structure may be modified by adding an edge
between existing nodes. \> In short, any structure
can be derived from the modification of precursor
structures. \> The full set of structure-modifying
operations available for a given representational
paradigm can be enumerated as (at least on part of)
the opset of a hypothetical (or realized)
Virtual Machine. \> As such, Virtual Machines
provide a potential formalizing environment
for analyzing data metamodels.\;\<
}

\p{\:\+Conversely, data-models can serve as a prompt for
motivating the proper scope of a Virtual Machine
(hereafter abbreviated to \VM{}). \> That is, \VM{}
may be designed subject to requirements that
they permit the accumulation of any data structure
conformant to a given metamodel. \> Similarly, \VM{}s
might be characterized in terms of how they support
various \q{calling conventions}, in the sense of
protocols through which computational procedures
delegate to other procedures (supplying inputs,
reading outputs, spawning concurrent
execution paths, and so forth). \> This chapter
will focus on \VM{}s based on \i{hypergraph}
data models, employing such structures
both from the perspective of data-representations
and interlocking procedures (i.e., describing
procedures in terms of sequences of
calls to other procedures).\;\<
}

\subsection{Virtual Machines and Database Engineering}
\p{\:\+Virtual Machines can be useful tools for studying software-interoperability
insofar as both data-representations and communications protocols
can be modeled in terms of \VM{}s (at least as abstract summaries
of working code; or, more ambitiously, application-networking
frameworks can be provide actual \VM{}s through which applications
can route their networking logic, analogous to query languages
as host-language-agnostic conduits for database access). \> In this
context \VM{} models overlap with constructions pertaining to
interoperability between applications and database engines.
\> Consider the general setup of database systems: applications
store data structures for future reuse by passing some
(suitably encoded) serialization of the relevant information
to a database, which arranges the data into a layout
optimized for storage and retrieval/querying. \> Typically the
database will attempt not merely to preserve the presented
data structure for future reconstruction, but will
index or destructure it in such a manner that such specific
data can be selected as matching a future query.\;\<
}

\p{\:\+At any moment in time, an application will be working with
one or more data structure that might be called \q{live}; they
are directly implicated in the state of the application at
that moment. \> The current user (or a different user) might,
accordingly, be interested in reconstructing that
application-state (or some relevant subset thereof) at a
future point in time; at which point database queries are
typically necessary, because the relevant information is
no longer in live memory. \> To be sure, a database does
not necessarily store \q{application state} as such
\mdash{} although developers can potentially
create \q{application state objects} and persist
those so that users can resume prior sessions \mdash{} but
a typical rationale for persistent data storage
in the first place is functionality supporting
users' desire to resume prior work, return to
previously-viewed files, and so forth. \> The
individual values stored in a database derive their
significance from how they interconnect
in users' experience in the context of any application.\;\<
}

\p{\:\+As a concrete example, suppose we are considering an application
which (at least as one of its features, and in the context
of particular \GUI{} windows or components accessed through
the software) supports viewing, tagging, and annotation
\TwoD{} images (e.g., photographs). \> More precisely, suppose
we are considering an image-processing application
specifically implemented for Ukrainian reconstruction.
\> A typical image for such an application might be a plot
of land on which real estate will be rebuilt; potentially
images would depict remnants of prior buildings that
were destroyed, and/or could be annotated with
data relevant to reconstruction designs
(e.g., in recreating damaged neighborhoods
a certain number of residential units might
be targeted for each parcel of land or each block
subject to redevelopment, perhaps optimized by
models measuring the ideal urban density for
that specific neighborhood based on environmental
criteria, utility grids, public transit, and so forth).
\> A typical session for such an software component
might involve users viewing individual images,
previewing image-series depicted via thumbnails,
searching for images (based on criteria such as
street address, city/district name, or perhaps
\GIS{} coordinates), switching between \TwoD{} image
and 3D street views, and \mdash{} once in the context
of a specific picture \mdash{} viewing annotations
and other associated data in tabular or
otherwise structured forms (e.g., tables or key-value
pairs displayed through independent \GUI{} windows
floating above the graphics viewport).\;\<
}

\p{\:\+Assuming such an application works with relatively
large image-collections (enough to be impractical
for users simply to browse images in a filesystem
folder, say), functionality for finding and tracking
images would need to be based on some
systematical query capabilities, i.e., some
form of image database (which need not imply
that images themselves are stored as database
\q{blobs} \mdash{} binary large objects \mdash{} but
at least that image file paths, feature sets
for retrieval/similarity searches, metadata
and formatting details, etc., can be hosted
in a database so that images can be selected
inside large series by variegated query strategies).
\> Suppose an image depicts a city block where
damaged buildings have to be replaced; data
associated with the image could include
estimates of the number of people who lived in
that location prior to 2022 war; the number of
residential units slated for redevelopment;
cost estimates; links to public transit info,
street views, data concerning utilities grid,
etc. \> Plausibly, such data would be held
in a database and loaded alongside the image,
or in response to user actions signaling an
interest in the relevant data-points. \> The
database, in effect, is a means to an ends,
whereas from a user's perspective the
important detail is that the application
can enter a state where multiple important
data-points are visible side-by-side: users
might view a photograph in one window juxtaposed
with a window or windows showing civic/residential
data.\;\<
}

\p{\:\+Continuing this specific (hypothetical) example, the
envisage scenario has image-viewport components
serve as an entry-point for a diversity of
civil/architectural information; presumably
the specific kinds of data available will vary
from one image to another, and users will
signal through interactive \GUI{} features
their interest in accessing certain branches of
the available data over others. \> That is, assume
there is not a fixed metadata/associated information
package that automatically accompanies each image,
but instead that data and images are linked on a
dynamically changing case-by-case basis. \> That setup
would call for a coding strategy which works to
organize the available information and
user-interaction pragmatics coherently.
\> The resulting software-design choices would,
moreover, propagate to \GUI{} and database designs
as well. \> Once some aggregate of data is identified
as a coherent unit, this structural decision
must be accounted for at the \GUI{} level
(insofar as users request \i{that specific}
data from application-states where they are
viewing an image carrying the appropriate
information) and the database integration
(\q{that specific} data has to be
queried from the larger database context when needed).
\> In other words, the interaction between \GUI{}, database, and
application logic is more complex than if each image were
given a fixed set of data fields isolated from user pragmatics.\;\<
}

\p{\:\+By way of illustration, suppose one information-bundle that
could be associated with (some) photographs outlines
redevelopment plans: data points such as the number
of residential units targeted, renderings of building
designs, contractors, contact information, and so forth.
\> Presumably, such data would only be applicable
to images showing blocks or plots where such plans
are in the works; and moreover such images would link
to other forms of data as well (about, say, utilities
grids). \> As part of the empirical background, in
effect, one might conclude that various facts
pertaining to individual reconstruction projects/contracts
can both be aggregated (as interconnected data-points)
and isolated from other information potentially
relevant to an viewed image. \> In general, software
design depends on sensitivity to how information spaces,
practically speaking, can be carved and organized.
\> The decision to isolate (say) construction-project
data as integral units would be made against
that kind of design/decision context. \> Having
this topic reified as a specific \q{category} of data,
say, affects \GUI{} programming and event-handling
(because user-visible components must be implemented
enabling users to access information in that
category, which in turn yields signals like
context-menu activations and the need for appropriate
event-handlers) as well as database interop (insofar
as such data has to be packaged in persistable forms).
\> Subsequently, representations of such integral construction-project
data (in the form of, e.g., a cluster of interrelated
datatypes) would be manifest as software artifacts
threaded through a code base.\;\<
}

\p{\:\+The specific patterns of data organization and programming-language
types engineered for an application reflect practical
concerns and aspirations to optimize User Experience; these patterns
do not necessarily map neatly to native database
architectures. \> Neither relational databases (built up from
tables with fixed tuples of single-value columns) nor
conventional graph databases (whose representations are confined to
one layer of labeled edges) generically match the
multivariate and multi-level structure of real-world information
typically managed at the application level. \> This is why
application-level data types generally need
to be restructured and transformed when routed between applications
and database back-ends.\;\<
}

\p{\:\+Software engineers are responsible for ensuring a proper
synchronicity between application and database state, although
(as intimated above) this does not (by and large) entail
application-state being directly persisted in a back-end.
\> Instead, back-end updates are a property of application-state at certain
moments; insofar as users have performed edits or in general
made changes resulting a mismatch between the data as
currently seen by the user and what is stored in the database,
the latter has to commit such changes for perpetuity.
\> Update-worthy application-state then needs to be
distributed over (potentially) multiple database
\q{sites} which are collectively implicated in an update.\;\<
}

\p{\:\+For sake of argument, consider an update (or part thereof) consolidated
into a single (application-level) datatype instance; e.g.,
one object of a given \Cpp{} class. \> Quite possibly, there is
no one-to-one correspondence between application objects and
database values or records, particularly if the classes
in questions contain many data fields and/or multiple
one-to-many relationships and values which are more
involved than simple strings or numbers (e.g., pointers
to other objects). \> The structural mismatch between
application data and database records is evident in
technologies such as Object-Relation Mapping (\ORM{})
\mdash{} or \q{Object Triple Mapping} for the Semantic Web \mdash{}
marshaling data for relational databases or triplestores,
respectively. \> Engines with more flexible
representation paradigms, needing less reconstruction
of application data exported to a back-end, can be
advantageous precisely because data-persistence
capabilities end up consuming less \q{boilerplate} code.
\> Hypergraph databases are a case-in-point: the kind of
complexity in datatypes' internal orgnanization
(multiple multivariate fields for a single object,
for instance) which give rise to \ORM{}/\OTM{}-style
transforms map organically to hypernode or hyperedge
constructions.\;\<
}

\p{\:\+Modifying a database entails conveying a package
of database between applications and the database
engine; the structure of this \communique{}
in turn reflecting database architecture.
\> The \SQL{} \INSERT{}
statement, for example, derives its form
from the layout of table-base data models:
adding a record entails naming the table to which it
will belong, which brings on board the specific
list of fields (columns) whose values have
to be accounted for in the query. \> Our impression that
relational algebra is a relatively crude or inflexible
meta-model derives, it would seem, in large part
from the quantity of bridge code needed to implement
database updates through query commands such as
\INSERT{} that are restricted to the relational
architecture. \> To the extent that hypergraph
engines (for example) are more flexible in principle,
this advantage only becomes concretely
evident to the degree that hypergraph databases
support a query system such that updates (and
analogous modifications to a database instance)
are initiated with relatively less
boilerplate code.\;\<
}

\p{\:\+As I alluded to earlier, hypergraph data models
have comparatively greater parameters available
for representing information; in particular,
this implies that we have greater
flexibility in formulating queries
to modify database instances. \> It's worth mentioning
at this point that \q{queries} need not involve
instructions passed to a database engine
in the form of character strings (e.g.,
\SQL{} code); indeed, forcing applications
to build query code on the fly is a often
an antipattern (spurring boilerplate code-bloat);
better solutions involve query \q{factories} that
assemble queries via procedure/method calls
in a host programming language (perhaps
through an embedded domain-specific language,
as one finds with \LINQ{} \visavis{} \CSharp{} and
its emulations in other languages). \> On the
other hand, in the best case scenario a database
would \i{also} support a query language that
could be executed as text strings outside of an
application context (and without relying on a
host programming language), for examining
the contents of a database in situations removed
from application-based access (debugging, analytics,
general admin functionality, and so forth).
\> In other words, ideally engines will encompass
a query engine that works with query-structured
compiled \i{either} from application-code factories
\i{or} standalone query code, which is one rationale
for embracing a query-evaluation Virtual Machine.\;\<
}

\p{\:\+The specifics of database updates \mdash{} sticking again for
sake of exposition to single type-instances \mdash{} depends
of course on types' internal organization.
\> For typically atomic types
(like 1, 2, 4, or 8-byte integers) updates might
only entail replacing one value with another, but
more complex values (\q{objects}, in effect\footnote{Taking
the perspective that in some systems objects are formally
defined as aggregate structures (rather than atomic data-points)
and, even if not precisely stipulated, object/atomic value
distinctions tend to align \i{de facto} with object-classes
against, say, built-in types (cf. \Cpp{})}) with multiple
and/or multi-value data-fields updates can include
adding or removing a value from a collections type-instance
as well as altering such a value, and so on.
\> For many compound types the collection of fields
includes more than one which are in turn \q{collections}
(e.g., vectors, stacks, queues, deques, and unordered sets/multi-sets,
plus map-arrays that are pair-lists with possible nonduplication
restrictions on the first element) subject to add/remove
operations, in contrast to full-on value-to-value
replacement. \> Some collections have modification-restrictions
(e.g., values can only be added or removed from one or
both ends of a list, or, as in typical \q{sets}, all added
values must be unique) or enforce constraints such as
monotone increase and decrease.\footnote{Consider a collections type
\mdash{} a construction supported by the Virtual Machine I use for
demo purposes (discussed below) \mdash{} which has only one insertion operation,
but will automatically place new values either at the beginning or
end of the list to preserve increase-direction; here,
new values have to be either greater or less than all
prior values. \> Or, automatically sorted lists need only
one insertion operation because the insert procedure would
deduce the proper insertion-point.
}  All of these are potential paths toward
legal mappings of type-instances between states
which initialized typed value can take on, given their
internal organization. \> Nor is this discussion complete;
we could also mention reclassifying \q{union}-type values
(whose instances can be one of several types) or
various types depending on binary arithmetic
(cf. tagged/\q{decorated} pointers, or
enumerations whose value can be members of
nominal-value lists \i{or} bitwise combinations
thereof, or unions where multiple type-tags
are valid by virtue of shared binary encoding,
in effect using one tagged value to update another
member of the union \mdash{} a simple case being
integer/bitset unions where setting the
integer automatically sets or clears corresponding
fields in the bitset).\;\<
}

\p{\:\+The mechanisms for aggregating multiple values
into individual type-instances tend to be
much more complex for general-purpose programming
languages such as \Cpp{} (see unions, pointers, arrays,
multiple inheritance, enumerations and their
base types, etc.) compared to databases
(see \SQL{} or \RDF{} types); this is a proximate
cause of complications in persisting application-level
data. \> Conversely, databases with more refined
type systems can (at least potentially) absorb
application data more conveniently. \> For this
to work in practice, however, the
engine needs a query system which can
duly leverage type-system expressivity; the issues
involved here are well-demonstrated by
update-queries as I've mentioned. \> Properly recognizing
application-level intra-type organization
depends on representing (and then
exposing to a query interface) the full spectrum
of morphisms through which a single type-instance
might be updated.\;\<
}

\p{\:\+For complex types, an effective query-system would have
update operations that are more targeted than
just replacing one value with another
\i{tout court}; instead, updates may only
involve one or some subset of all data-fields,
and (for multi-value fields) could involve
insertions/deletes in a collections context
rather than a direct value-change. \> The degree to
which a database engine seamlessly interoperates
with applications depends on the latter
constructing data-packages (without undo effort)
that signal the \i{kinds} of updates
requested alongside the relevant new values
(notating, e.g., collections-context changes as
distinct from single-value morphisms).
\> Intuitively, flexible architectures
(e.g., hypergraphs) accelerate the requisite
implementations, although actually
supporting a query-interface satisfying
such ambitions depends on a confluence
of factors (e.g., underlying database
architecture but also query-representation
and query-evaluation protocols and the
tools to compile code/text or procedurally-generated
queries to internally-represented structures
suitable for evaluation).\;\<
}

\p{\:\+This section's discussion has focused on
updating a single type-instance centering on
the point that complex intra-type layout implies
a diverse set of query-based update options.
\> An \UPDATE{} statement in \SQL{}, say,
only (directly) supports one
particular \q{genre} of updates (value-to-value
edits in one or more discrete columns). \> Given application
state which can be expressed in terms of modifications
to existing persisted values, keeping the
back-end in sync entails accounting for the changes
embodies in the new application-state by
presenting the back-end engine with (in general) a
series of updates; the more flexibly updates
can be encoded, the less development time
need be expended figuring out how to
translate application-state changes to updates
the engine can process. \> This is one reason why a
diversity of data-modeling parameters can
streamline application integration: a flexibly
tableau of recognized constructions implies that
applications can describe updates with
relatively less boilerplate destructuring.
\> Consider the multitude of \q{sites} where
data associated with a hypernode can
be asserted, in the context of property-hypergraphs; at least
(and some systems may have more elaborate
constructions as well) properties \i{on} a hypernode,
nodes \i{in} the hypernode, and edges \i{from} a
hypernode to its peers. \> This articulation
of site-varieties is, self-evidently, also a
list of update-forms. \> Having multiple protocols
for describing updates allows different forms of
updates to be recognized with distinct semantics.
\> For example \mdash{} consider again the attribute/property
distinction in \IFC{} \mdash{} updates to \i{properties}
(which would in general be ad-hoc annotations
on a hypernode less strictly regulated than
nodes encompassed \i{in} hypernodes) could be
subject to different validators than updates
performed via node-insertion or (potentially
Ontology-constrained) edge-insertion.\;\<
}

\p{\:\+In effect, multi-parametric modeling
tableaus in the database \i{architecture}
propagate to multi-dimensional
options for encoding updates, which in
turn allows for coexisting update
protocols each with their own semantics.
\> Applications can then choose which
protocol most efficiently describes
any particular change in application-state.\;\<
}

\p{\:\+Of course, these points \visavis{} changing
\i{existing} database value have analogs
in the context of inserting \i{new} values.
\> Ideally, applications will have flexibility
in how encode data structure for insertion
into a back-end via database queries.
\> This is not only a matter of notating
all information which should be persisted,
but also providing cues to the engine
about how the new data should interact
with other records (e.g., using
primary keys or globally-unique
identifiers to secure inter-record
links analogous to \mdash{} perhaps
translating \mdash{} live-memory pointers)
and subsequent find/select queries.
\> What are the criteria through which
a database record, once deposited, should
be retrieved again in the future? \> Most
database systems will give nodes/records
unique id's, but the whole point of
search queries may be that records
should be located based on known data
in contexts where an application does
\i{not} have the requisite \gid{}.\;\<
}

\p{\:\+Consider again the case-study of image-curation
software that could be used in a redevelopment/urban
planning context, such that photograph resources
include depictions of future building sites.
\> As suggested earlier, data associated with
each picture could reference construction-project
data (e.g., the number of units slated for
construction, or the identity of the firm contracted
to oversee the project), as well as, perhaps,
\q{generic} information (image format, dimensions,
color-depth, plus, say, \GIS{} coordinates).
\> One consideration when designing such an application
would be how users would find images
that they hadn't seen before, or had not
revisited for an extended period of time
(so that the presumptive image \gid{} is not
cached in recent history). \> In would make sense
to track images by longitude and latitude,
for example, assuming that users would know such
data. \> Perhaps street address (accounting for
the possibility that large-scale redevelopment
might alter the street grid so that pre-war
addresses, say, become obsolete; one might
still maintain a mapping from such addresses
to \GIS{} coordinates so they remain useful
for queries); or (less granularly) by district
or city. \> We can similarly envisage scenarios
where architectural details are mentioned
(\q{find images of sites within a
10-kilometer radius featuring planned
buildings over 4 stories tall}).\;\<
}

\p{\:\+For queries along these lines to work,
back-ends need to structure type-instances
such that (potentially large collections of)
values can be filtered into subsets
meeting specific criteria: \GIS{} coordinates
restricted to a given locale, housing
matching a given contractor-name, and so forth.
\> When exporting info to a back-end, the
relevant details are not only the specific values
comprised by the new data but also
which fields may serve as eventual query-parameters,
and how they should be indexed.
\> Such \q{selectability} criteria have to be
encoded alongside persisted data structures
themselves, and convenient application-integration
entails supporting protocols for
noting pathways for query-retrieval
and doing so with minimal boilerplate code
(for analogous reasons as with update queries).\;\<
}

\p{\:\+Of course, selection/retrieval and update protocols
tend to be defined on types rather than the type-instance
level. \> To the degree that certain data-fields are
indexed and queryable (for retrieving instances
when their global id's are not at hand \i{a priori})
decisions as to which fields to thereby expose
tend to be made for each type \mdash{} as part of the
type's design and contract \mdash{} rather than
negotiated on a case-by-case basis per
instance (though note that properties in property-graphs
are possible exceptions; indeed this is one
rationale for property-graphs in the first place).
\> Similarly, type-level modeling tends to
define which update protocols to invoke for
different state-changes. \> Accordingly, an expressive
query interface should allow stipulations regarding
updates and searches to be described as attributes
of \i{types} as well as single instances (e.g.,
records and/or hypernodes) and to be deferred
from instances to type-contracts (analogously,
inferred in instance-contexts by virtue of type
attributions). \> An obvious corollary here
is that query systems have to recognize type
descriptions as well as encodings of type-instances.
\> In effect, query languages need to include
\q{type-expression} languages where types'
attributes, internal organization, and
back-end protocols are duly notated (so that
type-information can be \q{loaded into} the
system and consulted as the engine resolves
how to accommodate insertions, updates, and
filtering for specific instances).\;\<
}

\p{\:\+I will delay further discussion about type-descriptions
until after examining relevant Virtual Machine
concepts in greater detail. \> Thus far I have
been alluding to \VM{} in the context of query evaluation:
one way to implement query engines is to decompose
(the steps needed to execute) queries into
sequences of primitive or \q{kernel} operations
supplied through a \VM{}. \> While this is a productive
facet of \VM{} applications, it overlaps with equally
consequential issues concerning how \VM{} model
procedures in general (not just those befitting the
profile of database queries). \> Indeed, in general
a \VM{} targeted at query-evaluation should either
natively extend to or solicit
(via some kind of native-function interface)
general-purpose procedures available within
computing environments where databases and
applications are situated, because (in principle)
the results from arbitrary procedures could
potentially be desired as query parameters.
\> If (assuming an object-oriented context)
back-ends warehouse records encapsulating
objects with their specific classes, any
methods called on candidate objects
(e.g., those not otherwise
filtered out via the suite of selection-criteria
in a retrieval query) might plausibly be
useful as means to narrow result-sets.
\> Object-databases proper point to how
full-fledged method calls may be
hard to optimize (database contents are
not typically \q{live} objects that can be
passed to methods directly), but
there is no reason \i{a prior} why queries
should be limited to optimizable
criteria (for instance, if selective
stipulations allow results to
be narrowed to a reasonably small
set of candidates, fully instantiating
such potential matches as live-memory
objects and calling methods accordingly
would be a reasonable execution strategy).
\> In short, even when we are primarily
interested in Virtual Machines pressed
into serve as query engines, it would
be incomplete to exclude consideration
of general-purpose procedure calls and
how these are described, validated, and
executed by concrete \VM{}. \> I therefore
turn to procedure-encoding considerations
in the remainder of this chapter.\;\<
}
