\section{ChasmVM and the Digamma Calculus}
\p{\:\+As a preliminary to analyses later in this
section, I will make a few comments which might
situate this discussion in the context of
topics covered elsewhere in this book. \> Specifically,
I intend to motivate the discussion by appeal
to technologies related to industrial computing
and cyber-physical systems (\CPS{}).\;\<
}

\p{\:\+One potential use-case for Virtual Machines is encapsulating
cyber-physical networks: we assume that \CPS{} devices are
sensors and/or actuators managed remotely
by more-conventional computers. \> In general, sensors measure
physical quantities (air quality in a room, say) whereas
actuators have tangible physical effects (optimizing air
quality by adjusting vents, say). \> Neither kind of
device typically has extensive computational capabilities;
instead, data is routed from the devices to central
location which use software to process sensor data
and convey instructions to actuators. \> These networks
can, in turn, cover multiple intermediate points
(consider \CPS{} data warehoused in the cloud) but,
for exposition, we can focus on the essential end-points,
on one hand the devices themselves and on the other
software applications through which humans
monitor and, if desired, manipulate cyber-physical systems.
\> Focusing for discussion on sensors, the actual devices typically
do not perform calculations (they do not have the means
to execute computer code) but they \i{are} capable of
converting some physical quantity to digital
signals which computers proper can interpret.
\> Programmers developing \CPS{} applications will rarely
interact with devices independently; instead,
the \q{physical} links with \CPS{} networks\footnote{Which
of course can be physical only in an ethereal
sense, e.g., passing data via wireless services} will typically
be handled by low-level driver programs, which might
expose capabilities to access sensor readings through,
for instance, \C{}-language functions.\;\<
}

\p{\:\+Developers may employ Virtual Machines in the
\CPS{} arena to streamline this process \mdash{} instead
of programmers needing to write code which
interfaces with \C{} drivers directly,
the low-level function calls can be managed
via virtual machines that interoperate
with higher-level applications in a language-agnostic
manner. \> In this sense \CPS{} \VM{}s can be analogous
to database query languages, in their design and
rationales: just as query formats give
engineers the option of communicating with a database
from a diversity of application-environments, the
protocols for accessing \CPS{} data could similarly
be encapsulated in \VM{} operations. \> Moreover,
as touched on above query engines can ideally
function either as standalone languages of their
own (processing query-code outside an application
context, e.g. on a command-line for debugging and
maintenance) or via query-factories programmable
from general-purpose languages; the situation is
comparable for \CPS{}, and one could pursue the
same duality in how \CPS{} network capabilities
are exposed (either through host languages or
special extra-application code).\;\<
}

\p{\:\+If a \VM{} is indeed devoted to specific Cyber-Physical
network (or group thereof) then a logical first step
when implementing the \VM{} would be to articulate the
collection of procedures (through driver libraries, say)
to be encapsulated. \> More likely, a more-generic
\VM{} would be deployed in different contexts, each of
which (by targeting a specific group of \CPS{} devices)
would work against particular groupings of driver code,
on a case-by-case basis. \> The overall \VM{} would thereby
introduce capabilities to interface with some
collection of kernel functions exposed by driver libraries,
with the actual integration to those procedures
finalized during deployment. \> In any case, though,
the general pattern is that any particular deployment of a
(\CPS{}-context) \VM{} would be \q{seeded} with some
preliminary set of functions (i.e., built-in \VM{} operations,
some perhaps added on in deployment).\;\<
}

\p{\:\+At a minimum, in short, \VM{}s should furnish functionality
imitating \q{kernel} driver or operating-system calls,
initializing input arguments with value passed from applications
and/or sending back to calling application procedures' results.
\> The \VM{} accordingly bridges application-level and low-level
kernel functions, which involves not only encapsulating access
to low-level functionality but also negotiating the
discrepancy between relatively high-level and low-level
programming environments. \> Most application programming
languages, for example, recognize constructs (e.g.,
Object-Orientation, or functional programming via lambda
closures/call-continuations, plus exceptions, mutable references,
and so on) outside the scope of low-level code (even \C{}). \> Effective
\q{bridge} protocols allow applications to access \CPS{} data
(as one category of low-level resources) within familiar
higher-level paradigms. \> For example, programmers already
working in an Object-Oriented environment might reasonably
find it intuitive to apply object models to \CPS{} elements, such
as individual devices. \> Driver code most likely would not
recognize Object-Oriented calling conventions directly, so
a \VM{} could be pressed into service to translate application-level
descriptions of procedure calls and their input parameters into
simpler binary packages which drivers can handle.\;\<
}

\p{\:\+In this sense, a \VM{} fitting these requirements would
simultaneously model \q{high level} calling conventions
and translate procedure-call requests between higher and
lower levels. \> I make this point in the context of
\CPS{}, but the idea would hold for many cases wherein
\VM{}s encapsulate access to some integral set of
low-level functions (at least low-level relative to
applications that would benefit from the requisite
functionality). \> Consider image-analysis: procedures
exposed via Computer Vision libraries are not directly
comparable to low-level driver code; they may themselves
be implemented in high-level languages like \Cpp{}); nevertheless,
libraries such as \OpenCV{} or \ITK{} tend to demand a
rather complex scaffolding set up to enable analytic
procedures being called. \> We could imagine circumstances
where wrapping complex image-processing pipelines in
a simpler \API{} would be convenient for
code libraries managing image series. \> For instance,
a database exposing image-processing operations through
query expressions would ideally support query-evaluation
covering Computer Vision capabilities
(going beyond obvious information one may want
to query from an image, such as dimensions and
color depth).\;\<
}

\p{\:\+Indeed, database queries, image-processing, and \CPS{} network
administration each represent plausible
use-cases wherein \VM{}s can serve as adapters
between application-style coding environments and
procedure-collections that are too specialized or low-level
to fit comfortably in application-development
norms. \> These cases can overlap, of course; database
can track or warehouse \CPS{} data such that queries monitoring
real-time device state would logically coexist with
queries against database content (presumably representing
temporally prior device info), or image databases can
support Computer Vision based queries. \> The database-query
perspective points to an interesting heuristic
analogy, insofar as packaging procedure-calls from an
application environment to a lower-level context
resembles the task of packaging application-level
datatypes and updates into records and fields natively
recognized by a persistence engine. \> Similar to how
live-memory data structures (objects, say, in
Object-Oriented environments where the relevant data
is interpreted in light of objects' polymorphic type,
which could be subclasses of their declared type)
are destructured for insertion into database sites,
high-level calling-conventions (again, Object-Orientation
provides useful examples insofar as \thisSlashSelf{}
values are represented apart from other input parameters)
need to be restructured into the (generally simpler)
forms suitable for low-level functions (e.g.,
moving \thisSlashSelf{} to be an ordinary
parameter, which may require truncating
to base-class binary layouts, and some level of type-erasure).\;\<
}

\p{\:\+Continuing this analogy, flexible database architectures
bridge application-level data types with persistent
data structures/records so that application code
can work with back-end values according
to the norms of application-context programming;
equivalently, \VM{}s can wrap low-level
functions such that they fit the profile of
high-level procedures called according to
high-level conventions (with objects, exceptions,
and so forth). \> A flexible \VM{} would play
this bridge role in multiple contexts, perhaps
allowing native functions to be registered as
kernel operations and striving to be usable
from multiple host languages: that is, from one
\VM{} we can envision encapsulating access
to a variety of procedure-collections
(examples I've cited here include functions
exposed \visavis{} database queries,
image-processing, and \CPS{} networks) and
routing descriptions of procedure-calls
from a variety of programming languages
(which might have object-oriented or
functional characteristics, or some combination).
\> To the degree that such generality
is desired, \VM{} should anticipate
integration with diverse application-level languages
preferring different calling-conventions and
procedural contracts.\;\<
}

\p{\:\+One maxim for \VM{} design, then, at least in
this sort of use-case context, is to
prioritize capabilities to model
and carry out procedure-calls described
through diverse calling-conventions, rather
than narrowing in on specific calling-conventions
which derive from a preferred programming
model (functional or Object-Oriented, say).
\> Object-Oriented conventions such as
method overrides/polymorphism and exceptions/exception-handling
might be natively expressed via the \VM{}, but likewise
functional idioms such as lazy evaluation and overloading
based on type-state. \> To clarify the last example:
procedures can potentially be given different implementations
by virtue of values' type-state at the moment of
call (essentially a more granular classification than
type-attribution itself), often mixing type-state with
value-destructuring. \> A canonical example
would be procedures operating on list-style collections,
which in one overload would accept only \i{non-empty}
collections where the last (or first) element is passed
separately, alongside a structure representing
all other elements (this convention is ubiquitous
in recursive algorithms, since the \q{tail} can then
be passed to the same procedure recursively,
becoming destructured by the calling mechanism
into its own head-plus-tail pair; of course, procedures
implemented via this strategy also need an
overload taking an empty list, which serves as a
halting-point).\;\<
}

\p{\:\+In short, an ambitious \VM{} can ideally work with
a broad set of calling styles embraced by
diverse programming styles, e.g. object-methods,
exceptions, lazy evaluation, typestates, and
parameter-destructuring. \> To this list
we might add dependent types and functional-reactive
idioms (e.g., so-called \q{signal/slot} conventions).\;\<
}

\p{\:\+As might be obvious by this point, the kinds of \VM{}s
I am envisioning bound to such requirements would
be relatively \q{high level}, closer in spirit
to Interface (or Service) Description Languages than
low-level emulators of actual machine code. \> This is consistent
with the spectrum of \VM{} technology; while certainly
some \VM{}s embrace the use-case of enabling
virtual \q{operating systems} or similar low-level
environments for portable code \mdash{} where the
execution and runtime of \VM{} instructions should
mimic machine-language steps \mdash{} other flavors of
\VM{}s are more concerned with engineering
language-neutral environments that interoperate
fluently with different kinds of programming front-ends.
\> The priority in such a case may still be cross-platform
flexibility, but the design goals emphasize
a desire for multiple origination-languages to
emit \VM{} code (through factories if not text streams)
according to protocols which are in sync with
host-language conventions (\q{host} language in the
sense that \VM{} capabilities can be embedded
via static or dynamic libraries linked against
software components, which in turn may
utilize such capabilities in different ways \mdash{} via
scripts, queries, workflow descriptions, etc.).\;\<
}

\p{\:\+Since not all \VM{} actually aspire to multi-language
support to the open-ended degree implied here,
theories informing \VM{} implementation do
not necessarily analyze data models or design
patterns targeted specifically at language
\q{agnosticity} (so to speak); a more common
scenario is that theoretical perspectives
emanate from coding paradigms which give
rise to distinct flavors of programming languages,
potentially at some level proselytizing for
favored paradigms rather than aiming for
broad generality (in the sense that \VM{}s
\i{for functional languages}, say, reflect a
general assumption that functional methodology
is in the general case a better coding style
than alternatives). \> By contrast, I
am interested here in describing theoretical
\VM{} models that remain nonjudgmental
as to which conventions are better in which
context (or to take the view that multiple
coding styles each have their own use-case
so should be supported as such). \> A rigorous
\VM{} \q{model} should have multiple
dimensions (addressing types and data-encoding,
for example), but of course an essential
concern is modeling procedure-calls and
calling-conventions, so I will sue
this topic as a starting-point for a
(relatively informal) system to
encapsulate \VM{} details in a schematic fashion.\;\<
}

\subsection{Applicative Structures and Mathematical Foundations}
\p{\:\+Theoretical (and applied) computer science often
approaches procedures from the viewpoint of mathematical
functions, essentially mapping that transform
inputs to outputs. \> Codifying the principles of
\q{functionhood} in general forms the central project
of analyses that bridge math and computers
(e.g., lambda calculus). \> In terms of mathematical
\q{foundations}, these various formulations
(including lambda calculus and its derivatives
and, for instance, Combinatory Logic) entail
strategies for clarifying what are
sometimes called \q{applicative structures},
or generic patterns involving the \i{application}
of a function/procedure to one or more
arguments/parameters. \> Since this is such a
basic phenomenon in the mathematical
realm, it is understandable that some
researchers in \q{philosophy} of mathematics
and its foundations would focus on applicative
structures (perhaps indirectly via, say, lambda
calculus) \mdash{} even though mathematical
expressions are written down according to
a wide variety of conventions (consider formulae
for integration, for ratios, for polynomials, and so forth)
we can imaging a system which translates
mathematical terms to a more systematic logical
representation (which would presumably
resemble something like \lisp{} code).\;\<
}

\p{\:\+The concerns evoked by applicative structures are
not only orthographic, however, because there is
also a \i{semantic} dimension in the sense of
spaces of functional values qua semantic entities.
\> The problem of semantically characterizing
\i{a} function or procedure (e.g., as a calculational
process, or a \mdash{} maybe time/context-dependent \mdash{}
input-to-output mapping, or the like) is one
matter, but whatever our foundational semantic
theory in this sense it is natural to extend
it via functional composition (analogous to
how linguistic semantics involves the semantics
of nouns, and verbs, but more thoroughly also
the compositional principles of verbs together
with nouns yielding sentences/propositions).
\> The set of possible applicative structures
\q{generated} by some collection of functions
(or function-symbols) is analogous to the
set of discrete functional values that can be
defined by the combination of multiple functions
wherein the results of one function applied to
its arguments becomes in turn (one of the)
parameters of a different (or, recursively, the same)
function. \> Systematically, an \q{applicative system}
would be a set of function-symbols alongside
\q{variable} symbols with a rule that applicative
\i{structures} include expressions of the form
\fxoneetc{} where \xs{} are variables,
and that for any applicative structure the modification
formed by replacing a variable-symbol with another
applicative structure is also an applicative structure.
\> Defined in terms of the close of such substitution operations.
\> applicative systems can be seen to follow a generative
pattern very similar to labeled trees \mdash{} each
node is either a symbol-node or a branch with
its set of child-nodes \mdash{} with the added detail
that labels are partitioned into two sets
(function and variable symbols, respectively)
such that left-most child nodes always have function-labels
and other non-branch nodes always have variable labels.\;\<
}

\p{\:\+The analytic purpose of applicative structures can potentially
be served by systems employed in other branches of
mathematics \mdash{} e.g., \q{term algebras},
in formal logic, or \nary{} trees in computer science,
or an extension of the \q{free magma} groupoid concept to
(sets of) \nary{} (not just binary) operations.
\> For instance, applicative structures might be seen as
term algebras if we consider only function-terms in
the context of first-order logic, ignoring predicates, constants,
or relations (constants can be treated as nullary functions, and
predicates/relations modeled via functions whose codomain
is the set of two boolean constants \i{true} and \i{false}).
\> Applicative structures may also be defined in terms
of freely generated strings in a language with
balanced parentheses \mdash{} essentially a
\q{Dyck} language (after Walther von Dyck) intersected
with a suitably restricted function/variable language.
\> In effect, applicative structures can be defined by restricting
the forms generated by first-order logic, or by regular
languages according to the Chomsky-\Schutzenberger{} representation
theorem (in formal linguistics). \> Different such formulations
could be preferred depending on which seems like an intuitive
basis in the specific context being analyzed. \> However,
although any language roughly equivalent to
\nary{} trees may be intuitively simple, it is
worth defining the precise structures one is working with
axiomatically to ensure that there is a strict isomorphism
between representations in different contexts, as opposed to
spaces which overlap but do not exactly align
(for example, balanced-parentheses strings allow for
nullary function applications which are not necessarily
substitutable with constant values; it is unclear
how to map that specific possibility to \nary{} trees).
\> In any case, my strategy here is to define applicative
structures recursively in a manner that works whether
we approach them syntactically or semantically,
and to analyze their properties mostly through a
representation based on matrices, which in turn
could potentially be used to verify isomorphisms
between these structures and alternative encodings
(here I carry out such an analysis in a graph-theoretic
context, though I propose similar methods might apply
to, e.g., term algebras).\;\<
}

\p{\:\+I'll make a couple of technical points about applicative
structures here, not so much because the mathematics
is particularly sophisticated or consequential but
so as to establish a baseline of comparison for
the graph-theoretic encoding of (generalizations of)
applicative structures I will discuss subsequently.

\defin{Applicative Systems}{for any fixed
set of \i{function symbols} and \i{variable symbols}
an \i{applicative structure} is any element of a
set which is the closure of the set of
primitive expressions (consisting of a function
symbol followed by a number of variable-symbols
which matches its arity, assuming we assign a
specific arity to each function, or else to any
number of variable-symbols) under substitution
operations wherein an applicative structure
is inserted in an enclosing structure taking the
place of a variable.  Applicative Systems
are then sets of function and variable
symbols (and possibly declarations of function-arity)
together the full set of possible applicative
structures generated on their basis.  For generality,
we can allow functions \f{} to have dynamic range
arity (multiple integers \n{} such that
\f{} followed by \n{} variables is a valid expression);
in this sense systems which do not recognize arity
limitations at all would implicitly allow arbitrary
range arity for all function symbols.\footnote{A \i{partial} applicative structure would be one
that belongs to an applicative system no subject to
arity restrictions but which is excluded when
arities are recognized, due to one or more function
symbols lacking a sufficient number of following
symbols or nested structures; partial applicative
structures in this sense can be semantically
interpreted as designating \q{meta-functions} which,
when the variables are replaced by fixed values,
reduce to actual functions whose parameters are
the missing symbols implied by insufficient arity
thresholds.  Of course, this discussion assumes
we also have a notion of \q{fixed values} being
assigned to functions as parameters).
}  Note that the generative
rules allow for zero-arity functions, whose
semantics would be procedures that yield results
even without inputs (nested structures can be
wholly comprised of one single function symbol).}

\anondefin{A given symbol may appear multiple times in an applicative
structure; the above definitions were formulated with the
idea that \q{the same} symbol in different positions
is, according to the generative rules, a different
symbol, but for clarity we can say that one
symbol can have multiple \i{tokens} in a given
structure. \> Each symbol-token has a
\i{nesting level} such that the nesting level of a
function-symbol matches that of variable-symbols
following it (that are not themselves part of a
further nested structure) and replacing a variable
with a nested structure forces the function symbol
at the left of the latter structure to have a
nesting level on greater than the replaced variable.
\> By the construction/generation process, there will always
be exactly one function-symbol with least nesting
level, which we can stipulate to be zero. \> Symbol-tokens
also have \i{positional indices} defined  such that
function symbols are assigned index zero and variable-symbols
following them (incrementing across but skipping over nested structures) are
assigned successively greater index numbers. \> Tokens are
uniquely identified by
a positional-index list whose length is
determined by (viz., one greater than) the token's
nesting-level, notating the tokens own positional-index
and also those that would be assigned to its
parent nodes (treating the structure as a tree) were
they tokens rather than nested structures). \> Positional-index
lists induce an ordering on all tokens in an applicative
structure (comparing the first number in two respective
lists, then the second as a tie-breaker, and so forth;
akin to ordering leaf nodes on a tree where
left-ward and higher nodes are prior to those
below and/or to their right). \> Applicative
structures are \i{non-recursive} if each function-symbol
has only one token. \> It is helpful to further
define non-\i{root}-recursive structures as those
where the function-symbol whose token has
zero nesting level appears only once.
}

\anondefin{Applicative structures can be grouped into equivalence
classes wherein any structure in the class would
map onto a peer structure under a permutation of
the applicative system's variable symbol-set. \> Since
we can give an ordering to the variable-set, assume
each variable is assigned a number code, which
in turn allows us to define a \i{canonical presentation}
of an applicative structure by permuting its
variable-symbols so that tokens which are
prior (in the ordering based on positional indices
just described) are assigned symbols with lesser
codes, choosing each new symbol to be the one
with  lowest code value not yet used. \> Each
applicative-structure equivalence-class is thereby
represented by one structure with
such canonical presentation, and we can restrict
attention to these structures in particular.
\> Note that the generative rules for applicative
structures have to be separated into two
groups, because (in the general case) we
have to avoid unrelated symbols with the same
\q{label} colliding according to the generative
step whereby a nested structure is substituted
for a variable. \> For a nested structure with its
own collection of variables, the structure may
need to be rewritten (cf. lambda-calculus alpha-conversion)
as an equivalent structure
(but with symbol-permutation) \mdash{} not necessarily
(indeed not usually) one in canonical-presentation \mdash{}
so that each nested symbol is different
from all symbols in the enclosing structure
(unless the symbol-representation is explicit,
i.e., the point is to model a function-application
where the same input value is copied in multiple places).\;\<
}
}

\p{\:\+Representing function-arguments via \q{variable} symbols
(which are understood to be substituted with values from
some domain in the context of \i{calling} functions)
is consistent with lambda calculus, and with the
idea that applicative structures describe valid strings
in a language codifying notions of function-application.
\> However, it is possible to develop alternative representations
of the same structures, such as De Bruijn indices
(due to Nicolas de Bruijn) which in effect model argument-places
by numeric indices rather than symbols. \> I propose
to use a similar convention based on the idea of
forming symbols which have numeric values but also
\q{colors}. \> Specifically, consider, first, characterizations
of the set of functions which may be identified
on the basis of one function-symbol \mdash{} say, \fRed{},
which assume has a fixed arity \arity{}, say, three.
\> Then \fRed{} itself could be notated \fOneTwoThree{}
(using the color red to signify function-symbols
and blue numbers for variable-symbols). \> Given
\f{} we implicitly also have a family of related functions
which are identical to \f{} but operate on different domains,
taking argument-lists longer than \f{}'s arity but
ignoring the extra values: I propose denoting these
via \fOneTwoThreeGFour{}, \fOneTwoThreeGFourFive{}, etc.
\> using \i{gray}-color numbers for discarded
extra elements. \> Conversely, we can also form
functions with arity \i{less} than \f{}'s by
repeating numbers for the arguments; e.g.,
\fOneTwoOne{} or \fOneTwoTwo{} where note that
numbers occurring multiple times are shaded in
light blue. \> I'll refer to this as
\q{red-blue-gray} notation. \> Schematically,
one can treat this as equivalent to
notation with symbols, each colored number
being just a symbol, except that the
colors allow grouping into ordered sets which
may be subject to further constraints.
\> Likewise, the function-symbols themselves
can be replaced by numbers encoding
any ordering of the list of available
function-symbols which form the core of
our applicative system.\;\<
}

\p{\:\+Suppose we start with any function-symbol list
\mdash{} for instance, as earlier in this chapter
I alluded to collections of kernel
functions which \q{seed} a Virtual Machine
wrapping access to \CPS{} networks.
\> Intuitively, it would seem to be natural
property of any computational or mathematical
(or even linguistic environment
for which a notation of \f{} itself would
be meaningful \mdash{} e.g., denotes a
possible computation, or a mathematical
function which yields a value when parameters
are fully specified \mdash{} then so too
would be variants of \f{} with greater
or lesser arity as notated (here) through
red-blue-gray strings. \> In other words,
if we take some semantic predicate \mdash{} say,
\i{describes a computation} \mdash{} then we may
want to consider the full set of functions
which meet this criterion that can be
generated from some initial \q{seed} collection;
given \f{} as a seed, notate the set of
functions (describable solely via \f{})
which describe a computation, for instance
(or replace \q{describe a computation} with
some other predicate). \> We seem to have:

\lemmastatement{Assuming \f{} designates a function
with arity \arity{}, there is a
unique list of red-blue-gray
strings which each describes
a function, mutually structurally
distinct, enumerating all functions
that can be described on the basis
of \f{} alone (including \f{} itself).\;
(\+Note that we are considering functions
generated only by extending or contracting
\f{}'s arity; a different
enumeration would involve recursive
descriptions where a call involving
\f{} yields a value which is input
to another call involving \f{}.)
}

\lemmaproof{Consider strings with exactly \arity{}
(not necessarily distinct) variable-symbols.
\> Without losing generality, we can assume that
symbols (blue-colored numbers) with
lesser numeric value appear to the
left in their first token. \> Let
\arityprimelessthanarity{} be
the number of \i{distinct} variable-symbols
in a given string; i.e., the arity
of the function being described as a
variant of \f{} (as compared to the arity of \f{} itself).
\> Then any \arity{}-length permutation of
\arityprime{} numbers (in the range
\onearityprime{}) \mdash{} restricted to cases
where the first occurrence of smaller numbers
is always before the first occurrence of
any larger number \mdash{} describes a
structurally unique variant of \f{}.
\> Analogously (distinguishing blue and
light-blue) the darker-blue-only
substrings are increasing by one
starting at one, and light-blue
symbols can be freely interspersed
except that any light-blue \nlightblue{}
has to follow the corresponding darker \ndarkblue{}.
\> For each of these strings, consider the
strings themselves and then modifications
which add the gray-color
numbers starting with \arityprimeplusone{}, and
so forth, where the gray-colored numbers
can only occur in an arithmetically (-by-one)
increasingly list. \> Each string generated in
this fashion represents a structurally
distinct variation on \f{}, suggesting
that the full set of strings generated
accordingly embodies all \f{} variations,
at least those which could be notated
by a language that encodes function-calls
through symbols representing input parameters.
}

\anondefin{I will call a red-blue-gray string
\q{lambda feasible} if it satisfies the
restrictions employed above \visavis{}
darker-blue and gray numbers.\;\<
}
}

\p{\:\+Generalizing this analysis to something like lambda
calculus requires notating function-composition,
by substituting nested applicative
structures for variable-symbols. \> For this
context I propose extending \q{red-blue-gray}
notation to include \q{black} numbers that stand in for
nested structures. \> Call a \q{red-blue-gray-black matrix}
a set of rows formed from red, blue, gray, or black-colored
numbers (unlike normal mathematical matrices the rows
need not have equal size, though if desired we
can always introduce a \q{dummy} symbol, e.g. a black
zero, to pad shorter rows on the left, yielding
something that looks like a normal matrix; likewise,
colored numbers can always be mapped on to disjoint
integer sub-sets, so we can if desired
treat red-blue-gray-black matrices as isomorphic
to normal integer-matrices).

\lemmastatement{For any fixed function-symbol and (ordered) variable-symbol
set there is exactly one (countably infinite) set of
finite applicative structures in canonical presentation
that can be generated from those symbols.
}

\lemmaproof{I'll proceed by representing each applicative structure
via a red-blue-gray-black matrix. \> The simplest applicative
structures have no nesting, just a function-symbol
followed by a list of variable-symbols.
\> Since we are attending strictly to canonical presentation,
we can permute the latter list so that
lower-numbered symbols appear to the left, and distinct
symbol-numbers are allocated incrementally.
\> Symbol-numbers in this context function similarly
to \q{De Bruijn indices} in the lambda calculus
(when formalized through de Bruijn's nonstandard notation).\footnote{See, e.g., \cite{x}
}  Lemma 1 argued that we can describe all
variations on an \f{} \i{without} composition as the
set of \q{lambda-feasible} red-blue-gray strings with
\f{} (which can be denoted by a red number as well as a
symbol) as the sole function notated. \> According
to the construction rules generating applicative
structures, we then have to consider
substituting nested structures for variable
symbols. \> Using red-blue-gray-black matrices,
encode such nesting by inserting a black-colored
number instead of a blue one, selecting
black numbers according to the rule
that lower numbers occur before (to the left
of and above) higher ones, and that the
black-number values map the index of subsequent
rows in the matrix; each row, then, is its
own red-blue-gray-black string. \> Re-encode
blue numbers in nested structures by adding
to their value the smallest offset possible
without conflicting with blue numbers in earlier rows. \> Call
red-blue-gray-black matrices subject to
these restrictions (on the black numbers
as well as lamdba-feasible restrictions for each row)
\q{lambda-feasible} matrices. \> Based on their
construction, I claim the set of such
feasible matrices is isomorphic to the
set of applicative structures, so that
the lambda-feasible red-blue-gray-black matrices
offer a systematic enumeration of all
applicative structures.\;\<
}
}

\p{\:\+There is a certain amount of mathematical bookkeeping
implicit in the above presentation, which might
obscure the fact that applicative structure are
very basic; they should indeed by seen as rudimentary
to the point of being rather uninteresting in themselves.
\> More involved systems however emerge by
relaxing certain conditions; for example, we
might consider allowing self-referential applicative
structures that express infinite recursion.
\> The \q{red-blue-gray-black} matrix form points to one
plausible avenue for modeling this process, because
negative matrix entries could be allowed
to \q{refer back} to prior rows, notating the
idea of instruction-sequences in a computing machine
looping back to prior instructions. \> Other
classical developments (often phrased via
lambda calculus rather than applicative structures
\i{per se}) involves encoding natural numbers
via repeated iterations of a single \q{successor}
function (one being the successor to zero, etc.),
such that \mdash{} again appealing to matrix-notion
\mdash{} there is a sequence of matrices encoding
the sequence of natural numbers. \> Any application
of functions \i{to} natural numbers can then
be encoded alternately as a \i{substitution}
of these special matrices for the relevant variables.\;\<
}

\p{\:\+Additional applications of applicative-structure
theory turn on the notion that
applicative structures model function-composition
semantics. \> To the degree that we can
(with suitable semantics) treat functions
as \i{values} \mdash{} as points in an ambient
function \q{space} \mdash{} then function can be
composed in various ways to generate new
such values. \> In the simplest case (functions
of arity one), at least without recursion,
composition can be
analogous to an algebraic operation \mdash{}
from \f{} and \g{} get \fofg{} (and \goff{}, which
is generally different). \> Once one or more
functions has arity two or greater, however, we
have a set of multiple composition-options
\mdash{} \fxgx{}, \fxgy{}, \fgxy{}, \fgxgx{}, \fgxgy{},
for example, are all potential compositions
of a two-arity \f{} with unary \g{}, listing
only non-root-recursive structures where \f{}
takes priority over \g{} (the full list
as such is larger, and, if we allow unrestricted
recursion, infinite). \> We can therefore
describe each form of composition via
applicative structures, yielding a more
complete description of function values' semantic
terrain than is designated via individual
function-symbols themselves (this discussion
leaves unaddressed the question of
whether there are function values that
can \i{not} be encoded via applicative structures).\;\<
}

\p{\:\+Depending on one's perspective, applicative structures
can be seen as either essentially semantic or
syntactic phenomena. \> Syntactically, we can
treat these structures as characterizing
valid strings in a language comprised of
function and variable symbols, and one
single grouping construct (via nested
structures, which syntactically take
the form of sub-terms that can be grouped
into quasi-atomic units, substituting
for the actual atoms, viz., variables). \> Certain
syntactic formations, however, also seem to have
semantic interpretations: as already observed,
the \fofg{} (and \goff{}) compositions correspond
to what are implicitly semantic relations, that
is, the composition of two functions to yield a
new function. \> Moreover, mappings
\i{between} applicative structures sometimes
appear to express semantic relations: \goff{} is
the compositional \i{inverse to} \fofg{}, and
the inverse of one (say, binary) function \fxy{}
could be notated as \fyx{} \mdash{} in short, we
can extend applicative systems to treat certain non-canonical
structures as notations for variational forms of
functions derived from their \q{base} forms by
substituting which argument is placed in which
position, the simplest example being \fxytofyx{}
(note the similar point mentioned earlier,
that partial applicative structures can be
read as designating \q{meta}-functions).
\> The fact that some applicative structures
thereby have semantic interpretations
\mdash{} even if we consider applicative
systems as essential syntactic constructions
(enumerations of valid expressions in a
certain simple class of formal languages)
\mdash{} has led some researchers to
consider applicative structures
(particularly in the guise of Combinatory
Logic) within fields as diverse as
linguistics and psychology. \> Combinatory
Logic essentially uses a set of combinator
symbols (external to both function and variable
symbols) and string-reduction rules
to enumerate all applicative structures
generated by a system's intrinsic symbol-lists:
for any structure formed from symbols
\fetcxetc{} (\f{}s and \x{}s being functions
and variables respectively) there is a
combinator \Ccomb{} such that the string
\Cfetcxetc{} reduces to the relevant
structure (where \Ccomb{} can itself be a string
of other, more primitive combinators).
\> In this context combinators play an
enumerative role analogous
to (what I am calling) red-blue-gray-black
matrices (although I personally find
combinators' reduction rules feel
more ad-hoc than the state-machine-like
interpretation one can give to a
red-blue-gray-black matrix; I'll leave the details
to a footnote).\footnote{An intuitive way to picture Combinatory
Logic is as follows: disregard the interpretation
of function/variable symbols as functions and
arguments, respective, and consider merely
symbol-strings as expressions in a formal
language. \> Extend this language
with combinator-symbols that are associated
with reduction rules that impose a
partial order on the set of permissible
strings (such strings are differentiated
by a grouping operation, potentially nested,
as well as symbol-lists). \> A canonical
example is the \Scomb{} combinator such
that \Sfgx{} \reduces{} \fxparengx{} (where \reduces{} notates
reducing to) or the \Bcomb{} that appears to connote
composition (seen as semantic): \Bfgx{} \reduces{}
\fparengx{} (or the famous \q{Y} combinator,
which can be described by the rule \YgreducgYg{}).
\> Note that \q{reduction} here, however, at least
in the mathematical presentation of Combinatory
Logic, does not have an explicit semantic
interpretation, but merely notes that
some strings come before others in the
\reduces{} partial order; moreover,
strings in the language contain
free-form admixtures of combinators, functions, and
variables, which have no apparent semantic
interpretation (the general premise being
that only \q{maximally} reduced strings should
be approached semantically). \> I find these
details to render Combinatory Logic proper
somewhat counter-intuitive, at least on
its own terms (to be fair, though,
although some papers systematize Combinatory Logic
in isolation, it is more common to adopt
combinators as merely notational conveniences
for certain constructions in lambda calculus).\;\<
}
}

\p{\:\+The linguistics-based interest
in Combinators evidently reflects the
idea that certain applicative structures
(and intra-structure morphisms) codify
compositions or modifications which express
semantic operators, and, in general,
certain applicative structure embody syntactic
constructions that are sufficiently common
or entrenched as to emerge as semantic
conventions, not just syntactic forms
(the underlying principle, articulated
for example in Construction Grammar, being that semantic
constructions originate at least in some
cases from recurring syntactic formations,
such that \i{syntax} \i{per se} \mdash{} novel
phrases, say, which rely on grammar rather
than idiomatics to signal the intended
composition \mdash{} represent forms that
are \i{not} sufficiently entrenched as to
have \q{automatic} meanings (like those
of single words, say) but instead need to be
parsed. \> In this context Combinators
are at least intuitive emblems suggesting
how syntactic entrenchment yields semantic
conventions. \> More generally, we can
also observe in the linguistic context
that the overall space of applicative structures
(over all words in a sentence, say)
can be partitioned into subspaces (semantic
entrenchment being one example, but
we can also analyze different applicative
structures as having more or less linguistic
coherence, based on criteria such as
verb-to-noun relations).\;\<
}

\p{\:\+Apart from \i{syntactic} interpretations,
Applicative Structure can also be seen from
semantic angles in more narrowly
mathematical contexts, for example when
we consider properties \i{of} functions,
such as those derived from algorithm-theory.
\> For example, if \f{} denotes a function which
can be evaluated through a calculation implemented
with guaranteed termination, then composing
\f{} with another function with the same
property yields (or describes) a different
function which also by guarantee terminates
(simply allow \g{}, say, to terminate,
then feed its value to \f{}). \> More generally,
we can take the applicative system
over a set of functions sharing properties
such as guaranteed-termination to be a
larger set with the same property. \> In this
sense we can regard applicative structures
(perhaps via lambda calculus) as part of
the backbone for analyzing different kinds
of function-spaces according to algorithm-theoretic
properties or profiles (e.g., computability,
termination, different complexity classes, and the like).
\> These potential applications, alongside
those mentioned above in the context
of recursive function theory and infinite
recursion, as well as encoding number theory/arithmetic,
constitute some of the extensions
to underlying applicative-structure theory which
have some mathematical significance.\;\<
}

\p{\:\+My concerns here are less mathematical,
so I will extend applicative
constructions in a different direction,
grounded in graph theory. \> I will
suggest that applicative structures
have natural correlates among
(directed) graphs, and that this
setting is in some ways more intuitive
and less cumbersome than the
above presentation appealing to
notions like \q{substitution} and
function/variable \q{symbols}.\;\<
}

\p{\:\+Outside of mathematics proper, there is still
some value in enumerating the full collection
of applicative structures (generated by a preliminary
function-list). \> If we treat these structures as
syntactic phenomena, then intuitively a plausible
\i{semantics} for notions of function-application
would consider formally distinct applicative
structures as semantically distinct entities.
\> The mechanisms for enumerating distinct
structures may be less important than the
mere fact of having a well-defined algorithm for
producing and screening for all valid constructions.
\> I use (what I call) red-blue-gray-black matrices
to provide these criteria: intuitively, a reasonable
semantics for applicative structures would
recognize every formation described by a distinct
such matrix as a semantically distinct, and recognize
the set of lambda-feasible red-blue-gray-black matrices as a
complete listing of all valid strings in a language
that describes functions (for some meaning of \q{function})
derived from kernel \q{seed} functions by arity-expansion (via
unused arguments), projection (arity-reduction via
reusing parameters), and composition. \> Later in the
section I will discuss more specific semantics,
but the enumeration through red-blue-gray-black matrices
yields an initial picture of the space
over which such a semantics should quantify.\;\<
}

\subsection{Hypergraph Models of Calling Conventions}
\p{\:\+The definition of function-application modeled via
applicative structures (and similarly the lambda
calculus) should be deemed insufficiently
expressive for analyses related to modern programming
languages. \> To be sure, in the above discussion
I was evasive about function's \q{semantics},
such that an incomplete account of function
\q{application} \i{per se} is arguably warranted
by generality. \> When turning to programming
languages in particular, however, we have more
detailed semantic notions to work with; for
instance, \i{functions} can be considered as
computational \i{processes} which follow
some \i{procedure} so as to (in general) derive
an output in the presence of given inputs.
\> Such an overview is still mostly intuitive
and informal, but it begins concretizing
the notion of \q{function} in \i{procedures}
which, presumably, in the general case
execute series of smaller steps in sequence
(by contrast, say, to
regarding functions set-theoretically
as mathematical relations between inputs
and outputs, each function being a subset of
the total space of mappings conceivable
between its domain and codomain).\;\<
}

\p{\:\+Given the functions-as-procedures paradigm, a first
step is to broaden the notion of applicative
structure appropriately to accommodate how
programming languages (via which procedures
are implemented) recognize multiple
\i{forms} of inputs and outputs (objects vs.
\> ordinary parameters, say, or thrown exceptions
as opposed to ordinary return values). \> To designate
these variations in forms of inputs/outputs,
I will describe procedure parameters as being
grouped into \q{channels}, which can be interpreted
as gathering a node's adjacency set into
(potentially) two or more partitions. \> I will
introduce some nonstandard terminology,
hopefully justified by the
Virtual Machine model which \q{drops out} of
the theory I'm sketching here, as a practical
use-case. \> The central underlying notion
is to designate certain graphs as \q{syntagmatic}
insofar as they model procedures call with
(in general) multiple parameters grouped
into multiple \q{channels} (an explanation for the
choice of the terminology \i{syntagm} and \i{syntagmatic},
usually encountered in linguistics, is
offered in Chapter 6 of \cite{CovidCancerCardiac}).

\anondefin{A \i{channelized in-neighborhood} (we can
drop the \q{in-} when it is clear that out-neighborhoods
are not relevant) of a directed
graph is the in-neighborhood of one vertex \vx{} where
the edges incident to \vx{} are ordered and grouped into (typically)
one or more \i{channels}. \> A \q{channel} in this context
can be considered most generally any collection of edges
(focusing on the edges themselves, not their incident
nodes; in the case of labeled graphs channels
would then be, in effect, sets of label-tokens) but
define a \i{syntagmatic channel} as a channel all of
whose directed edges share the same target node;
by default \i{channel} and \i{syntagmatic channel}
can be used interchangeably. \> The central
vertex \vx{} can be labeled with a string taken
from a prior collection of names (intuitively,
name of procedures insofar as graphs
diagram procedure-calls). \> Channels
may also have \q{descriptive} labels (which serve
to associate channels with channel \q{kinds}).
}

\anondefin{A \i{channel system}
on a directed graph is a set of criteria constraining
the legal channelized neighborhoods around any vertex;
example restrictions would be stipulations
that vertices cannot have multiple channels
with the same kind, or (in some context)
restricted to specific possible channel-kinds,
or that channels need some fixed number
or range of edges (e.g., a channel embodying
procedural \i{outputs} may be restricted to
have at most one edge). \> More detailed
restrictions can apply to multiple
channels in interaction: consider modeling
the alternative between normal procedure \i{outputs}
and \i{exceptions}: throwing an exception
precludes returning from a procedure normally,
and vice-versa. \> In terms of channels this
means that a non-empty channel representing
(normal) outputs precludes a non-empty
channel representing exceptions, and vice-versa
(more precisely, as clarified below, the two
channels cannot simultaneously have
edges out-incident to nodes with \i{non-void state}).
}

\anondefin{A \q{lambda-restricted} channel system is one
specific system which it is convenient
to name ahead of time, intended to
mimic a minimal lambda calculus. \> See the following lemma.\;\<
}
}

\p{\:\+Instead of an open-ended notion of \q{applicative
structures} we can speak of applicative
systems \i{constructed relative to} channel systems.
\> This more general sense of applicative
structures builds up incrementally from
channelized neighborhoods conformant to the
relevant channel system. \> By way of
illustration, consider a graph-theoretic
encoding of \q{classical} applicative
structures (the form embodied in lambda calculus)
as defined earlier.

\lemmastatement{Any \q{classical} applicative structure can
be modeled via channel systems with the
following characteristics: there are two
channel kinds \mdash{} inputs and outputs \mdash{}
and output channels must have exactly one edge.
\> Call channel systems subject to
these limitations \q{lambda-restricted} as
anticipated above.
}

\lemmaproof{Proceeding by induction on the generative rules
for (the kind of) applicative structures
defined above. \> The simplest case is
one function-symbol followed by some number
of variable-symbols. \> Express this via
a neighborhood with one \q{procedure} node
(standing for a procedure to be called) and a
collection of \q{argument-nodes} with edges
pointing in to the procedure-node (all these
edges grouped into one channel). \> Next, we
expand outward to encode nesting (substituting
structures for single variables). \> Consider
two channelized neighborhoods \None{} and \Ntwo{} restricted
to the channel system described in the lemma
statement (inputs plus exactly one output).
\> We can form a connection from \Ntwo{} to
\None{} by inserting an edge between
the vertex in \Ntwo{} which is adjacent
(in the outgoing sense) to the single
output-channel edge in \Ntwo{} and
one of the input-channel edges in \None{}.
\> That is, the out-adjacent vertex in \Ntwo{}
will now have two out-edges, one targeting
the function node in \Ntwo{} and one
targeting an argument node in \None{}.
\> Introducing these connecting edges
is structurally analogous to \q{substituting}
nested structures for variables.
\> I'll clarify that claim with a further definition.
}

\anondefin{The \i{syntagmatic neighborhood-set} of a
vertex is an expansion of channelized in-neighborhoods
by connecting a different neighborhood
to an initial neighborhood via an edge between
a \q{peripheral} node in each neighborhood
(i.e., excluding the vertex in-adjacent to
other nodes in the neighborhood). \> Because
each channelized neighborhood could have its own syntagmatic neighborhood-set
these structures can model nesting. \> Stipulate
that any syntagmatic neighborhood-set proper has exactly
one channelized neighborhood which does \i{not} have a
connection to another neighborhood; i.e., all of its
vertices have at most one out-edge. \> Call the central
node of this neighborhood central for the neighborhood-set
overall, and stipulate that syntagmatic neighborhood-sets
must be \i{connected} in the sense that the central
node of each component channelized neighborhood must
be connected to the central node (via paths
allowing either in- or out-adjacency).
}

\anondefin{An \i{input ring} around a syntagmatic neighborhood-set
is a collection of labeled nodes (each with distinct
labels) that are not otherwise parts of the construction.
\> Consider the ring nodes to be connected with (non-procedural)
nodes via a \q{supplication} edge (intuitively, to
represent the idea that the node is associated with a
value based on its supplier-node's label); two nodes
adjacent to the same ring node via such edges
have \q{shared supplication}.
}

\lemmastatement{Syntagmatic neighborhood-sets within a
channel-system restricted to output/input kinds
(and max-one output channels) can be unambiguously
encoded via matrices equivalent to \q{red-blue-gray-black}
matrices represented earlier for applicative
structures in canonical presentation.
}

\lemmaproof{Label procedure-nodes (i.e., center-nodes) with
\i{red} numbers based on their string labels (assigning
like numbers to like strings).
\> Edges in channelized neighborhoods
are ordered, so form matrix rows by notating the
procedure-number followed by \i{blue} numbers assigned
to argument-nodes; if an argument-node
shares supplication with a different node already
numerically labeled, adopt that number, otherwise
adopt the least available numeric label. \> This
applies only to argument-nodes which are not
connected to other nodes across syntagmatic
neighborhood-set connections. \> In the latter
case, label the \q{nested} channelized neighborhood
with \i{black} numbers (starting at one for the central node
overall and incrementing as needed) and,
for argument-nodes connected to other neighborhoods,
insert a negative integer whose absolute
value is the external neighborhood's index number.
\> Create a matrix row for each channelized
neighborhood, in order of their index numbers.
\> The resulting matrix will structurally
mimic red-blue-gray-black
matrices in the context of applicative
structures outlined above.
}

\theoremstatement{Applicative structures as presented earlier and syntagmatic
neighborhood-sets within a lambda-restricted channel
system are isomorphic to the same set of lambda-feasible
red-blue-gray-black matrices, assuming they share the same
set of function-symbols/procedure-labels.
}

\theoremproof{Lemmas () and () detail the construction of
De Bruijn matrices from applicative structures
and syntagmatic neighborhood-sets restrictively.
\> I claim that comparing the two constructions
documents that each step in the construction
would modify the resulting matrices in
equivalent ways, and so the set of
matrices generated from applicative structures
will be identical to that generated from
syntagmatic neighborhood-sets \mdash{} both are
precisely the lambda-feasible matrices
according to the earlier definition of
lambda-feasibility which precludes infinite
recursion.
}

In other words, lambda-restricted syntagmatic
neighborhood-sets are a graph-theoretic encoding
of applicative structures in the classical
lambda-calculus sense.\;\<
}

\p{\:\+I have formally reviewed \q{classical} applicative structures,
however, primarily to demonstrate that
within the context of channel systems these
structures represent only one (relatively restricted)
model of function/procedure-application, characterized
by a simplified (input/output) channel semantics
(plus at most one output per function). \> The constructions
for channelized neighborhoods and syntagmatic neighborhood-sets
can be naturally extended to more expressive channel systems,
yielding (so I claim) graph representations which are
more consistent with actual programming languages
(whereas lambda calculus models a kind of abstract
programming language for purposes of mathematical
treatment).\;\<
}

\p{\:\+There are at least two significant extensions which can be
made when transitioning from \q{lambda-restricted} channel
systems to ones that are more free-form: first,
channels can have multiple kinds (beyond just
input and output) and, second, it is possible
for the output to one procedure to be a
\i{function value} which in turn is applied to
other arguments. \> The latter scenario implies
that an output node (in one neighborhood)
can be linked (via a directed edge) to
the \i{central} node of a different neighborhood,
not just to one of its argument nodes
(in this case the central node would not be
labeled with a string denoting a function-name,
but rather would receive a function-value
from that output edge).\;\<

\defin{Syntagmatic Graphs}{Consider syntagmatic
neighborhood-sets as above, with the following
generalizations: each component syntagmatic neighborhood-set
may have channels of multiple kinds (though we can
maintain, as an optional restriction, the stipulation that each neighborhood has at
most one channel of each kind), and the central/procedural
node of a neighborhood may have an incoming edge that
passes a function value assigned to that node.
This latter possibility can be accommodated by defining a
special \q{function value} channel kind which can be
occupied by at most one edge; the non-central node
incident to that edge can be considered a special
form of argument node, one which has an incoming
edge from another neighborhood (representing a
passed function-value), but instead of this
argument node being an \i{input} parameter to
the central-node's procedure, it is a function
value representing the actual procedure which
that node iconifies.  This formation would thereby
model situations where nodes encapsulate numerous
possible procedures and the actual procedure
designated is dynamically calculated (presumably
just prior to the function-call which is
notated via the syntagmatic graph).}
}

\subsection{Semantic Interpretation of Syntagmatic Graphs}
\p{\:\+Intuitively, each Syntagmatic Graph describes a structure connecting
one or more procedure-calls, interconnected by parameter
passing \mdash{} outputs from one procedure become inputs to another.
\> Working in a more flexible \q{channel system}, however,
the connections between such calls may have more
subtle relationships than output-to-input chaining.
\> For example, multiple nodes in a syntagmatic
graph may represent the same \q{variable}
(cf. \q{shared supplication} nodes from earlier) which in turn
might be modified by one procedure, so an input given a
new value by a procedure and then passed to another
procedure is a form of inter-procedure precedence \mdash{} the
effects of the first are consequential to the second \mdash{}
even if the two are not linked by an explicit
output-to-input handoff. \> There are the kinds
of scenarios that a procedure-call semantics
reflection actual programming languages (as opposed
more strictly mathematical function theories) should address.\;\<
}

\p{\:\+Assuming we remain within the context of applicative structures
proper, the point of these formations is to represent
the idea of functions (in some sense) which take input
values. \> Notating this process via variable-symbols
allows function composition (and arity-reducing projections)
to be described; thus we have \q{substitution},
replacing variable-symbols with concrete values. \> When
all free variables in an expression describing
(potentially nested) function-applications are
thus substituted, we have sufficient information to
evaluate the function, or \mdash{} in the sense of
lambda-calculus \q{beta} reduction \mdash{} reduce
(or \q{collapse}) the
aggregate expression to a single resulting value.
\> This is the central dynamic figures by applicative
structures in their simpler, classical sense
\mdash{} input parameters yield applications which
\q{reduce} the inputs to a single output value,
that may in turn be input to other functions.
\> Chaining inputs and outputs in this sense
engenders a model of computations as graphs,
where edges encode have values travel
between applications, with multiple inputs
potentially each being outputs from
multiple precursor function-applications.\;\<
}

\p{\:\+The semantic interpretation I adopt here for \q{syntagmatic}
graphs is noticeably different than this model
(I have analyzed the specific differences in \cite{Chapter6}).
\> Rather than taking input/output as a fundamental
divide within function-parameters, I consider
more general channel systems here. \> Note that in
contrast to (more typical) graph-representations
of computation where outputs are marked by
directed edges \i{away from} procedure nodes,
in Syntagmatic Graphs (hereafter \SG{}s) all edges point \i{into}
procedure-nodes, including those embodying
\q{outputs} (\cite{Chapter6} has some comments
about why this can make sense). \> More elaborate
rationales for the notational and interpretive
differences between \SG{}s and
other procedure-models is tangential to the
current chapter, so I'll focus instead
on outlining the semantic interpretation itself.

\anondefin{Call a \i{marking} of a Syntagmatic
Graph to be an association between
some of its nodes and a collection of typed
values, against some type system. \> Markings
might be context-dependent; that is, a
graph could be subject to multiple
markings concurrently, each restricted
to one context. \> I'll say that values
are \i{bound to} nodes, but indirectly,
as explained in the following.
}

\anondefin{Syntagmatic Graph nodes can be associated with
\i{types} and \i{states}. \> For the present,
I will not rigorously define \q{types}, but
I'll comment that types are introduced
\visavis{} nodes \i{through} states.
\> For any type, each possible instance of that type
is a potential \i{state} for Syntagmatic Graph nodes,
the state of being \q{occupied} by that specific
value. \> However, not every node-state need correspond
to a type. \> In particular, nodes can have a state
corresponding to a \q{void} or lack-of-value.
}

That is, I approach the notion of \q{void} as a \i{state}
possibly evinced by nodes, rather than through type
systems themselves. \> That is, we do not need
a \q{bottom} or \q{nothing} type with some
special value selected (essentially by fiat)
to represent non-initialization or pre-initialization.
\> I find it more semantically coherent to
express such \q{nothing} states as the
state of \i{having no value}, rather than
as the presence of a construed \q{nothing} value
(and nothing-type which it instantiates).\;\<
}

\p{\:\+Against this background, the semantics of procedure-calls
can be expressed via before-and-after states
for each argument-node incident to a procedure-node.
\> The procedure's semantics is, as such,
the cumulative state-changes, or \i{change in marking state},
effectuated by the procedure. \> In a general case,
some of these changes will be void (no-value state) nodes
transitioning to having a value; these would
typically correspond to \q{output} nodes in
classical applicative models. \> However
\mdash{} accommodating scenarios like mutable references
\mdash{} nodes with the state of binding to some value
could migrate to the state of binding to a \i{different}
value. \> In general, notions such as input-versus-output
are expressed in this system \q{semantically} as a
manifestation of state-changes, rather than
\q{syntactically} as edge-direction or arrows assigning
different directions to inputs versus outputs.\;\<
}

\p{\:\+Roughly as an analog to \q{beta} reduction, I propose
the term \i{digamma} reduction to express
marking-state changes due to a procedure-call.
\> A digamma reduction is the cumulative state-change
in all nodes affected by the procedure (or, seen
syntactically, all nodes adjacent to the procedure-node).
\> For multi- (channelized) neighborhood \SG{}s,
digamma reduction happens in multiple stages, each
contextualized to a single neighborhood. \> Note
that I like the term \q{digamma reduction} partly
as an oblique reference to \q{sigma} calculus
(an object-oriented extension to lambda calculus,
and the Greek digamma numeric symbol looks like a enlarged
lower-case sigma) and partly because \q{gamma} is a common
symbol for graphs, so \q{digamma} suggests \q{two graphs},
or a computational interaction between one graph
assembling a procedure-call and one graph implementing it.\;\<
}

\p{\:\+Semantically, then, \SG{}s model digamma
reductions \i{in sequence}, each localized to
individual channelized neighborhoods. \> This
idea can then be extended to sequences of
\SG{}s in turn. \> We are getting closer
to representing actual computer code; one further
step might then be incorporating something like
\q{stack frames}. \> Assuming that \SG{}s
are represented \i{in sequence} and moreover
that such sequences occur in the context
of a \i{symbol scope}, or an environment
where string labels can serve as \q{carriers}
for typed values; I'll sometimes refer to \q{\SG{}-scopes},
scopes contextualizing \SG{} sequences wherein symbols
designated carriers which (when in an initialized
state) hold concrete values. \> In particular, assume that
carriers can reveal \q{carrier states}
reciprocating the states defined
on nodes earlier: for any type there is a
spectrum of states equivalent to each type-instance,
but there are also states involving
\i{lack} of typed values (the generic example
of such a state I'll call \i{pre-initialized}).
\> The states available for carriers can be called
\q{type-based} in that they are organized around
typed values but include non-type states as well.
\SG{} nodes can then take on states
by associating with carriers such that
carrier-state propagates to the nodes.
\> Nodes are in \q{shared supplication} if
their carriers are syncronized to be
perpetually in the same state (this can be
considered a refinement of the earlier definition).
\> Symbols in \SG{}-scopes can derive from multiple
sources: assume that scopes' smybol-list can be
grouped into \i{declared} symbols internal to the
scope, \i{argument} symbols bound to
procedures' signatures, and (potentially) \i{global}.
\> or perhaps \q{ambient} symbols
shared among multiple scopes (I'll use
\q{ambient} for the general case and \q{global}
for analogs to \q{global variables} in
typical programming language).

\defin{Tripartite Scope}{I'll call \SG{} scopes
\i{tripartite} in that they are mappings from symbol-names
to carriers (or groups of state-syncronized carriers)
with type-based states, and symbols can be
classified as \i{declared}, \i{argument},
or \i{ambient}.  The semantics of these groupings
will be clarified below.}

\anondefin{I will say an \q{
\SG{}-described procedure}
(or just \q{procedure} when the \SG{} topic is obvious)
is an \SG{}-sequence unfolding in a tripartite scope
where argument-symbols are bound to values
held by carriers in a \i{different} procedure (or,
recursively, the same procedure in a different context,
viz., different symbol-bindings). \> Such
interactions between \SG{}-described procedures
models the semantics of procedure calls.
\> Of course, procedure-calls in turn are modeled by
channelized neighborhoods in individual \SG{}s.
\> Specifically, channelized neighborhoods embody
procedure-calls wherein peripheral nodes'
carrier-states in the \i{calling} neighborhood
become the basis for initializing carriers
in the called procedure; specifically,
the carriers bound to argument symbols in the
latter's tripartite scope. \> The switch
in execution context from the calling to the
called graph is analogous to binding symbols
to values in process calculii, or beta-substitution
in lambda calculus, except that carriers'
connections (via their argument nodes) to
procedure-nodes is organized via channels,
which can affect procedure-call semantics.
\> For example, it may be stipulated that
carriers in channels modeling \q{outputs} are
understood to be in pre-initialized state
for the duration of called procedures'
execution, and attempting to utilize
(e.g., read values from) such carriers
\i{within} the procedure is a logical error.
\> There are various ways of enforcing this
kind of restriction, of course,
but channels offer a convenient
representation of the relevant
constraints viewed under a semantic interpretation.\;\<
}
}

\p{\:\+We now have a semantic interpretation of channelized
neighborhoods: each such neighborhood embodies a
procedure-call, which entails binding call-site
carriers to procedures' argument symbols and
then carrying through other procedure
calls notated through \SG{}s associated with
the \i{called} procedures. \> When they are
finished, the original carriers will (potentially)
be in a modified state, so the called
procedure effectuates a \q{digamma reduction}
on carriers around the calling site.
\> State-changes then propagate across
channelized neighborhoods insofar as
values from output-like channels
in one neighborhood are handed off
to input-like channels in later-executed
neighborhoods. \> Each \SG{} has a
\q{top-level} neighborhood which is the
last to \q{execute} (viz., compel a
procedure-call) or, from a different
perspective, the root of a directed acyclic
graph showing carrier hand-offs and shared
supplication across channelized neighborhoods.
\> Note that output channels on the
top-level neighborhood cannot be
connected to other neighborhoods
in the form of procedure outputs become
inputs to a subsequent procedure
(there is none) \mdash{} so either
top-level procedure-calls are
useful solely for their side-effects
or the carriers in their output-like
channels are bound to symbols in the
current scope. \> In the general case,
I'll say that an \SG{} is \i{anchored}
by one or more symbols if those
symbols (strictly, the carriers
associated with them) acquire values
from the \SG{}s top-level neighborhood
(particularly from the channels
there which obey output-like semantics).\;\<
}

\p{\:\+The mechanism just described constitutes
how \i{declared} symbols in an
\SG{} tripartite scope can acquire values;
they become initialized from procedure-calls
enacted \i{in} the procedure, rather than
from arguments passed \i{to} the procedure
(or from an ambient space of values
visible from multiple scopes). \> Once initialized,
declared symbols can then supply values
for procedures called \i{inside} a procedure
\mdash{}that is, an argument-symbol in one
procedure can derived from a declared symbol
in a calling procedure. \> It is reasonable
to assume in the typical case that all procedure-symbols
(whether declared, argument, or ambient) hold
values which originate from a \i{declared} symbol
somewhere, so the anchoring-points
where declared symbols are first initialized
represent precisely the points in any holistic
collection of \SG{}-described procedures where
new values are \q{introduced} into the system.\;\<
}

\p{\:\+This last point has implications for how we
theorize \i{typed values} as well. \> Implicitly
here I have assumed that we are operating in a
strongly-typed system where types and
values are interconnected: any value is a
\i{typed} value (the instantiation of a type) and
any type exists by virtue of its possible values.
\> I assume that in the general case types are
not equivalent to their \i{extensions},
that is, to any fixed set of values.
\> There are indeed some types whose extensions
are fully \q{enumerable}, so we can specify
exactly how many possible instances a type has:
the type corresponding to signed one-byte
integers, for example, denotes precisely
the set of numbers at least -128 and at most 127
(the analogous range for \i{unsigned} bytes
is 0-255). \> However, for types such as
\i{lists} of integers, the extension cannot
be specified \i{a priori}, in the sense that
it is impossible to know from a specific
computing environment (at a specific moment
in time) whether or not a particular
value (here, a particular list of integers)
which logically fits the type's criteria
can in fact be represented. \> The size of a list
which may be instantiated is limited by
factors such as the computer's available
memory; in theory, the extent of a collection-type
like \q{list of integers} is infinitely large,
because there is no rule to group
all potential such lists into a finite set
(if there were, take the longest list, an
an integer to the end, yielding a new
list which has no reason not to be included in the
first place). \> I assume moreover that
we are working in a \i{non-constructive} type
environment where we cannot consider types'
extensions as a logical construction abstracted
from computational feasibility in concrete
computing environments. \> We can discuss
types' \i{hypothetical} extension, which may indeed
be infinite, but I assume we need some
other mechanism to manage uncertainties
regarding types \i{actual} extension (in
some contexts, such as lazy-evaluated lists
or ranges, working directly with a logical
model abstracted from actual realizability
is appropriate, but in contrast to
functional-programming theory, for example,
I do not take such \q{constructive} formalizations
as an essential aspect to the relevant
type systems).\footnote{See \cite{CCC}, chapter 4, for a more
extensive analysis of non-constructive
type systems and why assuming
that types are non-constructive in general
(rather than the opposite) is practically appropriate.\;\<
}
}

\p{\:\+So as to give a rigorous semantics to \i{non-constructive}
types, instead, I leverage the interconnections
between types, symbols, and \SG{}-nodes insofar as
we are working in an \SG{} context. \> Recall the
heuristic that all symbols acquire values that
originate from declared symbols, which in turn
have a specific initialization-point
(although some symbols might be initialized
by non-constant reference, passed into a
procedure for initialization rather than
being bound to a procedure-result, the
values they are bound \i{to} in this
case will itself be determined
by an anchoring-initialization, either
in that procedure itself or some
other procedure called in turn; so,
even if not all declared symbols
receive their values via anchoring,
all declared symbols receive values
which \i{originate} from an anchoring
somewhere). \> In other words, when we say
that a symbol has some type value,
we mean that a symbol's value is
derived from an initialization from anchoring
point, where a declared symbol's value
was set via digamma reduction at an anchoring
point \mdash{} the symbol was associated with a node
which transitioned from a pre-initialized to a
type-bound state as a result of binding to nodes
in an output-like channel. \> Symbols may be
\q{re-anchored}, initialized to new values
due to anchoring points, but (assume we
work in a fixed-type environment) symbols
and carriers retain the same types
once they are initialized, so long as they
remain in an initialized state. \> Types
themselves, then, can be construed
in terms of continuities in carrier-state: once
a carrier is initialized to be in a
state determined by a type-instance, it will
remain in states determined by the same type
so long as it is initialized at all; and
all initializations can be traced to
anchoring-points, together with
synchronization among states of distinct
carriers. \> We do not need to theorize types' logical
extensions in general, because we aren't concerned
with the totality of a type's possible
values, only with the states directly
associated with the type at specific
anchoring-initialization sites.\;\<
}

\p{\:\+Many programming languages distinguish between \i{constructors},
which are intrinsic to a type's definition, from other
procedures that return values of a given type.
\> To the degree that this distinction is in effect,
any value can be traced back specifically to a
\i{constructor} for its type. \> To be sure,
a value might be bound to the output of a
\i{non}-constructor procedure, but in
that case this return value will have
been formed initially by a constructor
proper called in that procedure, or perhaps
in a procedure it in turn calls, iteratively.
\> In a channel system we can leverage a comparable
distinction be defining a special kind of
channel, which functions like a normal
output except that procedures whose signature
includes such a channel are considered
to be constructors, or analogous to constructors
(in \cite{x} I used the term \q{co-constructor}
to suggest allowing procedures being analyzed as if
they were constructors even in programming
contexts where, for technical reasons, they
would not be classified as such according to
a programming language whose code is being analyzed).
\> For generality I'll call procedures with
constructor channels \q{co-constructors} with the
assumption that different environments may be more
or less lenient in conditions where co-constructors
may be declared (only in code specifically
defining a type, say, in contexts where a narrower
policy is warranted, but without imposing restrictions
along these lines \i{a priori}).\;\<
}

\p{\:\+The significance of co-constructor channels
is that they allow us to pinpoint the
specific anchorings where new typed values
are assigned \i{ab initio}. \> Assuming an
\SG{} system uses co-constructors,
for any carrier in a type-bound state
we can be sure that the value traces back
to a digamma reduction on a carrier
specifically situated in a co-constructor
channel. \> As such, our semantics
for types may not give a set-theoretic
account of types' extensions, but we
do have a mechanism for locating the
origination point for all
typed values. \> In particular, values
are acquired via carrier state
changes localized to co-constructor
channels; such a change only occurs
when a procedure has completed with an
initialized value in that channel
available to be bound to its
eventual symbol. \> Any value
therefore witnesses the fact that
a procedure with a co-constructor
channel completed and
triggered a state-change accordingly.
\> If a type system is designed to
recognize types as (associated with)
certain contracts, such that
a type being instantiated
confirms certain properties
about the value thereby
manifest, we can recognize
such contracts in an \SG{}-style
system so long as co-constructors
adhere to type-specific contracts
in the course of depositing
values in a constructor channel.\;\<
}

\p{\:\+Once a carrier with a given type is initialized,
its value can of course be handed off to other
carriers assigned the same type (or potentially
a supertype thereof, or some other
related type wherein casting is possible).
\> Channel systems and \SG{} representations
do not preclude polymorphism \mdash{}
specifically, the labels assigned to procedure-nodes
need not uniquely identify one single
procedure available to be called, but may
instead provide a premise from which to
select one of numerous available
procedures, each with the same name \mdash{}
so type resolution may come into effect:
if there are multiple candidate procedures,
the one whose signature best matches
the procedure-calls argument types is
selected. \> In the context of channels,
such disambiguation extends to recognize
different channel-kinds, which become
part of a function's signature. \> In
effect, each node in a channelized
neighborhood has a type (based on
its carrier-state) that might be
matched against candidate procedures'
signatures. \> Types therefore a
consequential primarily in
the context of overload-resolution.
\> The issue here is not types' extension
\i{per se} (two types having the same extension
would have no bearing on their
appropriateness for matching
call-site neighborhoods to procedure
signatures; and co-extensive types
are no interconvertible unless such
casts are implicitly declared according
to the same conventions as casts between
types with different extensions). \> However,
since successfully matching call-sites
to signatures implies that all
affected carriers' types are suitably
aligned, whatever guarantees on
carrier-states are implicit in
type-bound states becomes transferred from
the calling to the called procedure, insofar as
the formers' carrier-states carry over to
the latter's argument nodes. \> Therefore,
carrier-states associated with being initialized
according to a given type propagate
from co-constructor anchorings across all
subsequent procedure-calls, with type-specific
guarantees propagated alongside.\;\<
}

\p{\:\+In effect, types' semantics in such a framework
is based on the premise that once there
is an \i{originating} initialization
via digamma reduction in a co-constructor
channel then there is one specific carrier
whose state is thereby changed, and
subsequently (presumably) a series of
further procedure-calls where that
originating carrier's state become
synchronized with subsequent carriers
in subsequent procedure-calls.
\> Each type is essentially a premise
warranting the propagation of
this shared carrier-state: the type's
presence as a condition on procedure-signatures
or on declared symbols indicates that a
carrier whose state is synchronized in
accordance with such propagation would
be in a valid state according to the
expectations of the procedure
which uses the corresponding value.
\> For the sake of discussion, I'll
refer to this approach to the
issue of type-semantics as a
\i{propagation semantics}, in contrast
to a \i{set-theoretic} semantics
which would construe types' semantic
interpretations in terms of their
extensions, or a \i{constructive}
semantics which would read type
semantics through the logic of
constructive patterns that could give
rise to values of a given type.
\> The essential point about
propagation semantics is that
type contracts would be checked
at origination points (to the degree
that they are in effect) and must be
presumed to have been enforced whenever
there is an originating digamma reduction
in the first place, and that subsequently
any use of a type as a polymorhpism-disambiguating
device should proceed under the assumption that
any carrier-state inherited as synchronized
with the state of an originating carrier
in this sense should be deemed a valid
state for the called procedure's purposes.\;\<
}

\p{\:\+This discussion regarding type semantics has passed
over some non-trivial details, such as issues
of type casting and inheritance, which I will
briefly address in the next subsection.\;\<
}

\subsection{Syntagmatic Graph Sequences as a Virtual Machine Protocol}
\p{\:\+A plausible objection to the above presentation is
that it is arguably burdened with a nontrivial list
of special terminology and representational
conventions to define constructions which are
not radically different from traditional
notions of, say, stack frames and lambda-abstraction.
\> Beyond just expanding from alternating inputs and outputs
to more flexible \q{channel} systems, I have
constructed a semantic interpretation for
these system based on formulations such as
type-based states and channelized neighborhoods,
without clarifying the theoretical merits of
this overall semantic presentation. \> My contention
is that the full semantic details (not just, say,
extending applicative structures to richer
channel systems) become valuable in the
context of Virtual Machines. \> I will
attempt to warrant this claim by addressing
\VM{} implementation, in particular,
to conclude this chapter.\;\<
}

\p{\:\+I am representing Syntagmatic Graphs as hypergraphs,
partly because \i{channels} representing a grouping
operation (gathering multiple edges, by analogous
to hypernodes being sets of nodes \mdash{} although
channels are not identical to hyper\i{edges},
wherein \i{one} edge spans multiple nodes), although
I leave open the possibility of \SG{} nodes
having internal structure (i.e., becoming hypernodes).
\> However, such hypergraph representation is
mostly heuristic; technically, I consider
hypergraphs to be essentially a visual shorthand
for \i{virtual machine} constructions. \> Here I appeal
to comments I made in Section 1: a hypergraph
(or, in essence, any strututred representation)
essentially iconifies the series of steps
needed to construct the graph in its precise state.
\> For example, focusing on channelized neighborhoods,
the elements of such structures \mdash{} procedure nodes,
peripheral nodes, and channels \mdash{} correlate with
operations setting out these details sequentially
(identifying a procedure node, e.g., by label;
identifying a channel-kind to serve as a
context for subsequent peripheral nodes; initializing
the latter nodes; switching to a different channel-kind
as needed; and so on). \> This outline could be
expanded to Syntagmatic Graphs in general by aggregating
multiple channelized neighborhoods and then
asserting where carrier handoffs (via cross-neighorhood
edges) occur, to \SG{}-sequences by constructing
multiple \SG{}s along with their respective anchorings;
and finally to procedure-descriptions by enumerating
declared, argument, and ambient symbols in a
tripartite scope. \> From this perspective,
\SG{} procedures can be seen as compact representations
summarizing sequences of \VM{} operations, or
perhaps as configurations which guide the construction of
\VM{} procedures by defining and end-goal toward which
\VM{} operations approach incrementally.\;\<
}

\p{\:\+Ordinary \VM{}s, of course, depend on such concepts as
stack frames and procedure calls; one of the fundamental
operations (or sequences thereof) typical \VM{}s
carry out involves push input values onto a stack
and then redirecting to the address of a procedure
which uses them, at least insofar as the \VM{} in
question emulates assembly code to provide a
runtime for higher-level programming languages.
\> Modeling procedure-calls via hypergraphs, and
in particular \i{channels} (in the sense I propose here)
adds representational parameters to this model,
by analogy to how hypergraphs present a richer
expressive tableau than other metamodels for
encoding data structures. \> The benefits of added
metamodeling detailing the procedure-call
context actually help illustrate why
expressivity pays dividends with respect to
data structures (e.g., in query evaluation) as well.\;\<
}

\p{\:\+The primary rationale for \VM{}s is to execute code, of course
(whether scripts, queries, or some other programming
category that does not feat neatly into one or the
other, such as workflows, or one or both ends of
remote-service functionality). \> Secondarily, \VM{} compilation
can support static code analysis even if the
intermediate code is not actually run. \> Code analysis factors into
execution as well, at least to the degree that a
\VM{} can internally support runtime checks and
guarantees \mdash{} insert debugging breaks where certain
conditions are met, preventing code from running
in certain circumstances, allowing procedures
to make contract-like assumptions (stronger than
type-checks alone could enforce), and so on.
\> For example, stipulations that values should fit
within a fixed range (narrower than theoretical
possible via their types) could potentially
be verified or implemented by weaving
\q{gatekeeping} code into \VM{} operation-sequences.
\> In short, \VM{}s acquire more flexible capabilities
\mdash{} they present opportunities to implement Requirements
Engineering style features, even if only via
add-ons \mdash{} to the degree that they present
op-sets and data models configured for static
and dynamic/runtime analysis. \> These
observations motivate my proposals to augment
the modeling parameters for \VM{}s (in procedure-call
contexts) via formulations such as
channels and carrier-state.\;\<
}

\p{\:\+To be sure, a suite of static-analysis capabilities
is endemic to traditional (e.g., stack-based)
\VM{}s as well (reachability, control paths, initialization
guarantees, etc.). \> I would suggest, however, that
future generations of \VM{} technology will seek
a broader scope \visavis{} contexts where cde analysis
is applied, considering \GUI{} programming,
Cyber-Physical Systems, multi-modal front-ends,
\AI{} integration, and other capabilities that
may be unified under the rubric of \q{Industry 4.0}.
\> Future \VM{} architecture may be optimized
for the \i{intersection} of such User-Experience and
multi-modality concerns with traditional
static analysis and Requirements Engineering.\;\<
}

\p{\:\+Consider the case of Channel Systems. \> According
to the \SG{} model I have outlined here,
procedure-calls are represented via channelized
neighborhoods, and in such neighborhoods
all edges incident to a central (\q{procedure})
node lie within a channel. \> In \VM{} translation,
this configuration implies that procedural
stack frames are split up into multiple channels,
so that whenever an argument is \q{pushed} as a
parameter for some near-future call this
action occurs in the context of a specific
channel-kind. \> As a result, argument-push
operations can be analyzed in terms of semantic
protocols specific to currently active
channel. \> For example, \VM{}s might want to
add extra information in the
context of channels representing \q{message receivers}
in Object-Oriented contexts (which
I have elsewhere called \q{sigma} channels, essentially
the \b{this} or \b{self} of languages like \Cpp{}, \Java{},
and \Ruby{}), particularly when such channels
contain more than one node (insofar as most languages
providing kernel functions accessed from a \VM{}
do not allow multiple \b{this} objects; the
demo code for this part's chapters has examples of
how to emulate multi-sigma calls in \Cpp{}, but
in general it would presumably take extra
effort to map multi-sigma \VM{} calls to the underlying
native functions). \> Moreover, channel-specific
semantics can be implemented on an extensible
basis, allowing the \VM{}s to be augmented
with special-purpose channel kinds which supply their
own functionality for managing (in effect) stack
frames when such channels are active.\;\<
}

\p{\:\+Furthermore, channels can act as a grouping mechanism
tieing together procedure arguments which are logically
interconnected (to a greater degree than merely co-existing
as arguments). \> An example would be unit/scale-decorated
types: suppose a function calculates a formula which
requires two arguments with the same dimension and measurement
units (kilometers, say) plus a third argument which
is just a scalar; evidently the first two arguments are
mutually constrained in a fashion distinct from the third.
\> Or consider a calculation with a scalar plus two mathematical vectors
which should have the same length \mdash{} the latter condition
might be modeled with a dependent-type construction on the second
vector, or some runtime check for the two vectors together,
in either case demarcating the vectors as paired arguments
distinct from the scalar. \> One strategy to satisfy these
runtime use-cases would be to implement channel semantics
where special guarantees (finer than type-checking alone)
are enforced while the channels are being populated.\footnote{In this case the one-channel-kind-per-neighborhood restriction
might be relaxed, since mutual connections could exist
between (say) some input parameters and not others; or
one could adopt something like \q{subchannels} which
semantically refine the channels around them.\;\<
}
}

\p{\:\+Doubtless, some of the features enabled by channels
could be achieved via other means.\footnote{For example,
aggregating input parameters as mentioned last paragraph
might be achieved alternatively via explicit
dependent typing or via smashing tuples into
single (aggregate) arguments. \> However, dependent
types are notoriously difficult to implement
in the context of Software Language Engineering,
and the latter alternative could be syntactically
unwieldy (if done explicitly in source code)
or complex in its own right to implement
(if done behind the scenes).
}  Channels,
however, present a convenient interface for
organizing the range of functionality
entailed by the relevant channel semantics.
\> Insofar as every argument-node in an \SG{}
occurs in the context of a channel, before
any such nodes are set in place the \VM{}
would first construct a channel of the
relevant kind; that is, there is a specific
operation to \q{open} a channel given its
kind. \> This operation is therefore an \VM{}-site
(for static or dynamic analysis) that
can be targeted by extension code enforcing or
examining channel semantics. \> Once channels
are opened, operations exist to
indicate the type and value-source for nodes
added to the channel. \> Because these latter
operations always occur in the context of a
specific channel-kind, they can be filtered or
re-implemented based on the channel kind
in effect, so extensions could modify the
treatment of certain channels while preserving
the underlying \VM{} implementation in most
cases. \> For example, a \VM{} extension could
modify exception-handling protocols by
re-implementing operations for inserting
nodes into (or, on the call side, initializing
carriers in) special \q{exception} channels,
which are partitioned from \q{ordinary}
output channels (given the obvious behavioral contrast
between exiting via exceptions versus normal
returns). \> Procedures typically involve multiple
channels, so there are \VM{} operations
for \q{closing} one channel and opening another
of a different kind, which results in subsequent
operations occurring in the context of a
different semantics. \> Explicitly introducing
channels as part of the underlying \VM{} machinery
allows variegated channel-semantic protocols
to be implemented in an organized and extensible manner.\;\<
}

\p{\:\+My earlier comments discussed overlapping concerns
of code-analysis with \AI{}, multi-modal interface,
and other \q{Industry 4.0} concerns. \> Last paragraph's
discussion regarding (for instance) special-purpose
channel semantics perhaps does not obviously
connect the idiosyncratic constructions endemic
to \SG{} representation with concerns in the
latter sense, so I'll try to present a case
more concretely. \> Consider first the
issue of \q{reactive programming} in \GUI{}
contexts. \> Analysis of procedures as a series
of calls to other procedures \mdash{} what I am
calling \SG{}-sequence descriptions \mdash{} fails
to directly address how call-sequences
in this sense fit into over application execution.
\> In general, an application is not a one-dimensional
\i{program} which simply executes some sequence of
operations and then terminates. \> Instead, applications
construct a graphical and runtime environment
and then wait for user-initiated actions,
responding accordingly, and resetting to a
passive state awaiting further user actions.
\> Users signal their intention for applications
to take specific steps via interactions
which can generically be called
\i{gestures} (e.g., typing something
via a keypad, or clicking somewhere via
a mouse). \> Gestures in turn are presented
to application code as \i{signals} that
are \i{handled} by implemented procedures.
\> The overall programming model entailed
by composing applications as collections
of procedures poised to handle signals
\mdash{} rather than fixed operation-sequences
designed in advanced \mdash{} is generically
called \i{reactive} programming. \> Such a
programming model introduces a variety of
issues for \VM{} implementation, to the
degree that \VM{}s would be used to model/analyze
or execute reactive procedures.\;\<
}

\p{\:\+In particular, the central idea of reactive
programming is that certain procedures
are called (or initiated) in response
to signals (typically those
due to user gestures, though certain
signals may be prompted by non-user
changes to the current environment
which might be relevant for applications,
such as a sudden loss or gain of
internet connectivity). \> Procedures
in this sense are not called \i{from other
procedures}, so the normal analysis of
procedure-calls in terms of stack frames
being transferred from one site to another
has to be modified. \> A canonical approach to
reactive programming involves \q{signals}
and \q{slots}, with the idea that
instead of one procedure directly calling
calling another, procedures instead emit
\q{signals} that are \i{connected} to
other procedures, which in such contexts
become \q{slots}. \> Unlike hard-coded
procedure-calls, signal-to-slot connections
can be dynamically altered, created, or
suspended. \> Such a mechanism depends on a
centralized routing component
to observe when a signal has been emitted
(for instance, added on to an \q{event queue})
and transfer control to one or more
slots registered as connected to that signal.
\> Insofar as a centralized processor in this
sense is active, it can also (in typical
application-runtime frameworks)
receive signals originating \i{outside}
the application, i.e., resulting from
external conditions apart from one
\i{procedure} emitting a signals. \> Typically,
such external signals would result from
user-generated events, such as clicking a
mouse button or moving the mouse.\;\<
}

\p{\:\+In effect, application code based on signals and
slots includes some procedures which are
called because they are registered as \q{slots}
whose signatures match signals that
may arise from \i{outside} the application
property. \> These procedures serve as \q{entry points}
where operations specific to the application
originate. \> In other words, applications
are environments which, after an initial setup,
wait in suspension until external signals
initiate a chain of actions in response.
\> The details of such signals cannot be known
ahead of time \mdash{} for example, one cannot
say which user gestures will occur (whether
the user first types something, or moves
the mouse, or clicks the mouse, etc.) nor
their specific details (which keyboard keys
are pressed, which mouse buttons are clicked,
where on-screen the mouse-cursor is located
in the latter event, etc.). \> Application
code therefore needs to prepare for
multiple forms of external signals,
and to analyze their properties to respond
correctly (in accord with user intentions
and/or design requirements). \> A left mouse click
with the cursor hovering over one \GUI{}
element typically signifies a different user
intent than right-mouse clicks over a
different element, for example.\;\<
}

\p{\:\+The unique characteristics of reactive programming
have numerous consequences for \VM{} design.
\> First, note that applications typically
implement many procedures so as to possess
requisite capabilities to handle user actions.
\> When and whether a given procedure is called
depends on user pragmatics; for instance, a
procedure involved in saving a file (at least a
file users know about, as opposed to,
e.g., a database-related file storing configuration
information) would only
be called in circumstances where users signal
their desire to save files they are
currently working on. \> A well-organized code base
will specify which procedures could
potentially be called as the \i{initial} handler
responding to external signals. \> This information
makes it possible to factor in external
signaling conditions during code-analysis.
\> For example, suppose file-saving is disabled
for some reason (e.g., the current user does not
have permission to modify the local file system).
\> In that case, procedures which are \i{only} called
within a call-chain leading from external
signals specific to saving files would become
effectively unreachable. \> Notions such as
reachability and execution paths have
to be evaluated in the context of
procedures registered as external signal-handlers.\;\<
}

\p{\:\+Consider a scenario where an application, being upgraded,
is redesigned to support two different file-handling
models: one for local filesystems and one for cloud
storage. \> Certain procedures (e.g., one to check
whether the file has been modified since the time
of last save) may be relevant for both scenarios;
others would only be active in contexts where
cloud-saving is possible (an open internet connection, say)
or, respectively, local file-system access. \> Introducing
new procedures to manage the cloud-storage case
alters the applications inter-procedural \q{connectivity},
potentially requiring analyses to be updated
with respect to conditions wherein a certain
procedure might be called (or might be
unreachable). \> The fact that such information
\i{about} applications is subject to change
(insofar as applications are continually
refined or redesigned) indicates that information
\i{about} application code should be managed in a
systematic fashion. \> This might be
done through \VM{} representations directly,
or at least source code and/or documentation
\i{available} to \VM{} compilers can draw
meta-data from such sources. \> In other words,
we can assume that applications are engineered
in coding environments where information
(including, for example, which procedures
play the role of external signal-handlers)
is detailed with enough rigor to be
read by \VM{} compilers and/or runtimes.
\> For example, \VM{} code could then internally
incorporate representations related
to reactive control flow and \GUI{} objects
which iconify pragmas for initiating
application actions from users' points of view.\;\<
}

\p{\:\+Analogous to \q{constructor channels} as
special-purpose sites notating the
origination of typed \i{values}, a
channel-based \VM{} could similarly support
special-purpose \i{input} channels
which carry external-signal data
(we might call these \q{reactive} channels).
\> In the same way that \q{co-constructors}
are declared by assembling constructor
channels, external-signal handlers could
then be identified through the presence
of reactive channels. \> This makes it
easy to identify all execution points
where external signals could
trigger procedure-chains: simply observe
for whenever a reactive channel is \q{opened}
during the course of building a
channelized neighborhood.\;\<
}

\p{\:\+External signals (as initiators of procedure-chains
that disrupt applications' passive \q{event-loop}
states) offer one example of how reactive programming
and \VM{} design intersect, but there are
related scenarios that could also be mentioned.
\> For example, applications' \GUI{}s are typically
seen as a \TwoD{} visual extent populated with
viewable objects that are simultaneously
spaces to represent pieces of information to
users (e.g., text, via readable characters;
or numbers, via printed characters or indicators
like dials and sliders; or graphics, via image
or \ThreeD{} displays) and origination-points
for user gestures (e.g., scrolling on a slider
to increase/decrease a value). \> My above comments
touched on the latter capabilities, but
interpreting user gestures depends on the
information currently presented through the
relevant \GUI{} control. \> As such, it is
useful to track systematically how
\GUI{}s are populated with data. \> A useful
maxim is that each class representing
distinct \GUI{} controls should be
paired with a separate class representing
the data which is visible within
such controls. \> This is straightforward
if a control (or in general a \GUI{} \q{gadget})
indicates one simple quantity (consider a slider
that adjusts the zoom-level for viewing an image),
but even more complex \GUI{} areas whose display
is spread over multiple subcontrols can be associated
with a multi-field datatype. \> Two datatypes
are then closely interconnected: one represents
some data-aggregate from a computational
perspective (ensuring that all fields are properly
initialized, packaging the data for persistence
or serialization, and so forth) while the
other translates the same information into
visual indicators for user interaction.
\> To the degree that \GUI{} and application-level
datatypes are closely aligned, \VM{} code
can be annotated to mark and leverage such
alignment.\;\<
}

\p{\:\+In general \mdash{} continuing these \GUI{}-related
examples \mdash{} inter-connections between \GUI{}
components, user actions, and the datatypes
shown and affected by either tend to appear
in multiple contexts. \> Consider (as above)
an image zoom level, indicated (and adjusted)
by a slider-control. \> It is not uncommon for
zooming (in and out) also to be initiated by
small arrows adjacent to a slider, or by
context-menu options on image-displays themselves,
or by keystroke sequences like control-plus either
plus or minus (holding the \q{control} key and
tapping the \q{plus} or \q{minus} keys to increase
or decrease zoom). \> Presumably, altering zoom levels
via these other gestures should cause the
zoom-slider to be adjusted propertionately.
\> There are, then, potentially five different
gestures which might affect zoom levels, each associated
with \GUI{} controls in different ways: directly
interacting with either a slider or arrow buttons,
or via a context menu (within an image-display), plus
via keyboard actions. \> Such interrelationships
should be tracked with some degree of rigor to
maintain a consistent \q{User Experience}.
\> For example, a common pattern in \GUI{} programming
is top implement \q{tool tips}, or floating text
blurbs that appear when users hover over \GUI{} controls,
explaining the purpose and pragmatics associated
with the control itself. \> Insofar as multiple controls
can be used for image-zoom, each should be
provided tool-tip text accordingly; it would
be useful to confirm that policies along these
links are sustained in an application code-base,
ideally via static code-analysis. \> Important
gesture/feature connections need be programmed
in multiple contexts, apart from the underlying
signal-handlers themselves, including tool-tips,
help-menu information, documentation, application
history and undo/redo capabilities,
notifications (e.g., small labels sometimes
displayed near the bottom of application
windows clarifying recent actions, such as a
string confirming that an image was zoom to a particular
percentage) and unit or integration testing.\;\<
}

\p{\:\+Or, consider again issues with application upgrades.
\> In the example I suggested \visavis{} cloud versus
filesystem saves, there may be multiple controls
and data types associated with the cloud
functionality, including windows where
users register credentials to access a cloud
service and strings giving a remote path
for files on cloud servers \mdash{} each of the
\GUI{} elements presenting the corresponding strings
(path names, users names, passwords, etc.) are
logically interconnected by their common utility
in the guise of cloud file backup. \> Meanwhile, the
\q{image zoom} case can likewise be extended to
hypothetical \q{upgrade} examples: consider the
fact that some image displays use modified
\i{mouse gestures} for zooms, such as cursor up/down
with a key pressed, often the shift key (these
pragmas are borrowed from \ThreeD{} displays, which try
to fit the 6 or 12 degrees of freedom in \ThreeD{} graphics
to conventional mouse and keyboard gestures
\mdash{} rotations; zoom; and translations, i.e. moving
parallel to x, y, or z axes; each of which can
occur within the model or within the \q{camera} \mdash{} using
the keyboard to compensate for mouse-move gestures
have only \i{two} degrees of freedom). \> Consider an
application which decides to support this latter
\ThreeD{}-style zoom gesture added on to prior
functionality; this decision would propagate to
concerns such as those itemized at the end
of last paragraph (test suites, documentation,
help menus, undo/redo, etc.). \> Applications
which tend to be intuitive and responsive
from a User Experience point of view
\mdash{} where it is easy for users to understand
how to initiate their desired actions
by interacting with the software, and
to learn the requisite steps when they
do not know the pragmas ahead to time
\mdash{} consistently model the full
network of interconnections between
application-level capabilities,
\GUI{} display elements, and user pragmatics.\;\<
}

\p{\:\+It is also worth pointing out that such
goals overlap with database engineering.
\> Controls to set zoom levels may be
factored in to a database profile
insofar as the optimal (or most recent)
zoom level for viewing an image might
be stored as one metadata-point in an
image database. \> Similarly, a file's
cloud-hosted save-path would be a relevant
piece of information to track in a database
for which that file is an external resource-object.
\> Cyber-Physical networks belong in the
discussion as well: \CPS{} devices, for
example, tend to have quantitative
profiles (data ranges and measurement units)
which would be relevant both for database
persistence and for \GUI{} admin controls.
\> When a \GUI{} indicator models
\CPS{} sensor readings or actuator settings
(a thermostat's temperature level, say) the
relevant value-range and units
(e.g., Fahrenheit or Celsius) should
be explicated in \GUI{} code (obviously, an
indicator needs to know how each value to
be displayed compares to valid minima/maxima,
and should identify and clarify for
the user, perhaps supporting alternation between,
Fahrenheit/Celsius, or metric/imperial, and so forth).\;\<
}

\p{\:\+User Experience (\UX{}) sometimes appears to be treated
as if it were a stylistic and subjective
dimension to software engineering more
than a technical or mathematical problem, but
effective User Interface design \mdash{} and its
interconnections with such themes as database
engineering and \CPS{} networks \mdash{} hopefully
show that rigorous \UX{} engineering requires
highly structured models of software and \GUI{}
components, as well as application-level data types,
with their interconnections and value-synchronizations.
\> There are various ways to enforce disciplined
engineering standards to sustain quality \UX{},
but at least one set of tools for this
purpose can derive from \VM{} resources
\mdash{} whether in the form of script-like layers
interposed between \GUI{} elements and application
procedures, or runtimes for unit-testing and
prototyping, or representations of application
code for static analysis (or some combination).\;\<
}

\p{\:\+I contend that trends such as Industry 4.0
foretell an increasing convergence between
technologies related to User Experience and
User Interface, Cyber-Physical Systems, \ThreeD{}
graphics, and Requirements Engineering.
\> The forces behind such dynamics reflect
how digitization and computer analytics
has the power to increasingly shapes
industries such as manufacturing, architecture,
green tech, and scientific research. \> Modern graphics
cards can fluidly display interactive \ThreeD{}
models of industrial parts (on a small scale)
to building renderings and urban designs
(on a large scale) and everything in between, allowing
users to visualize planned or manufactured objects/spaces
for design, training, or simulations. \> Digital \q{twins},
moreover, serve as presentations or anchoring-points for
data-aggregates specifying objects' or designs'
requirements and specifications. \> The digital twin
concept implicitly posits relationships
between an object's or design's \ThreeD{} spatial
form and its functional purpose, material/mechanical
properties, fault tolerance, energy consumptions, and
similar usability and quality-assurance metrics.
\> Stakeholders can use software both to form
visual images of materials and spaces as
physical environments and as functionally
organized systems, with software being able
to migrate between more geometric (e.g., \ThreeD{}
rendering) and more data-oriented (e.g., specs-table)
modes. \> Given these application capabilities, digital
platforms can simulate and analyze physical
and/or functional systems with an unprecedented
degree of computational accuracy and, simultaneously,
user interaction, allowing software ecosystems to be
a larger presence in industrial/architectural
(and etc.) planning, construction, and maintenance.\;\<
}

\p{\:\+I do not sketch this account from the
point of view of advocacy, but instead
to point out that maximizing the benefits
of existing digitization capabilities
depends on software which systematically
and interactive unifies visualization
and data curation/analytic aspects
(e.g., \ThreeD{} graphics alongside
specs presentations), which in turn
calls for advanced database and
application-development capabilities,
at least if software fitting these
characterizations is to be developed
in a cost-effective, extensible/adaptable,
and widely available manner. \> I have emphasized
how Virtual Machine engineering can fit into
this overall sea-change in application-development
requirements and standards, with a more
holistic attention to such details
as database query-processing and \GUI{}
design than were, arguably, emphasized
by earlier \VM{} paradigms.\;\<
}
