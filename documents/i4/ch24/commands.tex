
\newcommand{\pseudoIndent}{

\vspace{10pt}\hspace*{12pt}}


\renewcommand{\p}[1]{

\vspace{.7em}#1}

\usepackage{graphicx}
\usepackage{xcolor}

\setlength{\columnsep}{8.55mm}

\usepackage[letterpaper, left=.85in,right=.85in,top=.5in,bottom=.5in,
paperheight=13.6in,paperwidth=10.8in]{geometry}


\usepackage{enumitem}

\setlist[description]{leftmargin=4pt}

\setlength{\footskip}{20pt}

%\let\OldI\i
%\renewcommand{\i}[1]{\textit{#1}}


\newcommand{\mdash}{---}
\newcommand{\q}[1]{{\fontfamily{phv}\selectfont ``}#1{\fontfamily{phv}\selectfont ''}} 
\newcommand{\sq}[1]{{\fontfamily{phv}\selectfont `}#1{\fontfamily{phv}\selectfont '}} 

\newcommand{\qq}[1]{{\fontfamily{phv}\selectfont `}#1{\fontfamily{phv}\selectfont '}} 

\newcommand{\JaneJohn}{\i{Jane}\hspace{4pt}and\hspace{3pt}\i{John}}


\usepackage{setspace}

\colorlet{codegr}{black!80!blue}


\let\OldFootnoteSize\footnotesize
\renewcommand{\footnotesize}{\scriptsize}

\newif\iffootnote
\let\Footnote\footnote
\renewcommand\footnote[1]{\begingroup\footnotetrue\Footnote{#1}\endgroup}

\newcommand{\AcronymText}[1]{{\iffootnote\begin{footnotesize}{\textsc{#1}}\end{footnotesize}%
\else\begin{OldFootnoteSize}{\textsc{#1}}\end{OldFootnoteSize}\fi}}

\newcommand{\visavis}{vis-\`a-vis}

\newcommand{\acronymResizeBox}[1]{#1}

\newcommand{\Cpp}{\AcronymText{C++}}

\newcommand{\MRI}{\AcronymText{MRI}}
\newcommand{\fMRI}{f\AcronymText{MRI}}

\newcommand{\CPU}{\AcronymText{CPU}}

\newcommand{\lIFC}{\AcronymText{{\large{I}}FC}}

\newcommand{\IFC}{\AcronymText{IFC}}
\newcommand{\AEC}{\AcronymText{AEC}}

\newcommand{\CAD}{\AcronymText{CAD}}

\newcommand{\ThreeD}{\AcronymText{3D}}
\newcommand{\TwoD}{\AcronymText{2D}}

\newcommand{\NLP}{\AcronymText{NLP}}

\newcommand{\AAC}{\AcronymText{AAC}}
\newcommand{\GUI}{\AcronymText{GUI}}

\newcommand{\communique}{communiqu\'e}

\newcommand{\HRI}{\AcronymText{HRI}}




\newcommand{\qquadrantcodeagainst}{\q{quadrant\_code\_against}}

\newcommand{\quadrantcodeagainstimpl}{\begin{alltt}u1 quadrant_code_against(const ty##size##s& rhs) const { return (*this - rhs).spaceship_mask().plus((*this - rhs).zeros_mask()).floor(0).times({2, 1}).inner_sum(); }\end{alltt}}

\newcommand{\xypair}{\ensuremath{x, y}}
\newcommand{\xypluss}{\ensuremath{x + s, y + s}}
\newcommand{\abpair}{\ensuremath{a, b}}


\newcommand{\sval}

\newcommand{\xyplusab}{\ensuremath{x + a, y + b}}
\newcommand{\xplusy}{\ensuremath{x + y}}

\newcommand{\inum}{\ensuremath{i}}

\newcommand{\topbarleft}{\b{top} $|$ \b{left}}

\newcommand{\bardash}{\ensuremath{|-}}
\newcommand{\dashbar}{\ensuremath{-|}}




%fRed

\newcommand{\iGUI}{\textit{{\small{}GUI}}}


\usepackage{microtype}

\newcommand{\biburl}[1]{ {\fontfamily{gar}\selectfont{{\scriptsize \textls*[-70]{\url{#1}}}}}}

\usepackage[colorlinks=true]{hyperref}

\colorlet{urlclr}{red!10!magenta!70!orange}

\hypersetup{
 urlcolor = urlclr!50!black,
 urlbordercolor = cyan!60!black,
 linkcolor = red!30!black,
 citecolor = orange!30!black,
 citebordercolor = yellow!30!black,
} 


\raggedbottom

%\newcommand{\defin}[2]{\begin{definition}\textit{#1}: #2\end{definition}}

\newcommand{\bibtitle}[1]{{\footnotesize{}\textit{#1}}}
\newcommand{\intitle}[1]{{\hspace{3pt}\textls*[-80]{\texttt{\textit{#1}}}}\hspace{-1pt}}


\renewcommand{\b}[1]{{\small\textbf{#1}}}

\let\OldI\i
\renewcommand{\i}[1]{\textit{#1}}





\newcommand{\eacc}{\'e}
